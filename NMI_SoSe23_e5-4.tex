\documentclass[a4paper,12pt]{article}
\usepackage{fancyhdr}
\usepackage[ngerman,german]{babel}
\usepackage{german}
\usepackage[utf8]{inputenc}
\usepackage[active]{srcltx}
\usepackage{algorithm}
\usepackage[noend]{algorithmic}
\usepackage{amsmath}
\usepackage{amssymb}
\usepackage{amsthm}
\usepackage{bbm}
\usepackage{enumerate}
\usepackage{graphicx}
\usepackage{ifthen}
\usepackage{listings}
\usepackage{struktex}
\usepackage{hyperref}
\usepackage[onehalfspacing]{setspace}
\usepackage{geometry}
\usepackage{calc}

%%%%%%%%%%%%%%%%%%%%%%%%%%%%%%%%%%%%%%%%%%%%%%%%%%%%%%
%%%%%%%%%%%%%% EDIT THIS PART %%%%%%%%%%%%%%%%%%%%%%%%
%%%%%%%%%%%%%%%%%%%%%%%%%%%%%%%%%%%%%%%%%%%%%%%%%%%%%%

\newcommand{\Fach}{01270 Numerische Mathematik I}
\newcommand{\FachKurz}{NMI}
\newcommand{\Name}{Oliver Schwarz}
\newcommand{\Matrikelnummer}{6883389}
\newcommand{\Semester}{SoSe 23}
\newcommand{\Kurseinheit}{5}
\newcommand{\AufgabeNummer}{4}

%%%%%%%%%%%%%%%%%%%%%%%%%%%%%%%%%%%%%%%%%%%%%%%%%%%%%%
%%%%%%%%%%%%%% DO NOT EDIT THIS PART %%%%%%%%%%%%%%%%%
%%%%%%%%%%%%%%%%%%%%%%%%%%%%%%%%%%%%%%%%%%%%%%%%%%%%%%

\newcommand{\Aufgabe}[1]{
  {
  \vspace*{0.5cm}
  \textbf{Aufgabe #1}
  \vspace*{0.2cm}
  }
}

\newcommand{\cond}[0]{
  {
  \textrm{cond}
  }
}

% Beschreibung
% Unterscheidet im Mathe-Mode zwischen
% "Dezimal-komma" und Satzzeichen.
% Muss bei Eingabe durch kein/ein Leerzeichen
% dahinter angegeben werden.
% Literatur Richard Hirsch in: DTK, 1/1994, S. 42 ff.
\mathchardef\CommaOrdinary="013B \mathchardef\CommaPunct="613B
\mathcode`,="8000 % , im Math-Mode aktiv ("8000) machen
{\catcode`\,=\active \gdef
,{\obeyspaces\futurelet\next\CommaCheck}}
\def\CommaCheck{\if\space\next\CommaPunct\else\CommaOrdinary\fi}

\newcommand*\diff{\mathop{}\!\mathrm{\,d}}

\newcommand*\euler{\mathrm{e}}

\newcommand\redarrow{%
        \mathrel{\vcenter{\mathsurround0pt
                \ialign{##\crcr
                        \noalign{\nointerlineskip}$\searrow\vspace{2.5pt}$\crcr 
                        \noalign{\nointerlineskip}$\rightarrow$\crcr
                        \noalign{\nointerlineskip}$\vspace{12.5pt}$\crcr
                }%
        }}%
}
%%%%%%%%%%%%%%%%%%%%%%%%%%%%%%%%%%%%%%%%%%%%%%%%%%%%%%
%%%%%%%%%%%%%% PAGE SETTINGS %%%%%%%%%%%%%%%%%%%%%%%%%
%%%%%%%%%%%%%%%%%%%%%%%%%%%%%%%%%%%%%%%%%%%%%%%%%%%%%%

\setlength{\parindent}{0em}
\topmargin 0cm
\oddsidemargin 0cm
\evensidemargin 0cm

\geometry{%
  left=45.0mm,
  right=15.0mm,
  top=25mm,
  bottom=25mm,
  bindingoffset=0mm,
  headheight=30pt,
  includehead
}

\fancyheadoffset[L]{20mm}
\renewcommand{\headrulewidth}{1pt}

\pagestyle{fancy}

%%%%%%%%%%%%%%%%%%%%%%%%%%%%%%%%%%%%%%%%%%%%%%%%%%%%%%
%%%%%%%%%%%%%% PDF SETTINGS %%%%%%%%%%%%%%%%%%%%%%%%%%
%%%%%%%%%%%%%%%%%%%%%%%%%%%%%%%%%%%%%%%%%%%%%%%%%%%%%%

\hypersetup{
    %pdftitle={\Fach{}: Übungsblatt \Uebungsblatt{}},
    pdftitle={\Fach{}: Kurseinheit \Kurseinheit{} Aufgabe \AufgabeNummer{}},
    pdfauthor={\Name},
    pdfborder={0 0 0}
}

%%%%%%%%%%%%%%%%%%%%%%%%%%%%%%%%%%%%%%%%%%%%%%%%%%%%%%
%%%%%%%%%%%%%% CODE SETTINGS %%%%%%%%%%%%%%%%%%%%%%%%%
%%%%%%%%%%%%%%%%%%%%%%%%%%%%%%%%%%%%%%%%%%%%%%%%%%%%%%

\lstset{ %
language=java,
basicstyle=\footnotesize\tt,
showtabs=false,
tabsize=2,
captionpos=b,
breaklines=true,
extendedchars=true,
showstringspaces=false,
flexiblecolumns=true,
}

%%%%%%%%%%%%%%%%%%%%%%%%%%%%%%%%%%%%%%%%%%%%%%%%%%%%%%
%%%%%%%%%%%%%% DOCUMENT %%%%%%%%%%%%%%%%%%%%%%%%%%%%%%
%%%%%%%%%%%%%%%%%%%%%%%%%%%%%%%%%%%%%%%%%%%%%%%%%%%%%%

\title{Kurseinheit \Kurseinheit{} Aufgabe \AufgabeNummer{}}
\author{\Name{}}

\begin{document}

%\renewcommand{\theequation}{L\Kurseinheit{}.\AufgabeNummer{}.\arabic{equation}}
\renewcommand{\theequation}{L \AufgabeNummer{}.\arabic{equation}}

%%%%%%%%%%%%%%%%%%%%%%%%%%%%%%%%%%%%%%%%%%%%%%%%%%%%%%
%%%%%%%%%%%%%% HEADER %%%%%%%%%%%%%%%%%%%%%%%%%%%%%%%%
%%%%%%%%%%%%%%%%%%%%%%%%%%%%%%%%%%%%%%%%%%%%%%%%%%%%%%

\lhead{\sf \large \Fach{} \\ \small \Name{} - \Matrikelnummer{}}
%\rhead{\sf \Semester{}}
\rhead{\sf \FachKurz{} \quad E \Kurseinheit{}/\AufgabeNummer{}}

%%%%%%%%%%%%%%%%%%%%%%%%%%%%%%%%%%%%%%%%%%%%%%%%%%%%%%
%%%%%%%%%%%%%% START HERE %%%%%%%%%%%%%%%%%%%%%%%%%%%%
%%%%%%%%%%%%%%%%%%%%%%%%%%%%%%%%%%%%%%%%%%%%%%%%%%%%%%

\Aufgabe{\AufgabeNummer{}}
\begin{enumerate}[a)]
    \item %a)
Es ist
$$\Delta_n^{\shortparallel} := \left\{x_k : x_k = \frac{k}{n}, \; k= 0, \dots, n \right\}$$
eine Menge äquidistanter Stützstellen auf $[0, 1]$. $I_{\Delta_n^{\shortparallel}}f \in \Pi_n$ ist das mit den Stützstellen aus $\Delta_n^{\shortparallel}$ gebildete Interpolationspolynom zu $f$.

Nach Satz 6.5.1 existiert zu jedem $x \in [0,1]$ ein $\xi = \xi(x) \in (0,1)$ mit
\begin{equation}
    f(x) - I\Delta_n^{\shortparallel}f(x) = \frac{f^{(n+1)}(\xi)}{(n+1)!}\omega_{n+1}(x), \label{eqn_4-1}
\end{equation}
und dem Knotenpolynom $$\omega_{n+1}(x) = (x - x_0)(x - x_1) \cdots (x - x_n) = \prod\limits_{k=0}^{n}\left(x-\frac{k}{n}\right).$$

Äquivalent zum Beispiel 6.5.2 schätzen wir $|| \omega_{n+1} ||_{[0,1]}$ ab und nutzen dabei das Hilfspolynom 
\begin{align*}
    \psi(x) &:= \left(x-\frac{k}{n}\right) \left(x-\frac{n-k}{n}\right) \textrm{ mit} \\
    ||\psi||_{[0,1]} &\leq \frac{1}{4} = 2^{-2}, \; x \in [0,1], \; 0 \leq k \leq n.
\end{align*}

Es ist $$\omega_{n+1}(x) = \prod\limits_{k=0}^{n}\left(x-\frac{k}{n}\right) = \prod\limits_{k=0}^{m}\left(x-\frac{k}{n}\right)\left(x-\frac{n-k}{n}\right) = \prod\limits_{k=0}^{m}\psi(x).$$

Hiermit können wir $|| \omega_{n+1} ||_{[0,1]}$ abschätzen:
$$|| \omega_{n+1} ||_{[0,1]} \leq 2^{-(n+1)}.$$

Mit Satz 6.5.2 bzw. \ref{eqn_4-1} erhalten wir die Behauptung
$$||f - I\Delta_n^{\shortparallel}f ||_{[0,1]} \leq \frac{||f^{(n+1)}||_{[0,1]}}{2^{(n+1)}(n+1)!}.$$

\item %b)
Es ist
$$\Delta_n^{T} := \left\{x^*_k : k= 0, \dots, n \right\}$$
eine Menge von Stützstellen auf $[0,1]$ und $I_{\Delta_n^{T}}f \in \Pi_n$ ist das mit den Stützstellen aus $\Delta_n^{T}$ gebildete Interpolationspolynom zu $f$.

Analog zu \ref{eqn_4-1} existiert nach Satz 6.5.1 zu jedem $x \in [0,1]$ ein $\xi = \xi(x) \in (0,1)$ mit
\begin{equation}
    f(x) - I\Delta_n^{T}f(x) = \frac{f^{(n+1)}(\xi)}{(n+1)!}\omega_{n+1}(x), \label{eqn_4-2}
\end{equation}
und dem Knotenpolynom $$\omega_{n+1}(x) = (x - x^*_0)(x - x^*_1) \cdots (x - x^*_n) = \prod\limits_{k=0}^{n}(x-x^*_k).$$

Um den Interpolationsfehler abschätzen zu können, müssen wir wiederum $||\omega_{n+1}||_{[0,1]}$ abschätzen.

Wenn wir nun mit $\varphi : [-1, 1] \rightarrow [0,1] : x \mapsto \displaystyle\frac{1}{2} + \displaystyle\frac{1}{2} x$ die Nullstellen der \textsc{Tschebyscheff}-Polynome transformieren, erhalten wir
\begin{align*}
    \omega_{n+1}(x) &= \prod\limits_{k=0}^{n}\left(\frac{1}{2} + \frac{1}{2} x- \frac{1}{2} - \frac{1}{2} x^*_k\right) \\
    &= \prod\limits_{k=0}^{n}\left(\frac{1}{2} (x- x_k)\right) = 2^{-(n+1)} \underbrace{\prod\limits_{k=0}^{n} (x- x_k)}_{= 2^{-n} T_{n+1}(x)} \\
    \omega_{n+1}(x) &= 2^{-(2n+1)}  T_{n+1}(x).
\end{align*}

Mit Bemerkung 6.9.3 erhalten wir dann die Behauptung
\begin{equation*}
    ||f - I\Delta_n^{T}f ||_{[0,1]} \leq \frac{||f^{(n+1)}||_{[0,1]}}{2^{2n+1}(n+1)!}.
\end{equation*}

\end{enumerate}


\end{document}
