\documentclass[a4paper,12pt]{article}
\usepackage{fancyhdr}
\usepackage[ngerman,german]{babel}
\usepackage{german}
\usepackage[utf8]{inputenc}
\usepackage[active]{srcltx}
\usepackage{algorithm}
\usepackage[noend]{algorithmic}
\usepackage{amsmath}
\usepackage{amssymb}
\usepackage{amsthm}
\usepackage{bbm}
\usepackage{enumerate}
\usepackage{graphicx}
\usepackage{ifthen}
\usepackage{listings}
\usepackage{struktex}
\usepackage{hyperref}
\usepackage[onehalfspacing]{setspace}
\usepackage{geometry}
\usepackage{calc}

%%%%%%%%%%%%%%%%%%%%%%%%%%%%%%%%%%%%%%%%%%%%%%%%%%%%%%
%%%%%%%%%%%%%% EDIT THIS PART %%%%%%%%%%%%%%%%%%%%%%%%
%%%%%%%%%%%%%%%%%%%%%%%%%%%%%%%%%%%%%%%%%%%%%%%%%%%%%%

\newcommand{\Fach}{01270 Numerische Mathematik I}
\newcommand{\FachKurz}{NMI}
\newcommand{\Name}{Oliver Schwarz}
\newcommand{\Matrikelnummer}{6883389}
\newcommand{\Semester}{SoSe 23}
\newcommand{\Kurseinheit}{2}
\newcommand{\AufgabeNummer}{1}

%%%%%%%%%%%%%%%%%%%%%%%%%%%%%%%%%%%%%%%%%%%%%%%%%%%%%%
%%%%%%%%%%%%%% DO NOT EDIT THIS PART %%%%%%%%%%%%%%%%%
%%%%%%%%%%%%%%%%%%%%%%%%%%%%%%%%%%%%%%%%%%%%%%%%%%%%%%

\newcommand{\Aufgabe}[1]{
  {
  \vspace*{0.5cm}
  \textbf{Aufgabe #1}
  \vspace*{0.2cm}
  }
}

%%%%%%%%%%%%%%%%%%%%%%%%%%%%%%%%%%%%%%%%%%%%%%%%%%%%%%
%%%%%%%%%%%%%% PAGE SETTINGS %%%%%%%%%%%%%%%%%%%%%%%%%
%%%%%%%%%%%%%%%%%%%%%%%%%%%%%%%%%%%%%%%%%%%%%%%%%%%%%%

\setlength{\parindent}{0em}
\topmargin 0cm
\oddsidemargin 0cm
\evensidemargin 0cm

\geometry{%
  left=45.0mm,
  right=15.0mm,
  top=25mm,
  bottom=25mm,
  bindingoffset=0mm,
  headheight=30pt,
  includehead
}

\fancyheadoffset[L]{20mm}
\renewcommand{\headrulewidth}{1pt}

\pagestyle{fancy}

%%%%%%%%%%%%%%%%%%%%%%%%%%%%%%%%%%%%%%%%%%%%%%%%%%%%%%
%%%%%%%%%%%%%% PDF SETTINGS %%%%%%%%%%%%%%%%%%%%%%%%%%
%%%%%%%%%%%%%%%%%%%%%%%%%%%%%%%%%%%%%%%%%%%%%%%%%%%%%%

\hypersetup{
    %pdftitle={\Fach{}: Übungsblatt \Uebungsblatt{}},
    pdftitle={\Fach{}: Kurseinheit \Kurseinheit{} Aufgabe \AufgabeNummer{}},
    pdfauthor={\Name},
    pdfborder={0 0 0}
}

%%%%%%%%%%%%%%%%%%%%%%%%%%%%%%%%%%%%%%%%%%%%%%%%%%%%%%
%%%%%%%%%%%%%% CODE SETTINGS %%%%%%%%%%%%%%%%%%%%%%%%%
%%%%%%%%%%%%%%%%%%%%%%%%%%%%%%%%%%%%%%%%%%%%%%%%%%%%%%

\lstset{ %
language=java,
basicstyle=\footnotesize\tt,
showtabs=false,
tabsize=2,
captionpos=b,
breaklines=true,
extendedchars=true,
showstringspaces=false,
flexiblecolumns=true,
}

%%%%%%%%%%%%%%%%%%%%%%%%%%%%%%%%%%%%%%%%%%%%%%%%%%%%%%
%%%%%%%%%%%%%% DOCUMENT %%%%%%%%%%%%%%%%%%%%%%%%%%%%%%
%%%%%%%%%%%%%%%%%%%%%%%%%%%%%%%%%%%%%%%%%%%%%%%%%%%%%%

\title{Kurseinheit \Kurseinheit{} Aufgabe \AufgabeNummer{}}
\author{\Name{}}

\begin{document}

%\renewcommand{\theequation}{L\Kurseinheit{}.\AufgabeNummer{}.\arabic{equation}}
\renewcommand{\theequation}{L \AufgabeNummer{}.\arabic{equation}}

%%%%%%%%%%%%%%%%%%%%%%%%%%%%%%%%%%%%%%%%%%%%%%%%%%%%%%
%%%%%%%%%%%%%% HEADER %%%%%%%%%%%%%%%%%%%%%%%%%%%%%%%%
%%%%%%%%%%%%%%%%%%%%%%%%%%%%%%%%%%%%%%%%%%%%%%%%%%%%%%

\lhead{\sf \large \Fach{} \\ \small \Name{} - \Matrikelnummer{}}
%\rhead{\sf \Semester{}}
\rhead{\sf \FachKurz{} \quad E \Kurseinheit{}/\AufgabeNummer{}}

%%%%%%%%%%%%%%%%%%%%%%%%%%%%%%%%%%%%%%%%%%%%%%%%%%%%%%
%%%%%%%%%%%%%% START HERE %%%%%%%%%%%%%%%%%%%%%%%%%%%%
%%%%%%%%%%%%%%%%%%%%%%%%%%%%%%%%%%%%%%%%%%%%%%%%%%%%%%

\Aufgabe{\AufgabeNummer{}}

Wir betrachten das \textsc{Gram-Schmidt}-Verfahren und das modifizierte \textsc{Gram-Schmidt}-Verfahren. (Wir verwenden $\breve{\pmb{u}}^i$ statt $\tilde{\pmb{u}}^i$ im modifizierten \textsc{Gram-Schmidt}-Verfahren. Analog zum modifizierten Verfahren in 2.3.10 wurde im Algorithmus 2.3.8 die Summe durch eine Schleife ersetzt.)


\textsc{Gram-Schmidt}-Verfahren:
\begin{align*}
\text{S1.} \quad & \pmb{u}^1 := \frac{\pmb{x}^1}{||\pmb{x}^1||} \\
\text{S2.} \quad & \textsc{for} \quad i = 2, 3, \dots , k \quad \textsc{do} \\
                  & \qquad \tilde{\pmb{u}}^{i,1} := \pmb{x}^i \\
                  & \qquad \textsc{for} \quad j = 1, 2, \dots , i-1 \quad \textsc{do} \\
                  & \qquad \qquad \tilde{\pmb{u}}^{i, j+1} := \tilde{\pmb{u}}^{i,j} - \langle \pmb{x}^i, \pmb{u}^j \rangle \pmb{u}^j \\
                  & \qquad \textsc{end do} \\
                  & \qquad \tilde{\pmb{u}}^i := \tilde{\pmb{u}}^{i,i} \\
                  & \qquad \pmb{u}^i := \frac{\tilde{\pmb{u}}^i}{|| \tilde{\pmb{u}}^i ||} \\
                  & \textsc{end do}
\end{align*}


Modifiziertes \textsc{Gram-Schmidt}-Verfahren:
\begin{align*}
\text{S1.} \quad & \pmb{u}^1 := \frac{\pmb{x}^1}{||\pmb{x}^1||} \\
\text{S2.} \quad & \textsc{for} \quad i = 2, 3, \dots , k \quad \textsc{do} \\
                  & \qquad \breve{\pmb{u}}^{i,1} := \pmb{x}^i \\
                  & \qquad \textsc{for} \quad j = 1, 2, \dots , i-1 \quad \textsc{do} \\
                  & \qquad \qquad \breve{\pmb{u}}^{i, j+1} := \breve{\pmb{u}}^{i,j} - \langle \pmb{u}^{i,j}, \pmb{u}^j \rangle \pmb{u}^j \\
                  & \qquad \textsc{end do} \\
                  & \qquad \breve{\pmb{u}}^i := \breve{\pmb{u}}^{i,i} \\
                  & \qquad \pmb{u}^i := \frac{\breve{\pmb{u}}^i}{|| \breve{\pmb{u}}^i ||} \\
                  & \textsc{end do}
\end{align*}


Die beiden Verfahren unterscheiden sich in der inneren Schleife:
\begin{align}
    \text{(\textsc{Gram-Schmidt}-Verfahren)} \quad \tilde{\pmb{u}}^{i, j+1} & := \tilde{\pmb{u}}^{i,j} - \langle \pmb{x}^i, \pmb{u}^j \rangle \pmb{u}^j \label{eq:formel1}\\
    \text{(Modifiziertes \textsc{Gram-Schmidt}-Verfahren)} \quad \breve{\pmb{u}}^{i, j+1} & := \breve{\pmb{u}}^{i,j} - \langle \pmb{u}^{i,j}, \pmb{u}^j \rangle \pmb{u}^j \label{eq:formel2}
\end{align}

Wenn beide Verfahren dieselben orthonormalen Vektoren berechnen, muss $\tilde{\pmb{u}}^{i, j} = \breve{\pmb{u}}^{i, j}$ gelten.

Die beiden Formeln \ref{eq:formel1} und \ref{eq:formel2} unterscheiden sich in ihren Linearfaktoren und genügen der Berechnungsvorschrift:
\begin{equation}
    \pmb{\tilde{u}}^{i,j} = \pmb{x}^i - \sum_{k=1}^{j-1} \lambda_k \pmb{u}^k \label{eq:formel3}
\end{equation}

Wir multiplizieren \ref{eq:formel3} mit $\pmb{u}^j$ und erhalten:
\begin{equation*}
    \langle \pmb{u}^{i,j}, \pmb{u}^j \rangle = \langle \pmb{x}^i, \pmb{u}^j \rangle - \sum_{k=1}^{j-1} \lambda_k\langle \pmb{u}^{k}, \pmb{u}^j \rangle = \langle \pmb{x}^i, \pmb{u}^j \rangle,
\end{equation*}
da $\pmb{u}^k \perp \pmb{u}^j$ für alle $k < j$ sind.

Daraus folgt, dass mit $\pmb{x}^i = \tilde{\pmb{u}}^{i,1} = \breve{\pmb{u}}^{i,1}$ alle $\tilde{\pmb{u}}^{i,j} = \breve{\pmb{u}}^{i,j}$ sind und somit auch $\tilde{\pmb{u}}^{i} = \breve{\pmb{u}}^{i}$.

\end{document}
