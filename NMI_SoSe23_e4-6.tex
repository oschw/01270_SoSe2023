\documentclass[a4paper,12pt]{article}
\usepackage{fancyhdr}
\usepackage[ngerman,german]{babel}
\usepackage{german}
\usepackage[utf8]{inputenc}
\usepackage[active]{srcltx}
\usepackage{algorithm}
\usepackage[noend]{algorithmic}
\usepackage{amsmath}
\usepackage{amssymb}
\usepackage{amsthm}
\usepackage{bbm}
\usepackage{enumerate}
\usepackage{graphicx}
\usepackage{ifthen}
\usepackage{listings}
\usepackage{struktex}
\usepackage{hyperref}
\usepackage[onehalfspacing]{setspace}
\usepackage{geometry}
\usepackage{calc}

%%%%%%%%%%%%%%%%%%%%%%%%%%%%%%%%%%%%%%%%%%%%%%%%%%%%%%
%%%%%%%%%%%%%% EDIT THIS PART %%%%%%%%%%%%%%%%%%%%%%%%
%%%%%%%%%%%%%%%%%%%%%%%%%%%%%%%%%%%%%%%%%%%%%%%%%%%%%%

\newcommand{\Fach}{01270 Numerische Mathematik I}
\newcommand{\FachKurz}{NMI}
\newcommand{\Name}{Oliver Schwarz}
\newcommand{\Matrikelnummer}{6883389}
\newcommand{\Semester}{SoSe 23}
\newcommand{\Kurseinheit}{4}
\newcommand{\AufgabeNummer}{6}

%%%%%%%%%%%%%%%%%%%%%%%%%%%%%%%%%%%%%%%%%%%%%%%%%%%%%%
%%%%%%%%%%%%%% DO NOT EDIT THIS PART %%%%%%%%%%%%%%%%%
%%%%%%%%%%%%%%%%%%%%%%%%%%%%%%%%%%%%%%%%%%%%%%%%%%%%%%

\newcommand{\Aufgabe}[1]{
  {
  \vspace*{0.5cm}
  \textbf{Aufgabe #1}
  \vspace*{0.2cm}
  }
}

\newcommand{\cond}[0]{
  {
  \textrm{cond}
  }
}

% Beschreibung
% Unterscheidet im Mathe-Mode zwischen
% "Dezimal-komma" und Satzzeichen.
% Muss bei Eingabe durch kein/ein Leerzeichen
% dahinter angegeben werden.
% Literatur Richard Hirsch in: DTK, 1/1994, S. 42 ff.
\mathchardef\CommaOrdinary="013B \mathchardef\CommaPunct="613B
\mathcode`,="8000 % , im Math-Mode aktiv ("8000) machen
{\catcode`\,=\active \gdef
,{\obeyspaces\futurelet\next\CommaCheck}}
\def\CommaCheck{\if\space\next\CommaPunct\else\CommaOrdinary\fi}

\newcommand*\diff{\mathop{}\!\mathrm{\,d}}
%%%%%%%%%%%%%%%%%%%%%%%%%%%%%%%%%%%%%%%%%%%%%%%%%%%%%%
%%%%%%%%%%%%%% PAGE SETTINGS %%%%%%%%%%%%%%%%%%%%%%%%%
%%%%%%%%%%%%%%%%%%%%%%%%%%%%%%%%%%%%%%%%%%%%%%%%%%%%%%

\setlength{\parindent}{0em}
\topmargin 0cm
\oddsidemargin 0cm
\evensidemargin 0cm

\geometry{%
  left=45.0mm,
  right=15.0mm,
  top=25mm,
  bottom=25mm,
  bindingoffset=0mm,
  headheight=30pt,
  includehead
}

\fancyheadoffset[L]{20mm}
\renewcommand{\headrulewidth}{1pt}

\pagestyle{fancy}

%%%%%%%%%%%%%%%%%%%%%%%%%%%%%%%%%%%%%%%%%%%%%%%%%%%%%%
%%%%%%%%%%%%%% PDF SETTINGS %%%%%%%%%%%%%%%%%%%%%%%%%%
%%%%%%%%%%%%%%%%%%%%%%%%%%%%%%%%%%%%%%%%%%%%%%%%%%%%%%

\hypersetup{
    %pdftitle={\Fach{}: Übungsblatt \Uebungsblatt{}},
    pdftitle={\Fach{}: Kurseinheit \Kurseinheit{} Aufgabe \AufgabeNummer{}},
    pdfauthor={\Name},
    pdfborder={0 0 0}
}

%%%%%%%%%%%%%%%%%%%%%%%%%%%%%%%%%%%%%%%%%%%%%%%%%%%%%%
%%%%%%%%%%%%%% CODE SETTINGS %%%%%%%%%%%%%%%%%%%%%%%%%
%%%%%%%%%%%%%%%%%%%%%%%%%%%%%%%%%%%%%%%%%%%%%%%%%%%%%%

\lstset{ %
language=java,
basicstyle=\footnotesize\tt,
showtabs=false,
tabsize=2,
captionpos=b,
breaklines=true,
extendedchars=true,
showstringspaces=false,
flexiblecolumns=true,
}

%%%%%%%%%%%%%%%%%%%%%%%%%%%%%%%%%%%%%%%%%%%%%%%%%%%%%%
%%%%%%%%%%%%%% DOCUMENT %%%%%%%%%%%%%%%%%%%%%%%%%%%%%%
%%%%%%%%%%%%%%%%%%%%%%%%%%%%%%%%%%%%%%%%%%%%%%%%%%%%%%

\title{Kurseinheit \Kurseinheit{} Aufgabe \AufgabeNummer{}}
\author{\Name{}}

\begin{document}

%\renewcommand{\theequation}{L\Kurseinheit{}.\AufgabeNummer{}.\arabic{equation}}
\renewcommand{\theequation}{L \AufgabeNummer{}.\arabic{equation}}

%%%%%%%%%%%%%%%%%%%%%%%%%%%%%%%%%%%%%%%%%%%%%%%%%%%%%%
%%%%%%%%%%%%%% HEADER %%%%%%%%%%%%%%%%%%%%%%%%%%%%%%%%
%%%%%%%%%%%%%%%%%%%%%%%%%%%%%%%%%%%%%%%%%%%%%%%%%%%%%%

\lhead{\sf \large \Fach{} \\ \small \Name{} - \Matrikelnummer{}}
%\rhead{\sf \Semester{}}
\rhead{\sf \FachKurz{} \quad E \Kurseinheit{}/\AufgabeNummer{}}

%%%%%%%%%%%%%%%%%%%%%%%%%%%%%%%%%%%%%%%%%%%%%%%%%%%%%%
%%%%%%%%%%%%%% START HERE %%%%%%%%%%%%%%%%%%%%%%%%%%%%
%%%%%%%%%%%%%%%%%%%%%%%%%%%%%%%%%%%%%%%%%%%%%%%%%%%%%%

\Aufgabe{\AufgabeNummer{}}

Wir betrachten die \textsc{Tschebyscheff}-Polynome auf $\mathbb{R}$ für $x \in [-1, 1]$ nach Satz 5.2.3.

Offensichtlich ist $\langle U_i, U_j \rangle_{\varrho_U} = 0$ für $|x| = 1$. Also beschränken wir uns mit $x \in (-1, 1)$ auf die Darstellung $$U_n(x) = \frac{\sin{((n+1) \arccos{x})}}{\sin{(\arccos{x})}}.$$

Damit ist
\begin{align*}
    \langle U_i, U_j \rangle_{\varrho_U} &:= \int\limits_{-1}^{1} U_i(x)U_j(x)\sqrt{1-x^2} \diff x \\
    &= \int\limits_{-1}^{1} \frac{\sin{((i+1) \arccos{x})}}{\sin{(\arccos{x})}} \cdot \frac{\sin{((j+1) \arccos{x})}}{\sin{(\arccos{x})}} \cdot \sqrt{1-x^2} \diff x %\\
    %&= \int\limits_{-1}^{1} \frac{\sin{((i+1) \arccos{x})} \cdot \sin{((j+1) \arccos{x})}}{\sin^2{(\arccos{x})}} \sqrt{1-x^2} \diff x \\
    %&= \int\limits_{-1}^{1} \frac{\sin{((i+1) \arccos{x})} \cdot \sin{((j+1) \arccos{x})}}{1-x^2} \sqrt{1-x^2} \diff x
\end{align*}

Wir substituieren im Integranden $u := \arccos{x}$ und $\diff u = - (\sqrt{1-x^2})^{-1} \diff x$.

Da im Integranden $\sqrt{1-x^2} \diff x$ vorkommt und mit $\sin{(\arccos{x})} = \sqrt{1-x^2}$, lohnt sich die Umformung
$$\sqrt{1-x^2} \diff x = -(\sqrt{1-x^2})^2 \diff u = - \sin^2{(\arccos{x})} \diff u = - \sin^2{u} \diff u.$$

Wir passen die Integrationsgrenzen an und erhalten
\begin{align}
    \langle U_i, U_j \rangle_{\varrho_U} &= \int\limits_{\pi}^{0} \frac{\sin{(i+1)u} \cdot \sin{(j+1)u}}{\sin^2{u}} \cdot - \sin^2{u} \diff u \nonumber \\
    &= - \int\limits_{\pi}^{0}\sin{(i+1)u} \cdot \sin{(j+1)u} \diff u  \label{eqn}
\end{align}

Für den Fall $i = j$ erhalten wir aus \ref{eqn}
\begin{align*}
    \langle U_i, U_i \rangle_{\varrho_U} &= - \int\limits_{\pi}^{0}\sin^2{(i+1)u}  \diff u
\end{align*}

Und mit der Substitution $t := (i+1)u$ sowie $\diff t = (i+1) \diff u$ ist
\begin{align*}
    \langle U_i, U_i \rangle_{\varrho_U} &= - \frac{1}{i+1} \int\limits_{\pi (i+1)}^{0}\sin^2{t}  \diff t \\
    &= - \frac{1}{i+1} \left( \frac{t}{2} - \frac{1}{4} \sin(2t) \right) \Bigg |_{\pi (i + 1)}^{0} \\
    &= \frac{1}{i+1} \left( \frac{\pi (i+1)}{2} - \frac{1}{4} \sin(2 \pi (i+1)) \right) \\
    \langle U_i, U_i \rangle_{\varrho_U} &= \frac{\pi}{2}. \, &\Box
\end{align*}

Für den Fall $j \neq i$ ist \ref{eqn} mit $a := (i+1)u$ und $b := (j+1)u$ $$\sin(a) \sin(b) = \frac{1}{2}(\cos(a-b) - \cos(a+b)) \textrm{, also}$$ 
\begin{align*}
    \langle U_i, U_j \rangle_{\varrho_U} &= - \frac{1}{2} \int\limits_{\pi}^{0} \cos(iu - ju) - \cos(iu + ju + 2u) \diff u \\
    &= \frac{1}{2} \int\limits_{\pi}^{0} \cos(iu + ju + 2u) \diff u - \frac{1}{2} \int\limits_{\pi}^{0} \cos(iu - ju) \diff u
\end{align*}

Die Substitution im ersten Integral $t := iu + ju + 2u$ und $\diff t = (i + j + 2) \diff u$ führt zu
\begin{align*}
    \langle U_i, U_j \rangle_{\varrho_U} &= \frac{1}{2} \int\limits_{\pi (i + j + 2)}^{0} \cos(t) (i + j + 2)^{-1} \diff t - \frac{1}{2} \int\limits_{\pi}^{0} \cos(iu - ju) \diff u \\
    &= -\frac{\sin t}{2(i + j + 2)} \Bigg |_{\pi (i + j + 2)}^{0} - \frac{1}{2} \int\limits_{\pi}^{0} \cos(iu - ju) \diff u \\
    &= -\frac{\sin \pi (i + j + 2)}{2(i + j + 2)} - \frac{1}{2} \int\limits_{\pi}^{0} \cos(iu - ju) \diff u
\end{align*}

Wir substituieren erneut im zweiten Integral $t := iu - ju$ und $\diff t = (i - j) \diff u$:
\begin{align*}
    \langle U_i, U_j \rangle_{\varrho_U} &= -\frac{\sin \pi (i + j + 2)}{2(i + j + 2)} - \frac{1}{2} \int\limits_{\pi (i - j)}^{0} \cos(t) (i-j)^{-1} \diff t \\
    &= -\frac{\sin \pi (i + j + 2)}{2(i + j + 2)} - \frac{1}{2(i-j)} \int\limits_{\pi (i - j)}^{0} \cos(t) \diff t \\
    &= -\frac{\sin \pi (i + j + 2)}{2(i + j + 2)} + \frac{\sin t}{2(i-j)} \Bigg |_{\pi (i - j)}^{0} \\
    &= \frac{1}{2} \left( \frac{\sin \pi (i-j)}{(i-j)} -\frac{\sin \pi (i + j + 2)}{(i + j + 2)} \right) \textrm{, und da } i \neq j \\
    \langle U_i, U_j \rangle_{\varrho_U} &= 0. \, &\Box
\end{align*}

Nach Voraussetzung ist $U_n(x) = U_n(-x)$ und für die Gewichtsfunktion $\varrho_U$ ist $\varrho_U(x) := \sqrt{1-x^2} = \sqrt{1-(-x)^2} = \varrho_U(-x)$. Folglich gilt
\begin{align*}
    \langle \mu_1 U_i, U_i \rangle_{\varrho_U} &= \int\limits_{-1}^1 x U_i(x) U_i(x) \varrho_U(x) \diff x \\
    &= \int\limits_{0}^1 x U_i(x) U_i(x) \varrho_U(x) \diff x + \int\limits_{0}^1 -x \underbrace{U_i(-x) U_i(-x)}_{=U_i(x)U_i(x)} \underbrace{\varrho_U(-x)}_{= \varrho_U(x)} \diff x \\
    &= \int\limits_{0}^1 x U_i(x) U_i(x) \varrho_U(x) \diff x - \int\limits_{0}^1 x U_i(x) U_i(x) \varrho_U(x) \diff x \\
    \langle \mu_1 U_i, U_i \rangle_{\varrho_U} &= 0. \, &\Box
\end{align*}

\end{document}
