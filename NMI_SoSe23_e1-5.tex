\documentclass[a4paper,12pt]{article}
\usepackage{fancyhdr}
\usepackage[ngerman,german]{babel}
\usepackage{german}
\usepackage[utf8]{inputenc}
\usepackage[active]{srcltx}
\usepackage{algorithm}
\usepackage[noend]{algorithmic}
\usepackage{amsmath}
\usepackage{amssymb}
\usepackage{amsthm}
\usepackage{bbm}
\usepackage{enumerate}
\usepackage{graphicx}
\usepackage{ifthen}
\usepackage{listings}
\usepackage{struktex}
\usepackage{hyperref}
\usepackage[onehalfspacing]{setspace}
\usepackage{geometry}
\usepackage{calc}

%%%%%%%%%%%%%%%%%%%%%%%%%%%%%%%%%%%%%%%%%%%%%%%%%%%%%%
%%%%%%%%%%%%%% EDIT THIS PART %%%%%%%%%%%%%%%%%%%%%%%%
%%%%%%%%%%%%%%%%%%%%%%%%%%%%%%%%%%%%%%%%%%%%%%%%%%%%%%

\newcommand{\Fach}{01270 Numerische Mathematik I}
\newcommand{\FachKurz}{NMI}
\newcommand{\Name}{Oliver Schwarz}
\newcommand{\Matrikelnummer}{6883389}
\newcommand{\Semester}{SoSe 23}
\newcommand{\Kurseinheit}{1}
\newcommand{\AufgabeNummer}{5}

%%%%%%%%%%%%%%%%%%%%%%%%%%%%%%%%%%%%%%%%%%%%%%%%%%%%%%
%%%%%%%%%%%%%% DO NOT EDIT THIS PART %%%%%%%%%%%%%%%%%
%%%%%%%%%%%%%%%%%%%%%%%%%%%%%%%%%%%%%%%%%%%%%%%%%%%%%%

\newcommand{\Aufgabe}[1]{
  {
  \vspace*{0.5cm}
  \textbf{Aufgabe #1}
  \vspace*{0.2cm}
  }
}

%%%%%%%%%%%%%%%%%%%%%%%%%%%%%%%%%%%%%%%%%%%%%%%%%%%%%%
%%%%%%%%%%%%%% PAGE SETTINGS %%%%%%%%%%%%%%%%%%%%%%%%%
%%%%%%%%%%%%%%%%%%%%%%%%%%%%%%%%%%%%%%%%%%%%%%%%%%%%%%

\setlength{\parindent}{0em}
\topmargin 0cm
\oddsidemargin 0cm
\evensidemargin 0cm

\geometry{%
  left=45.0mm,
  right=15.0mm,
  top=25mm,
  bottom=25mm,
  bindingoffset=0mm,
  headheight=30pt,
  includehead
}

\fancyheadoffset[L]{20mm}
\renewcommand{\headrulewidth}{1pt}

\pagestyle{fancy}

%%%%%%%%%%%%%%%%%%%%%%%%%%%%%%%%%%%%%%%%%%%%%%%%%%%%%%
%%%%%%%%%%%%%% PDF SETTINGS %%%%%%%%%%%%%%%%%%%%%%%%%%
%%%%%%%%%%%%%%%%%%%%%%%%%%%%%%%%%%%%%%%%%%%%%%%%%%%%%%

\hypersetup{
    %pdftitle={\Fach{}: Übungsblatt \Uebungsblatt{}},
    pdftitle={\Fach{}: Kurseinheit \Kurseinheit{} Aufgabe \AufgabeNummer{}},
    pdfauthor={\Name},
    pdfborder={0 0 0}
}

%%%%%%%%%%%%%%%%%%%%%%%%%%%%%%%%%%%%%%%%%%%%%%%%%%%%%%
%%%%%%%%%%%%%% CODE SETTINGS %%%%%%%%%%%%%%%%%%%%%%%%%
%%%%%%%%%%%%%%%%%%%%%%%%%%%%%%%%%%%%%%%%%%%%%%%%%%%%%%

\lstset{ %
language=java,
basicstyle=\footnotesize\tt,
showtabs=false,
tabsize=2,
captionpos=b,
breaklines=true,
extendedchars=true,
showstringspaces=false,
flexiblecolumns=true,
}

%%%%%%%%%%%%%%%%%%%%%%%%%%%%%%%%%%%%%%%%%%%%%%%%%%%%%%
%%%%%%%%%%%%%% DOCUMENT %%%%%%%%%%%%%%%%%%%%%%%%%%%%%%
%%%%%%%%%%%%%%%%%%%%%%%%%%%%%%%%%%%%%%%%%%%%%%%%%%%%%%

\title{Kurseinheit \Kurseinheit{} Aufgabe \AufgabeNummer{}}
\author{\Name{}}

\begin{document}

%\renewcommand{\theequation}{L\Kurseinheit{}.\AufgabeNummer{}.\arabic{equation}}
\renewcommand{\theequation}{L \AufgabeNummer{}.\arabic{equation}}

%%%%%%%%%%%%%%%%%%%%%%%%%%%%%%%%%%%%%%%%%%%%%%%%%%%%%%
%%%%%%%%%%%%%% HEADER %%%%%%%%%%%%%%%%%%%%%%%%%%%%%%%%
%%%%%%%%%%%%%%%%%%%%%%%%%%%%%%%%%%%%%%%%%%%%%%%%%%%%%%

\lhead{\sf \large \Fach{} \\ \small \Name{} - \Matrikelnummer{}}
%\rhead{\sf \Semester{}}
\rhead{\sf \FachKurz{} \quad E \Kurseinheit{}/\AufgabeNummer{}}

%%%%%%%%%%%%%%%%%%%%%%%%%%%%%%%%%%%%%%%%%%%%%%%%%%%%%%
%%%%%%%%%%%%%% START HERE %%%%%%%%%%%%%%%%%%%%%%%%%%%%
%%%%%%%%%%%%%%%%%%%%%%%%%%%%%%%%%%%%%%%%%%%%%%%%%%%%%%

\Aufgabe{\AufgabeNummer{} \quad (Kondition)} 

Es ist $z := f(x) := 1 - \cos^2(x)$ mit $x \in \mathbb{R}$ und $|x| \ll 1$. Weiterhin ist $f(0) = 0$ und $0 \leq f(x) \leq 1$ für alle $x \in \mathbb{R}$.

\begin{enumerate}[a)]
\item % a)
\begin{align*}
    \kappa_{f(x)} &= \frac{x}{f(x)} f'(x) \\
    &= \frac{x}{1 - \cos^2(x)} \, 2 \sin(x) \cos(x) \\
    &= 2 x \frac{\sin(x) \cos(x)}{\sin^2(x)} \\
    \kappa_{f(x)} &= 2 x \frac{\cos(x)}{\sin(x)}.
\end{align*}

Nach \textsc{de l'Hospital} ist $\lim\limits_{x \rightarrow 0} \frac{x}{\sin{x}} = \lim\limits_{x \rightarrow 0} \frac{1}{\cos{x}} = 1$, also $\lim\limits_{x \rightarrow 0} \kappa_{f(x)} = 2$ und $\kappa_{f(x)} \leq 2$ für $|x| \ll 1$: Die Aufgabe ist gut konditioniert.

\item % b)
Wir betrachten den Algorithmus
$$y_1 := \cos x, \quad y_2 := y_1^2, \quad \textrm{und} \quad z := 1 - y_2.$$

Wir rollen den Algorithmus von hinten auf und ersetzen $y_2 = y_1^2$:
\begin{align}
    z &= 1 - y_2 \label{eq:1}\\
    z &= 1 - y_1^2 \label{eq:2}
\end{align}

Wir bestimmen die Konditionszahlen wie in Teilaufgabe (a) mit \eqref{eq:1} und \eqref{eq:2}:
\begin{align*}
    \kappa_1 &:= \frac{y_1}{z} \frac{\partial z}{\partial y_1} = -2\frac{y_1^2}{z} 
              = -2 \frac{\cos^2(x)}{\sin^2(x)} \longrightarrow - \infty \qquad \text{für} \ x \rightarrow 0,\\
    \kappa_2 &:= \frac{y_2}{z} \frac{\partial z}{\partial y_2} = -\frac{y_2}{z}
              = - \frac{\cos^2(x)}{\sin^2(x)} \longrightarrow - \infty \qquad \text{für} \ x \rightarrow 0.
\end{align*}

Die Beträge der Konditionszahlen werden sehr groß für $x \rightarrow 0$ im Vergleich zur Konditionszahl der Aufgabe. Der Algorithmus ist also numerisch instabil.
\item % c)
Für die Berechnung der Konditionszahl in Teilaufgabe (a) haben wir die Beziehung $1 - \cos^2(x) = \sin^2(x)$ genutzt. Setzen wir $z := 1 - \cos^2(x) = \sin^2(x) =: \tilde{z}$, dann erhalten wir den Algorithmus
$$\tilde{y}_1 := \sin x \quad \textrm{und} \qquad \tilde{z} := \tilde{y}_1^2.$$

Wir bestimmen die Konditionszahl für $\tilde{z} = \tilde{y}_1^2$:
\begin{equation*}
    \tilde{\kappa}_1 := \frac{\tilde{y}_1}{\tilde{z}} \frac{\partial \tilde{z}}{\partial \tilde{y}_1} = 2\frac{\tilde{y}_1^2}{\tilde{z}} = 2 \frac{\sin^2(x)}{\sin^2(x)} = 2.\\    
\end{equation*}

Der Algorithmus ist numerisch stabil, denn die Konditionszahl $\tilde{\kappa}_1 = 2$ des Algorithmus ist klein und gleich der Konditionszahl $\kappa_{f(x)} = 2$ der Aufgabe. 
\end{enumerate}
\end{document}
