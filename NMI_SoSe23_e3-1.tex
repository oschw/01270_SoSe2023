\documentclass[a4paper,12pt]{article}
\usepackage{fancyhdr}
\usepackage[ngerman,german]{babel}
\usepackage{german}
\usepackage[utf8]{inputenc}
\usepackage[active]{srcltx}
\usepackage{algorithm}
\usepackage[noend]{algorithmic}
\usepackage{amsmath}
\usepackage{amssymb}
\usepackage{amsthm}
\usepackage{bbm}
\usepackage{enumerate}
\usepackage{graphicx}
\usepackage{ifthen}
\usepackage{listings}
\usepackage{struktex}
\usepackage{hyperref}
\usepackage[onehalfspacing]{setspace}
\usepackage{geometry}
\usepackage{calc}

%%%%%%%%%%%%%%%%%%%%%%%%%%%%%%%%%%%%%%%%%%%%%%%%%%%%%%
%%%%%%%%%%%%%% EDIT THIS PART %%%%%%%%%%%%%%%%%%%%%%%%
%%%%%%%%%%%%%%%%%%%%%%%%%%%%%%%%%%%%%%%%%%%%%%%%%%%%%%

\newcommand{\Fach}{01270 Numerische Mathematik I}
\newcommand{\FachKurz}{NMI}
\newcommand{\Name}{Oliver Schwarz}
\newcommand{\Matrikelnummer}{6883389}
\newcommand{\Semester}{SoSe 23}
\newcommand{\Kurseinheit}{3}
\newcommand{\AufgabeNummer}{1}

%%%%%%%%%%%%%%%%%%%%%%%%%%%%%%%%%%%%%%%%%%%%%%%%%%%%%%
%%%%%%%%%%%%%% DO NOT EDIT THIS PART %%%%%%%%%%%%%%%%%
%%%%%%%%%%%%%%%%%%%%%%%%%%%%%%%%%%%%%%%%%%%%%%%%%%%%%%

\newcommand{\Aufgabe}[1]{
  {
  \vspace*{0.5cm}
  \textbf{Aufgabe #1}
  \vspace*{0.2cm}
  }
}

\newcommand{\cond}[0]{
  {
  \textrm{cond}
  }
}

% Beschreibung
% Unterscheidet im Mathe-Mode zwischen
% "Dezimal-komma" und Satzzeichen.
% Muss bei Eingabe durch kein/ein Leerzeichen
% dahinter angegeben werden.
% Literatur Richard Hirsch in: DTK, 1/1994, S. 42 ff.
\mathchardef\CommaOrdinary="013B \mathchardef\CommaPunct="613B
\mathcode`,="8000 % , im Math-Mode aktiv ("8000) machen
{\catcode`\,=\active \gdef
,{\obeyspaces\futurelet\next\CommaCheck}}
\def\CommaCheck{\if\space\next\CommaPunct\else\CommaOrdinary\fi}

%%%%%%%%%%%%%%%%%%%%%%%%%%%%%%%%%%%%%%%%%%%%%%%%%%%%%%
%%%%%%%%%%%%%% PAGE SETTINGS %%%%%%%%%%%%%%%%%%%%%%%%%
%%%%%%%%%%%%%%%%%%%%%%%%%%%%%%%%%%%%%%%%%%%%%%%%%%%%%%

\setlength{\parindent}{0em}
\topmargin 0cm
\oddsidemargin 0cm
\evensidemargin 0cm

\geometry{%
  left=45.0mm,
  right=15.0mm,
  top=25mm,
  bottom=25mm,
  bindingoffset=0mm,
  headheight=30pt,
  includehead
}

\fancyheadoffset[L]{20mm}
\renewcommand{\headrulewidth}{1pt}

\pagestyle{fancy}

%%%%%%%%%%%%%%%%%%%%%%%%%%%%%%%%%%%%%%%%%%%%%%%%%%%%%%
%%%%%%%%%%%%%% PDF SETTINGS %%%%%%%%%%%%%%%%%%%%%%%%%%
%%%%%%%%%%%%%%%%%%%%%%%%%%%%%%%%%%%%%%%%%%%%%%%%%%%%%%

\hypersetup{
    %pdftitle={\Fach{}: Übungsblatt \Uebungsblatt{}},
    pdftitle={\Fach{}: Kurseinheit \Kurseinheit{} Aufgabe \AufgabeNummer{}},
    pdfauthor={\Name},
    pdfborder={0 0 0}
}

%%%%%%%%%%%%%%%%%%%%%%%%%%%%%%%%%%%%%%%%%%%%%%%%%%%%%%
%%%%%%%%%%%%%% CODE SETTINGS %%%%%%%%%%%%%%%%%%%%%%%%%
%%%%%%%%%%%%%%%%%%%%%%%%%%%%%%%%%%%%%%%%%%%%%%%%%%%%%%

\lstset{ %
language=java,
basicstyle=\footnotesize\tt,
showtabs=false,
tabsize=2,
captionpos=b,
breaklines=true,
extendedchars=true,
showstringspaces=false,
flexiblecolumns=true,
}

%%%%%%%%%%%%%%%%%%%%%%%%%%%%%%%%%%%%%%%%%%%%%%%%%%%%%%
%%%%%%%%%%%%%% DOCUMENT %%%%%%%%%%%%%%%%%%%%%%%%%%%%%%
%%%%%%%%%%%%%%%%%%%%%%%%%%%%%%%%%%%%%%%%%%%%%%%%%%%%%%

\title{Kurseinheit \Kurseinheit{} Aufgabe \AufgabeNummer{}}
\author{\Name{}}

\begin{document}

%\renewcommand{\theequation}{L\Kurseinheit{}.\AufgabeNummer{}.\arabic{equation}}
\renewcommand{\theequation}{L \AufgabeNummer{}.\arabic{equation}}

%%%%%%%%%%%%%%%%%%%%%%%%%%%%%%%%%%%%%%%%%%%%%%%%%%%%%%
%%%%%%%%%%%%%% HEADER %%%%%%%%%%%%%%%%%%%%%%%%%%%%%%%%
%%%%%%%%%%%%%%%%%%%%%%%%%%%%%%%%%%%%%%%%%%%%%%%%%%%%%%

\lhead{\sf \large \Fach{} \\ \small \Name{} - \Matrikelnummer{}}
%\rhead{\sf \Semester{}}
\rhead{\sf \FachKurz{} \quad E \Kurseinheit{}/\AufgabeNummer{}}

%%%%%%%%%%%%%%%%%%%%%%%%%%%%%%%%%%%%%%%%%%%%%%%%%%%%%%
%%%%%%%%%%%%%% START HERE %%%%%%%%%%%%%%%%%%%%%%%%%%%%
%%%%%%%%%%%%%%%%%%%%%%%%%%%%%%%%%%%%%%%%%%%%%%%%%%%%%%

\Aufgabe{\AufgabeNummer{}}
Es seien gegeben 
\begin{equation*}
\pmb{A} := 
\begin{pmatrix}
10 & 7 & 8 & 7\\
7 & 5 & 6 & 5\\
8 & 6 & 10 & 9\\
7 & 5 & 9 & 10
\end{pmatrix} \textrm{, }
\pmb{b} :=
\begin{pmatrix}
    32 \\ 23 \\ 33 \\ 31    
\end{pmatrix} \textrm{ und }
\Delta_b := \frac{1}{10}
\begin{pmatrix}
    1 \\ -1 \\ 1 \\ -1    
\end{pmatrix}
\end{equation*}

\begin{enumerate}[a)]
    \item %a)
Es ist $\cond_{\infty}\, \pmb{A} := ||\pmb{A}||_{\infty} ||\pmb{A}^{-1}||_{\infty}$.

Wir berechnen zunächst $\pmb{A}^{-1}$
%https://www.wolframalpha.com/input?i=%7B%7B10%2C7%2C8%2C7%7D%2C%7B7%2C5%2C6%2C5%7D%2C%7B8%2C6%2C10%2C9%7D%2C%7B7%2C5%2C9%2C10%7D%7D
%{{10,7,8,7},{7,5,6,5},{8,6,10,9},{7,5,9,10}}
\begin{equation*}
\pmb{A}^{-1} = 
\begin{pmatrix}
25 & -41 & 10 & -6\\
-41 & 68 & -17 & 10\\
10 & -17 & 5 & -3\\
-6 & 10 & -3 & 2
\end{pmatrix}.
\end{equation*}
Somit ist $\cond_{\infty}\, \pmb{A} = ||\pmb{A}||_{\infty} ||\pmb{A}^{-1}||_{\infty} = 33 \cdot 136 = 4488$.

    \item \label{b:a1} %b)
Gesucht sind $\pmb{x}$ und ${\pmb{\tilde{x}}}$ als Lösungen der Gleichungssysteme $$\pmb{Ax} = \pmb{b} \textrm{ und } \pmb{A\tilde{x}} = \pmb{b} + \pmb{\Delta_{b}}.$$ Da wir bereits $\pmb{A}^{-1}$ kennen, multiplizieren wir die Gleichungssysteme auf beiden Seiten mit $\pmb{A}^{-1}$ und erhalten
\begin{equation*}
\pmb{x} = \pmb{A}^{-1}\pmb{b} \textrm{ und } \pmb{\tilde{x}} = \pmb{A}^{-1} (\pmb{b} + \pmb{\Delta_{b}}).
\end{equation*}
Hieraus ergibt sich
\begin{equation*}
    \pmb{x} =
\begin{pmatrix}
    1 \\ 1 \\ 1 \\ 1    
\end{pmatrix} \textrm{ und }
    \pmb{\tilde{x}} =
\begin{pmatrix}
    9,2 \\ -12,6 \\ 4,5 \\ -1,1    
\end{pmatrix}.
\end{equation*}

    \item %c)
Nach Kapitel 3.1 gilt (mit der Definition 3.1.3) die Abschätzung
\begin{equation}
    \frac{||\pmb{\Delta_x}||_{\infty}}{||\pmb{x}||_{\infty}} \leq \cond_{\infty}\, \pmb{A} \frac{||\pmb{\Delta_b}||_{\infty}}{||\pmb{b}||_{\infty}}. \label{formel1:a1}
\end{equation}
Es sind $||\pmb{\Delta_b}||_{\infty} = \frac{1}{10}$ und $||\pmb{b}||_{\infty} = 33$.
Wir erhalten also aus \ref{formel1:a1} für die Abschätzung des relativen Fehlers
\begin{equation*}
    \frac{||\pmb{\Delta_x}||_{\infty}}{||\pmb{x}||_{\infty}} \leq 13,6.
\end{equation*}
Aus Teilaufgabe \ref{b:a1} kennen wir $\pmb{x}$ und $\pmb{\tilde{x}}$. Es ist $||\pmb{x}||_{\infty} = 1$ und
\begin{equation*}
    \pmb{\Delta_x} = \pmb{x} - \pmb{\tilde{x}} = \begin{pmatrix}
    -8,2 \\ 13,6 \\ -3,5 \\ 2,1    
\end{pmatrix} \textrm{ und somit }
||\pmb{\Delta_x}||_{\infty} = 13,6.
\end{equation*}
Für den relativen Fehler erhalten wir
\begin{equation*}
    \frac{||\pmb{\Delta_x}||_{\infty}}{||\pmb{x}||_{\infty}} = 13,6.
\end{equation*}
Unsere Abschätzung, die obere Schranke, und der tatsächliche relative Fehler stimmen überein.
\end{enumerate}

\end{document}
