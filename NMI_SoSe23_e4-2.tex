\documentclass[a4paper,12pt]{article}
\usepackage{fancyhdr}
\usepackage[ngerman,german]{babel}
\usepackage{german}
\usepackage[utf8]{inputenc}
\usepackage[active]{srcltx}
\usepackage{algorithm}
\usepackage[noend]{algorithmic}
\usepackage{amsmath}
\usepackage{amssymb}
\usepackage{amsthm}
\usepackage{bbm}
\usepackage{enumerate}
\usepackage{graphicx}
\usepackage{ifthen}
\usepackage{listings}
\usepackage{struktex}
\usepackage{hyperref}
\usepackage[onehalfspacing]{setspace}
\usepackage{geometry}
\usepackage{calc}

%%%%%%%%%%%%%%%%%%%%%%%%%%%%%%%%%%%%%%%%%%%%%%%%%%%%%%
%%%%%%%%%%%%%% EDIT THIS PART %%%%%%%%%%%%%%%%%%%%%%%%
%%%%%%%%%%%%%%%%%%%%%%%%%%%%%%%%%%%%%%%%%%%%%%%%%%%%%%

\newcommand{\Fach}{01270 Numerische Mathematik I}
\newcommand{\FachKurz}{NMI}
\newcommand{\Name}{Oliver Schwarz}
\newcommand{\Matrikelnummer}{6883389}
\newcommand{\Semester}{SoSe 23}
\newcommand{\Kurseinheit}{4}
\newcommand{\AufgabeNummer}{2}

%%%%%%%%%%%%%%%%%%%%%%%%%%%%%%%%%%%%%%%%%%%%%%%%%%%%%%
%%%%%%%%%%%%%% DO NOT EDIT THIS PART %%%%%%%%%%%%%%%%%
%%%%%%%%%%%%%%%%%%%%%%%%%%%%%%%%%%%%%%%%%%%%%%%%%%%%%%

\newcommand{\Aufgabe}[1]{
  {
  \vspace*{0.5cm}
  \textbf{Aufgabe #1}
  \vspace*{0.2cm}
  }
}

\newcommand{\cond}[0]{
  {
  \textrm{cond}
  }
}

% Beschreibung
% Unterscheidet im Mathe-Mode zwischen
% "Dezimal-komma" und Satzzeichen.
% Muss bei Eingabe durch kein/ein Leerzeichen
% dahinter angegeben werden.
% Literatur Richard Hirsch in: DTK, 1/1994, S. 42 ff.
\mathchardef\CommaOrdinary="013B \mathchardef\CommaPunct="613B
\mathcode`,="8000 % , im Math-Mode aktiv ("8000) machen
{\catcode`\,=\active \gdef
,{\obeyspaces\futurelet\next\CommaCheck}}
\def\CommaCheck{\if\space\next\CommaPunct\else\CommaOrdinary\fi}

% For comparison, the existing overlap macros:
% \def\llap#1{\hbox to 0pt{\hss#1}}
% \def\rlap#1{\hbox to 0pt{#1\hss}}
\def\clap#1{\hbox to 0pt{\hss#1\hss}}
\def\mathllap{\mathpalette\mathllapinternal}
\def\mathrlap{\mathpalette\mathrlapinternal}
\def\mathclap{\mathpalette\mathclapinternal}
\def\mathllapinternal#1#2{%
\llap{$\mathsurround=0pt#1{#2}$}}
\def\mathrlapinternal#1#2{%
\rlap{$\mathsurround=0pt#1{#2}$}}
\def\mathclapinternal#1#2{%
\clap{$\mathsurround=0pt#1{#2}$}}

\DeclareSymbolFont{matha}{OML}{txmi}{m}{it}% txfonts
\DeclareMathSymbol{\varv}{\mathord}{matha}{118}
\DeclareMathSymbol{\varw}{\mathord}{matha}{119}
%%%%%%%%%%%%%%%%%%%%%%%%%%%%%%%%%%%%%%%%%%%%%%%%%%%%%%
%%%%%%%%%%%%%% PAGE SETTINGS %%%%%%%%%%%%%%%%%%%%%%%%%
%%%%%%%%%%%%%%%%%%%%%%%%%%%%%%%%%%%%%%%%%%%%%%%%%%%%%%

\setlength{\parindent}{0em}
\topmargin 0cm
\oddsidemargin 0cm
\evensidemargin 0cm

\geometry{%
  left=45.0mm,
  right=15.0mm,
  top=25mm,
  bottom=25mm,
  bindingoffset=0mm,
  headheight=30pt,
  includehead
}

\fancyheadoffset[L]{20mm}
\renewcommand{\headrulewidth}{1pt}

\pagestyle{fancy}

%%%%%%%%%%%%%%%%%%%%%%%%%%%%%%%%%%%%%%%%%%%%%%%%%%%%%%
%%%%%%%%%%%%%% PDF SETTINGS %%%%%%%%%%%%%%%%%%%%%%%%%%
%%%%%%%%%%%%%%%%%%%%%%%%%%%%%%%%%%%%%%%%%%%%%%%%%%%%%%

\hypersetup{
    %pdftitle={\Fach{}: Übungsblatt \Uebungsblatt{}},
    pdftitle={\Fach{}: Kurseinheit \Kurseinheit{} Aufgabe \AufgabeNummer{}},
    pdfauthor={\Name},
    pdfborder={0 0 0}
}

%%%%%%%%%%%%%%%%%%%%%%%%%%%%%%%%%%%%%%%%%%%%%%%%%%%%%%
%%%%%%%%%%%%%% CODE SETTINGS %%%%%%%%%%%%%%%%%%%%%%%%%
%%%%%%%%%%%%%%%%%%%%%%%%%%%%%%%%%%%%%%%%%%%%%%%%%%%%%%

\lstset{ %
language=java,
basicstyle=\footnotesize\tt,
showtabs=false,
tabsize=2,
captionpos=b,
breaklines=true,
extendedchars=true,
showstringspaces=false,
flexiblecolumns=true,
}

%%%%%%%%%%%%%%%%%%%%%%%%%%%%%%%%%%%%%%%%%%%%%%%%%%%%%%
%%%%%%%%%%%%%% DOCUMENT %%%%%%%%%%%%%%%%%%%%%%%%%%%%%%
%%%%%%%%%%%%%%%%%%%%%%%%%%%%%%%%%%%%%%%%%%%%%%%%%%%%%%

\title{Kurseinheit \Kurseinheit{} Aufgabe \AufgabeNummer{}}
\author{\Name{}}

\begin{document}

%\renewcommand{\theequation}{L\Kurseinheit{}.\AufgabeNummer{}.\arabic{equation}}
\renewcommand{\theequation}{L \AufgabeNummer{}.\arabic{equation}}

%%%%%%%%%%%%%%%%%%%%%%%%%%%%%%%%%%%%%%%%%%%%%%%%%%%%%%
%%%%%%%%%%%%%% HEADER %%%%%%%%%%%%%%%%%%%%%%%%%%%%%%%%
%%%%%%%%%%%%%%%%%%%%%%%%%%%%%%%%%%%%%%%%%%%%%%%%%%%%%%

\lhead{\sf \large \Fach{} \\ \small \Name{} - \Matrikelnummer{}}
%\rhead{\sf \Semester{}}
\rhead{\sf \FachKurz{} \quad E \Kurseinheit{}/\AufgabeNummer{}}

%%%%%%%%%%%%%%%%%%%%%%%%%%%%%%%%%%%%%%%%%%%%%%%%%%%%%%
%%%%%%%%%%%%%% START HERE %%%%%%%%%%%%%%%%%%%%%%%%%%%%
%%%%%%%%%%%%%%%%%%%%%%%%%%%%%%%%%%%%%%%%%%%%%%%%%%%%%%

\Aufgabe{\AufgabeNummer{}}

\begin{align*}
\text{S1.} \quad &\\
\mathrlap{k := 2} \\
\mathrlap{\pmb{A}^{(0)} :=
\begin{pmatrix}
    0 & -2 & -1 \\
    0 &  0 & -1 \\
    4 &  5 &  2
\end{pmatrix}}\\
\mathrlap{\pmb{b}^{(0)} :=
\begin{pmatrix}
    1 \\
    1 \\
    1
\end{pmatrix}}\\
&i = 1: &&\\
&&&\pmb{a} :=
\begin{pmatrix}
    0 \\
    0 \\
    4
\end{pmatrix}, \quad
\xi := ||\pmb{a}||_2 = 4, \quad
\alpha := -\xi = -4,\\
&&&\pmb{\varv} := \pmb{a} - \alpha \pmb{e}^1 =
\begin{pmatrix}
    4 \\
    0 \\
    4
\end{pmatrix}, \quad
\gamma := \frac{2}{\pmb{\varv}\pmb{\varv}^H} = \frac{1}{16}, \quad
\pmb{\varw} := \pmb{\varv}^H \pmb{A}^{(0)} = (16, 12, 4),\\
&&&\pmb{H}_{\pmb{\varv}}^{(1)} := \pmb{I} - \gamma \pmb{\varv \varv}^H =
\begin{pmatrix}
    0 &  0 & -1 \\
    0 &  1 &  0 \\
   -1 &  0 &  1
\end{pmatrix}, \\
&&&\pmb{A}^{(1)} := \pmb{A}^{(0)} - \gamma \pmb{\varv \varw} =
\begin{pmatrix}
   -4 & -5 & -2 \\
    0 &  0 & -1 \\
    0 &  2 &  1
\end{pmatrix}, \\
&&&\pmb{b}^{(1)} := \pmb{b}^{(0)} - \gamma(\pmb{\varv}^H\pmb{b}^{(0)})\pmb{\varv} =
\begin{pmatrix}
   -1 \\
    1 \\
   -1
\end{pmatrix}.
\end{align*}
\begin{align*}
&i = 2: &&\\
&&&\pmb{a} :=
\begin{pmatrix}
    0 \\
    0 \\
    2
\end{pmatrix}, \quad
\xi := ||\pmb{a}||_2 = 2, \quad
\alpha := -\xi = -2,\\
&&&\pmb{\varv} := \pmb{a} - \alpha \pmb{e}^2 =
\begin{pmatrix}
    0 \\
    2 \\
    2
\end{pmatrix}, \quad
\gamma := \frac{2}{\pmb{\varv}\pmb{\varv}^H} = \frac{1}{4}, \quad
\pmb{\varw} := \pmb{\varv}^H \pmb{A}^{(0)} = (0, 4, 0),\\
&&&\pmb{H}_{\pmb{\varv}}^{(2)} := \pmb{I} - \gamma \pmb{\varv \varv}^H =
\begin{pmatrix}
    0 &  0 & -1 \\
    0 &  1 &  0 \\
   -1 &  0 &  1
\end{pmatrix}, \\
&&&\pmb{A}^{(2)} := \pmb{A}^{(1)} - \gamma \pmb{\varv \varw} =
\begin{pmatrix}
   -4 & -5 & -2 \\
    0 & -2 & -1 \\
    0 &  0 &  1
\end{pmatrix}, \\
&&&\pmb{b}^{(2)} := \pmb{b}^{(1)} - \gamma(\pmb{\varv}^H\pmb{b}^{(1)})\pmb{\varv} =
\begin{pmatrix}
   -1 \\
    1 \\
   -1
\end{pmatrix}.\\
&\mathrlap{\textit{Zwischenergebnis:}} &&\\
&&&\pmb{R} =
\begin{pmatrix}
   -4 & -5 & -2 \\
    0 & -2 & -1 \\
    0 &  0 &  1
\end{pmatrix}, \quad
\pmb{Q} = \pmb{H}_{\pmb{\varv}}^{(1)} \pmb{H}_{\pmb{\varv}}^{(2)} = 
\begin{pmatrix}
    0 &  1 &  0 \\
    0 &  0 & -1 \\
   -1 &  0 &  0
\end{pmatrix}\\
\text{S2.} \quad & \\
&&&\pmb{Rx} = \pmb{Q}^H \pmb{b},\\
&&&\begin{pmatrix}
   -4 & -5 & -2 \\
    0 & -2 & -1 \\
    0 &  0 &  1
\end{pmatrix}
\begin{pmatrix}
   x_1 \\
   x_2 \\
   x_3
\end{pmatrix} = 
\begin{pmatrix}
   -1 \\
    1 \\
   -1
\end{pmatrix}, \quad
\pmb{x} =
\begin{pmatrix}
   \frac{3}{4} \\
    0 \\
   -1
\end{pmatrix}.
\end{align*}


\end{document}
