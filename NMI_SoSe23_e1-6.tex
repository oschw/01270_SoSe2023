\documentclass[a4paper,12pt]{article}
\usepackage{fancyhdr}
\usepackage[ngerman,german]{babel}
\usepackage{german}
\usepackage[utf8]{inputenc}
\usepackage[active]{srcltx}
\usepackage{algorithm}
\usepackage[noend]{algorithmic}
\usepackage{amsmath}
\usepackage{amssymb}
\usepackage{amsthm}
\usepackage{bbm}
\usepackage{enumerate}
\usepackage{graphicx}
\usepackage{ifthen}
\usepackage{listings}
\usepackage{struktex}
\usepackage{hyperref}
\usepackage[onehalfspacing]{setspace}
\usepackage{geometry}
\usepackage{calc}

%%%%%%%%%%%%%%%%%%%%%%%%%%%%%%%%%%%%%%%%%%%%%%%%%%%%%%
%%%%%%%%%%%%%% EDIT THIS PART %%%%%%%%%%%%%%%%%%%%%%%%
%%%%%%%%%%%%%%%%%%%%%%%%%%%%%%%%%%%%%%%%%%%%%%%%%%%%%%

\newcommand{\Fach}{01270 Numerische Mathematik I}
\newcommand{\FachKurz}{NMI}
\newcommand{\Name}{Oliver Schwarz}
\newcommand{\Matrikelnummer}{6883389}
\newcommand{\Semester}{SoSe 23}
\newcommand{\Kurseinheit}{1}
\newcommand{\AufgabeNummer}{6}

%%%%%%%%%%%%%%%%%%%%%%%%%%%%%%%%%%%%%%%%%%%%%%%%%%%%%%
%%%%%%%%%%%%%% DO NOT EDIT THIS PART %%%%%%%%%%%%%%%%%
%%%%%%%%%%%%%%%%%%%%%%%%%%%%%%%%%%%%%%%%%%%%%%%%%%%%%%

\newcommand{\Aufgabe}[1]{
  {
  \vspace*{0.5cm}
  \textbf{Aufgabe #1}
  \vspace*{0.2cm}
  }
}

%%%%%%%%%%%%%%%%%%%%%%%%%%%%%%%%%%%%%%%%%%%%%%%%%%%%%%
%%%%%%%%%%%%%% PAGE SETTINGS %%%%%%%%%%%%%%%%%%%%%%%%%
%%%%%%%%%%%%%%%%%%%%%%%%%%%%%%%%%%%%%%%%%%%%%%%%%%%%%%

\setlength{\parindent}{0em}
\topmargin 0cm
\oddsidemargin 0cm
\evensidemargin 0cm

\geometry{%
  left=45.0mm,
  right=15.0mm,
  top=25mm,
  bottom=25mm,
  bindingoffset=0mm,
  headheight=30pt,
  includehead
}

\fancyheadoffset[L]{20mm}
\renewcommand{\headrulewidth}{1pt}

\pagestyle{fancy}

%%%%%%%%%%%%%%%%%%%%%%%%%%%%%%%%%%%%%%%%%%%%%%%%%%%%%%
%%%%%%%%%%%%%% PDF SETTINGS %%%%%%%%%%%%%%%%%%%%%%%%%%
%%%%%%%%%%%%%%%%%%%%%%%%%%%%%%%%%%%%%%%%%%%%%%%%%%%%%%

\hypersetup{
    %pdftitle={\Fach{}: Übungsblatt \Uebungsblatt{}},
    pdftitle={\Fach{}: Kurseinheit \Kurseinheit{} Aufgabe \AufgabeNummer{}},
    pdfauthor={\Name},
    pdfborder={0 0 0}
}

%%%%%%%%%%%%%%%%%%%%%%%%%%%%%%%%%%%%%%%%%%%%%%%%%%%%%%
%%%%%%%%%%%%%% CODE SETTINGS %%%%%%%%%%%%%%%%%%%%%%%%%
%%%%%%%%%%%%%%%%%%%%%%%%%%%%%%%%%%%%%%%%%%%%%%%%%%%%%%

\lstset{ %
language=java,
basicstyle=\footnotesize\tt,
showtabs=false,
tabsize=2,
captionpos=b,
breaklines=true,
extendedchars=true,
showstringspaces=false,
flexiblecolumns=true,
}

%%%%%%%%%%%%%%%%%%%%%%%%%%%%%%%%%%%%%%%%%%%%%%%%%%%%%%
%%%%%%%%%%%%%% DOCUMENT %%%%%%%%%%%%%%%%%%%%%%%%%%%%%%
%%%%%%%%%%%%%%%%%%%%%%%%%%%%%%%%%%%%%%%%%%%%%%%%%%%%%%

\title{Kurseinheit \Kurseinheit{} Aufgabe \AufgabeNummer{}}
\author{\Name{}}

\begin{document}

%\renewcommand{\theequation}{L\Kurseinheit{}.\AufgabeNummer{}.\arabic{equation}}
\renewcommand{\theequation}{L \AufgabeNummer{}.\arabic{equation}}

%%%%%%%%%%%%%%%%%%%%%%%%%%%%%%%%%%%%%%%%%%%%%%%%%%%%%%
%%%%%%%%%%%%%% HEADER %%%%%%%%%%%%%%%%%%%%%%%%%%%%%%%%
%%%%%%%%%%%%%%%%%%%%%%%%%%%%%%%%%%%%%%%%%%%%%%%%%%%%%%

\lhead{\sf \large \Fach{} \\ \small \Name{} - \Matrikelnummer{}}
%\rhead{\sf \Semester{}}
\rhead{\sf \FachKurz{} \quad E \Kurseinheit{}/\AufgabeNummer{}}

%%%%%%%%%%%%%%%%%%%%%%%%%%%%%%%%%%%%%%%%%%%%%%%%%%%%%%
%%%%%%%%%%%%%% START HERE %%%%%%%%%%%%%%%%%%%%%%%%%%%%
%%%%%%%%%%%%%%%%%%%%%%%%%%%%%%%%%%%%%%%%%%%%%%%%%%%%%%

\Aufgabe{\AufgabeNummer{} \quad (Kondition von Verkettungen)} 

Ist

$$f := f_1 \circ f_2 \circ \cdots \circ f_k = f_1( f_2( \cdots f_k(\cdot) \cdots)), \, k \in \mathbb{N},$$

die Verkettung von $k$ differenzierbaren Funktionen $f_{i}: \mathbb{R} \rightarrow \mathbb{R}, i = 1, \ldots, k$, und ist $f_{i}( f_{i+1} \circ \cdots \circ f_{k}(x)) \neq 0$ für $i = 1, \ldots, k-1$ sowie $f_{k}(x) \neq 0$, dann sind

\begin{align}
    \kappa_{f(x)} &= \frac{x}{f(x)} f'(x) \notag \\
    &= \frac{x}{f_1 \circ f_2 \circ \cdots \circ f_k(x)} \, f_1'(f_2 \circ f_3 \circ \cdots \circ f_k(x)) \, (f_2 \circ f_3 \circ \cdots \circ f_k)'(x) \label{eq:1}
\end{align}

und

\begin{equation}
    \kappa_{f_1}(f_2 \circ f_3 \circ \cdots \circ f_k(x)) = \frac{f_2 \circ f_3 \circ \cdots \circ f_k(x)}{f_1 \circ f_2 \circ \cdots \circ f_k(x)} \, f_1'(f_2 \circ f_3 \circ \cdots \circ f_k(x)). \label{eq:2}
\end{equation}

Erweitern wir \eqref{eq:1} mit $\displaystyle\frac{f_2 \circ f_3 \circ \cdots \circ f_k(x)}{f_2 \circ f_3 \circ \cdots \circ f_k(x)}$, können wir (einen Teil von) \eqref{eq:1} in \eqref{eq:2} überführen:

\begin{align*}
    \kappa_{f(x)} &= \frac{f_2 \circ f_3 \circ  \cdots \circ f_k(x)}{f_1 \circ f_2 \circ \cdots \circ f_k(x)} \, f_1'(f_2 \circ \cdots \circ f_k(x)) \quad \frac{x}{f_2 \circ f_3  \circ \cdots \circ f_k(x)} \, (f_2\circ \cdots \circ f_k)'(x) \\
    &= \kappa_{f_1}(f_2 \circ f_3 \circ \cdots \circ f_k(x)) \, \frac{x}{f_2 \circ f_3 \circ \cdots \circ f_k} \, (f_2 \circ \cdots \circ f_k)'(x).
\end{align*}

Dies wiederholen wir für alle $i = 2, \ldots, k-1$ und erhalten (für die Übersichtlichkeit etwas ungenauer notiert):

\begin{align*}
    \kappa_{f(x)} &=  \kappa_{f_1} \kappa_{f_2} \cdots \kappa_{f_{k-1}}\, \frac{x}{f_k} \, f_k' \\
    &= \frac{x}{f_k} \, f_k' \, \prod_{i=1}^{k-1}\kappa_{f_i} \\
    &= \kappa_{f_k} \, \prod_{i=1}^{k-1}\kappa_{f_i}.
\end{align*}

\end{document}
