\documentclass[a4paper,12pt]{article}
\usepackage{fancyhdr}
\usepackage[ngerman,german]{babel}
\usepackage{german}
\usepackage[utf8]{inputenc}
\usepackage[active]{srcltx}
\usepackage{algorithm}
\usepackage[noend]{algorithmic}
\usepackage{amsmath}
\usepackage{amssymb}
\usepackage{amsthm}
\usepackage{bbm}
\usepackage{enumerate}
\usepackage{graphicx}
\usepackage{ifthen}
\usepackage{listings}
\usepackage{struktex}
\usepackage{hyperref}
\usepackage[onehalfspacing]{setspace}
\usepackage{geometry}
\usepackage{calc}

%%%%%%%%%%%%%%%%%%%%%%%%%%%%%%%%%%%%%%%%%%%%%%%%%%%%%%
%%%%%%%%%%%%%% EDIT THIS PART %%%%%%%%%%%%%%%%%%%%%%%%
%%%%%%%%%%%%%%%%%%%%%%%%%%%%%%%%%%%%%%%%%%%%%%%%%%%%%%

\newcommand{\Fach}{01270 Numerische Mathematik I}
\newcommand{\FachKurz}{NMI}
\newcommand{\Name}{Oliver Schwarz}
\newcommand{\Matrikelnummer}{6883389}
\newcommand{\Semester}{SoSe 23}
\newcommand{\Kurseinheit}{4}
\newcommand{\AufgabeNummer}{3}

%%%%%%%%%%%%%%%%%%%%%%%%%%%%%%%%%%%%%%%%%%%%%%%%%%%%%%
%%%%%%%%%%%%%% DO NOT EDIT THIS PART %%%%%%%%%%%%%%%%%
%%%%%%%%%%%%%%%%%%%%%%%%%%%%%%%%%%%%%%%%%%%%%%%%%%%%%%

\newcommand{\Aufgabe}[1]{
  {
  \vspace*{0.5cm}
  \textbf{Aufgabe #1}
  \vspace*{0.2cm}
  }
}

\newcommand{\cond}[0]{
  {
  \textrm{cond}
  }
}

% Beschreibung
% Unterscheidet im Mathe-Mode zwischen
% "Dezimal-komma" und Satzzeichen.
% Muss bei Eingabe durch kein/ein Leerzeichen
% dahinter angegeben werden.
% Literatur Richard Hirsch in: DTK, 1/1994, S. 42 ff.
\mathchardef\CommaOrdinary="013B \mathchardef\CommaPunct="613B
\mathcode`,="8000 % , im Math-Mode aktiv ("8000) machen
{\catcode`\,=\active \gdef
,{\obeyspaces\futurelet\next\CommaCheck}}
\def\CommaCheck{\if\space\next\CommaPunct\else\CommaOrdinary\fi}

% For comparison, the existing overlap macros:
% \def\llap#1{\hbox to 0pt{\hss#1}}
% \def\rlap#1{\hbox to 0pt{#1\hss}}
\def\clap#1{\hbox to 0pt{\hss#1\hss}}
\def\mathllap{\mathpalette\mathllapinternal}
\def\mathrlap{\mathpalette\mathrlapinternal}
\def\mathclap{\mathpalette\mathclapinternal}
\def\mathllapinternal#1#2{%
\llap{$\mathsurround=0pt#1{#2}$}}
\def\mathrlapinternal#1#2{%
\rlap{$\mathsurround=0pt#1{#2}$}}
\def\mathclapinternal#1#2{%
\clap{$\mathsurround=0pt#1{#2}$}}

\DeclareSymbolFont{matha}{OML}{txmi}{m}{it}% txfonts
\DeclareMathSymbol{\varv}{\mathord}{matha}{118}
\DeclareMathSymbol{\varw}{\mathord}{matha}{119}

\newcommand*\wc{{}\cdot{}}
%%%%%%%%%%%%%%%%%%%%%%%%%%%%%%%%%%%%%%%%%%%%%%%%%%%%%%
%%%%%%%%%%%%%% PAGE SETTINGS %%%%%%%%%%%%%%%%%%%%%%%%%
%%%%%%%%%%%%%%%%%%%%%%%%%%%%%%%%%%%%%%%%%%%%%%%%%%%%%%

\setlength{\parindent}{0em}
\topmargin 0cm
\oddsidemargin 0cm
\evensidemargin 0cm

\geometry{%
  left=45.0mm,
  right=15.0mm,
  top=25mm,
  bottom=25mm,
  bindingoffset=0mm,
  headheight=30pt,
  includehead
}

\fancyheadoffset[L]{20mm}
\renewcommand{\headrulewidth}{1pt}

\pagestyle{fancy}

%%%%%%%%%%%%%%%%%%%%%%%%%%%%%%%%%%%%%%%%%%%%%%%%%%%%%%
%%%%%%%%%%%%%% PDF SETTINGS %%%%%%%%%%%%%%%%%%%%%%%%%%
%%%%%%%%%%%%%%%%%%%%%%%%%%%%%%%%%%%%%%%%%%%%%%%%%%%%%%

\hypersetup{
    %pdftitle={\Fach{}: Übungsblatt \Uebungsblatt{}},
    pdftitle={\Fach{}: Kurseinheit \Kurseinheit{} Aufgabe \AufgabeNummer{}},
    pdfauthor={\Name},
    pdfborder={0 0 0}
}

%%%%%%%%%%%%%%%%%%%%%%%%%%%%%%%%%%%%%%%%%%%%%%%%%%%%%%
%%%%%%%%%%%%%% CODE SETTINGS %%%%%%%%%%%%%%%%%%%%%%%%%
%%%%%%%%%%%%%%%%%%%%%%%%%%%%%%%%%%%%%%%%%%%%%%%%%%%%%%

\lstset{ %
language=java,
basicstyle=\footnotesize\tt,
showtabs=false,
tabsize=2,
captionpos=b,
breaklines=true,
extendedchars=true,
showstringspaces=false,
flexiblecolumns=true,
}

%%%%%%%%%%%%%%%%%%%%%%%%%%%%%%%%%%%%%%%%%%%%%%%%%%%%%%
%%%%%%%%%%%%%% DOCUMENT %%%%%%%%%%%%%%%%%%%%%%%%%%%%%%
%%%%%%%%%%%%%%%%%%%%%%%%%%%%%%%%%%%%%%%%%%%%%%%%%%%%%%

\title{Kurseinheit \Kurseinheit{} Aufgabe \AufgabeNummer{}}
\author{\Name{}}

\begin{document}

%\renewcommand{\theequation}{L\Kurseinheit{}.\AufgabeNummer{}.\arabic{equation}}
\renewcommand{\theequation}{L \AufgabeNummer{}.\arabic{equation}}

%%%%%%%%%%%%%%%%%%%%%%%%%%%%%%%%%%%%%%%%%%%%%%%%%%%%%%
%%%%%%%%%%%%%% HEADER %%%%%%%%%%%%%%%%%%%%%%%%%%%%%%%%
%%%%%%%%%%%%%%%%%%%%%%%%%%%%%%%%%%%%%%%%%%%%%%%%%%%%%%

\lhead{\sf \large \Fach{} \\ \small \Name{} - \Matrikelnummer{}}
%\rhead{\sf \Semester{}}
\rhead{\sf \FachKurz{} \quad E \Kurseinheit{}/\AufgabeNummer{}}

%%%%%%%%%%%%%%%%%%%%%%%%%%%%%%%%%%%%%%%%%%%%%%%%%%%%%%
%%%%%%%%%%%%%% START HERE %%%%%%%%%%%%%%%%%%%%%%%%%%%%
%%%%%%%%%%%%%%%%%%%%%%%%%%%%%%%%%%%%%%%%%%%%%%%%%%%%%%

\Aufgabe{\AufgabeNummer{}}

Wir wenden das \textsc{Gram}-\textsc{Schmidt}-Verfahren an. Dabei nutzen wir die \textsc{Euklid}ische Norm $||{\cdot}||_2$.
\begin{align*}
\text{S1.} \quad &\\
\mathrlap{\pmb{u}^1 := \displaystyle\frac{\pmb{a}^1}{\underbrace{||\pmb{a}^1||_2}_{=:r_{11}}} = 
\begin{pmatrix}
    \frac{i}{2}\\
    \frac{1}{2}\\
    0\\
    -\frac{i}{2}\\
    \frac{1}{2}
\end{pmatrix}
, \quad r_{11} = 4}\\
\text{S2.} \quad & \\
&i = 2: &&\\
&&&\tilde{\pmb{u}}^2 := \pmb{a}^2 - \underbrace{\langle \pmb{a}^2, \pmb{u}^1 \rangle}_{=: r_{12}} \pmb{u}^1 =
\begin{pmatrix}
    1+i \\
    0 \\
    0 \\0 \\
    i-1
\end{pmatrix}, \quad
\pmb{u}^2 := \displaystyle\frac{\tilde{\pmb{u}}^2}{\underbrace{||\tilde{\pmb{u}}^2||_2}_{=: r_{22}}} = 
\begin{pmatrix}
    \frac{1}{2} + \frac{i}{2}\\
    0\\
    0\\
    0\\
    -\frac{1}{2} + \frac{i}{2}
\end{pmatrix}, \\
&&&r_{12} = 2 + 2i, \quad
r_{22} =  2.\\
&i = 3: &&\\
&&&\tilde{\pmb{u}}^3 := \pmb{a}^3 - \underbrace{\langle \pmb{a}^3, \pmb{u}^1 \rangle}_{=: r_{13}} \pmb{u}^1 -  \underbrace{\langle \pmb{a}^3, \pmb{u}^2 \rangle}_{=: r_{23}} \pmb{u}^2=
\begin{pmatrix}
    0 \\
    3i \\
    3+3i \\
    -3 \\
    0
\end{pmatrix}, \quad \\
&&&\pmb{u}^3 := \displaystyle\frac{\tilde{\pmb{u}}^3}{\underbrace{||\tilde{\pmb{u}}^3||_2}_{=: r_{33}}} = 
\begin{pmatrix}
    0\\
    \frac{i}{2}\\
    \frac{1}{2} + \frac{i}{2}\\
    -\frac{1}{2}\\
    0
\end{pmatrix}, \quad
r_{13} = 0, \quad
r_{23} = 2i, \quad 
r_{33} = 6.
\end{align*}
\textit{Ergebnis:}
\begin{align*}
\pmb{Q}_0 &= (\pmb{u}^1 \quad \pmb{u}^2 \quad \pmb{u}^3) = 
\begin{pmatrix}
    \frac{i}{2} &  \frac{1}{2} + \frac{i}{2} & 0                         \\
    \frac{1}{2}  &  0   & \frac{i}{2}               \\
    0  &  0   & \frac{1}{2} + \frac{i}{2} \\
   -\frac{i}{2} &  0   & - \frac{1}{2}             \\
    \frac{1}{2}  & -\frac{1}{2} + \frac{i}{2} & 0
\end{pmatrix},\\
\pmb{R} &= \pmb{R}_0 = 
\begin{pmatrix}
    4 & 2+2i & 0\\
    0 & 2 & 2i\\
    0 & 0 & 6
\end{pmatrix} \left [ = \pmb{Q}^H_0 \pmb{A} \right ].
\end{align*}
Nun lösen wir $\pmb{R}_0 \pmb{x} = \tilde{\pmb{b}}_0 = \pmb{Q}^H_0 \pmb{b}$ (siehe Bemerkung 4.2.2):
\begin{align*}
&\begin{pmatrix}
    4 & 2+2i & 0 \\
    0 & 2 & 2i \\
    0 & 0 & 6
\end{pmatrix}
\begin{pmatrix}
    x_1 \\
    x_2 \\
    x_3
\end{pmatrix}
=
\begin{pmatrix}
    6i - 2 \\
    2i - 2 \\
    6
\end{pmatrix} \quad \Longrightarrow \quad %x_3 = 1, x_2 = -1, x_1 = 2i. \\
\pmb{x} = 
\begin{pmatrix}
    2i \\
    -1 \\
    1
\end{pmatrix}.
\end{align*}
Das zugehörige (minimale) Residuum ist: $$||\pmb{Ax} - \pmb{b}||_2 = \left( ||\pmb{b}||_2^2 - ||\tilde{\pmb{b}}_0||_2^2 \right)^{\frac{1}{2}}=\sqrt{12}.$$
\end{document}
