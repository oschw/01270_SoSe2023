\documentclass[a4paper,12pt]{article}
\usepackage{fancyhdr}
\usepackage[ngerman,german]{babel}
\usepackage{german}
\usepackage[utf8]{inputenc}
\usepackage[active]{srcltx}
\usepackage{algorithm}
\usepackage[noend]{algorithmic}
\usepackage{amsmath}
\usepackage{amssymb}
\usepackage{amsthm}
\usepackage{bbm}
\usepackage{enumerate}
\usepackage{graphicx}
\usepackage{ifthen}
\usepackage{listings}
\usepackage{struktex}
\usepackage{hyperref}
\usepackage[onehalfspacing]{setspace}
\usepackage{geometry}
\usepackage{calc}

%%%%%%%%%%%%%%%%%%%%%%%%%%%%%%%%%%%%%%%%%%%%%%%%%%%%%%
%%%%%%%%%%%%%% EDIT THIS PART %%%%%%%%%%%%%%%%%%%%%%%%
%%%%%%%%%%%%%%%%%%%%%%%%%%%%%%%%%%%%%%%%%%%%%%%%%%%%%%

\newcommand{\Fach}{01270 Numerische Mathematik I}
\newcommand{\FachKurz}{NMI}
\newcommand{\Name}{Oliver Schwarz}
\newcommand{\Matrikelnummer}{6883389}
\newcommand{\Semester}{SoSe 23}
\newcommand{\Kurseinheit}{1}
\newcommand{\AufgabeNummer}{2}

%%%%%%%%%%%%%%%%%%%%%%%%%%%%%%%%%%%%%%%%%%%%%%%%%%%%%%
%%%%%%%%%%%%%% DO NOT EDIT THIS PART %%%%%%%%%%%%%%%%%
%%%%%%%%%%%%%%%%%%%%%%%%%%%%%%%%%%%%%%%%%%%%%%%%%%%%%%

\newcommand{\Aufgabe}[1]{
  {
  \vspace*{0.5cm}
  \textbf{Aufgabe #1}
  \vspace*{0.2cm}
  }
}

%%%%%%%%%%%%%%%%%%%%%%%%%%%%%%%%%%%%%%%%%%%%%%%%%%%%%%
%%%%%%%%%%%%%% PAGE SETTINGS %%%%%%%%%%%%%%%%%%%%%%%%%
%%%%%%%%%%%%%%%%%%%%%%%%%%%%%%%%%%%%%%%%%%%%%%%%%%%%%%

\setlength{\parindent}{0em}
\topmargin 0cm
\oddsidemargin 0cm
\evensidemargin 0cm

\geometry{%
  left=45.0mm,
  right=15.0mm,
  top=25mm,
  bottom=25mm,
  bindingoffset=0mm,
  headheight=30pt,
  includehead
}

\fancyheadoffset[L]{20mm}
\renewcommand{\headrulewidth}{1pt}

\pagestyle{fancy}

%%%%%%%%%%%%%%%%%%%%%%%%%%%%%%%%%%%%%%%%%%%%%%%%%%%%%%
%%%%%%%%%%%%%% PDF SETTINGS %%%%%%%%%%%%%%%%%%%%%%%%%%
%%%%%%%%%%%%%%%%%%%%%%%%%%%%%%%%%%%%%%%%%%%%%%%%%%%%%%

\hypersetup{
    %pdftitle={\Fach{}: Übungsblatt \Uebungsblatt{}},
    pdftitle={\Fach{}: Kurseinheit \Kurseinheit{} Aufgabe \AufgabeNummer{}},
    pdfauthor={\Name},
    pdfborder={0 0 0}
}

%%%%%%%%%%%%%%%%%%%%%%%%%%%%%%%%%%%%%%%%%%%%%%%%%%%%%%
%%%%%%%%%%%%%% CODE SETTINGS %%%%%%%%%%%%%%%%%%%%%%%%%
%%%%%%%%%%%%%%%%%%%%%%%%%%%%%%%%%%%%%%%%%%%%%%%%%%%%%%

\lstset{ %
language=java,
basicstyle=\footnotesize\tt,
showtabs=false,
tabsize=2,
captionpos=b,
breaklines=true,
extendedchars=true,
showstringspaces=false,
flexiblecolumns=true,
}

%%%%%%%%%%%%%%%%%%%%%%%%%%%%%%%%%%%%%%%%%%%%%%%%%%%%%%
%%%%%%%%%%%%%% DOCUMENT %%%%%%%%%%%%%%%%%%%%%%%%%%%%%%
%%%%%%%%%%%%%%%%%%%%%%%%%%%%%%%%%%%%%%%%%%%%%%%%%%%%%%

\title{Kurseinheit \Kurseinheit{} Aufgabe \AufgabeNummer{}}
\author{\Name{}}

\begin{document}

%\renewcommand{\theequation}{L\Kurseinheit{}.\AufgabeNummer{}.\arabic{equation}}
\renewcommand{\theequation}{L \AufgabeNummer{}.\arabic{equation}}

%%%%%%%%%%%%%%%%%%%%%%%%%%%%%%%%%%%%%%%%%%%%%%%%%%%%%%
%%%%%%%%%%%%%% HEADER %%%%%%%%%%%%%%%%%%%%%%%%%%%%%%%%
%%%%%%%%%%%%%%%%%%%%%%%%%%%%%%%%%%%%%%%%%%%%%%%%%%%%%%

\lhead{\sf \large \Fach{} \\ \small \Name{} - \Matrikelnummer{}}
%\rhead{\sf \Semester{}}
\rhead{\sf \FachKurz{} \quad E \Kurseinheit{}/\AufgabeNummer{}}

%%%%%%%%%%%%%%%%%%%%%%%%%%%%%%%%%%%%%%%%%%%%%%%%%%%%%%
%%%%%%%%%%%%%% START HERE %%%%%%%%%%%%%%%%%%%%%%%%%%%%
%%%%%%%%%%%%%%%%%%%%%%%%%%%%%%%%%%%%%%%%%%%%%%%%%%%%%%

\Aufgabe{\AufgabeNummer{} \quad (Gleitkommadarstellung)} 

\begin{enumerate}[a)]
\item % a)
\begin{enumerate}[\itshape(i)]
    \item % (i) 133.7
        \begin{equation*}
        133.7 = + \, 0.1337 \, \times \, 10^3 \, \notin \mathbb{M}_{10}(4, [-2, 2]),
        \end{equation*}
        da für den Exponenten $e \in [-2, 2]$ gilt und $3 \notin [-2, 2]$. Es findet ein \textit{Überlauf} statt.
    
    \item % (ii) 0.0001
        \begin{align*}
            0.0001 
             &= + \, 0.0001 \, \times \, 10^{0} \\%\, \in \mathbb{M}_{10}(4, [-2, 2]) \\
             &= + \, 0.0010 \, \times \, 10^{-1} \\%\, \in \mathbb{M}_{10}(4, [-2, 2]) \\
             &= + \, 0.0100 \, \times \, 10^{-2} \, \in \mathbb{\mathring{M}}_{10}(4, [-2, 2]) \setminus \mathbb{M}_{10}(4, [-2, 2])
        \end{align*}
        Die Zahl gehört somit zu den \textit{denormalisierten Gleitkommazahlen}.
\end{enumerate}

\item % b)
\begin{enumerate}[\itshape(i)]
    \item % (i) 16/27
    
    \begin{equation*}
      \frac{16}{27} = \frac{5}{9} + \frac{1}{27} = \frac{1}{3} + \frac{2}{9} + \frac{1}{27}
    \end{equation*}
    \begin{equation*}
      \bigg( \frac{16}{27} \bigg)_{10} = (0.121)_{3} = (0.\overline{592})_{10}
    \end{equation*}
    
    Die 3-adische Darstellung ist endlich; die 10-adische Darstellung ist es nicht.
    
    \item % (ii) 0.0001

    \begin{align*}
        0.3 = \frac{3}{10} &=  \frac{2}{3^2} + \frac{7}{3^2 \times 10}\\
        \frac{7}{3^2 \times 10} &= \frac{2}{3^3} + \frac{1}{3^3 \times 10}\\
        \frac{1}{3^3 \times 10} &= \frac{2}{3^6} + \frac{7}{3^6 \times 10}\\
        \frac{7}{3^6 \times 10} &= \frac{2}{3^7} + \frac{1}{3^7 \times 10}
    \end{align*}
 
    \begin{equation*}
        (0.3)_{10} = (0.\overline{0220})_{3}
    \end{equation*}
    
    Die 10-adische Darstellung ist endlich; die 3-adische Darstellung ist es nicht.
\end{enumerate}

\item % c)
Betrachtet werden die normalisierten $(i)$ und denormalisierten $(ii)$ Gleitkommazahlen mit Mantissenlänge $m = 2$, Basis $b = 3$ und Exponentenbereich $e \in \{-1, 0\}$. Die in der Darstellung möglichen Ziffern sind 0, 1 und 2. Bei $(i)$ muss die erste Nachkommastelle der Mantisse von 0 verschieden sein.

In den unten stehenden Tabellen stehen alle vorkommenden Mantissen in der ersten Spalte und deren Werte in der zweiten Spalte. Die Ergebnisse der Multiplikation von Mantisse und $3^e$ stehen in der dritten und vierten Spalte.

\begin{enumerate}[\itshape(i)]
    \item % (i)

    $\mathbb{M}_3(2,[-1,0])$
    \begin{equation*}    
        \begin{array}[t]{| c | l | c | c |}\hline
            m    & x = \frac{\alpha_1}{3} + \frac{\alpha_2}{9} & x \cdot 3^{-1} & x \cdot 3^{0} \\\hline
            0.10 & \frac{1}{3}                                 & \frac{1}{9}    & \frac{1}{3} \\
            0.20 & \frac{2}{3}                                 & \frac{2}{9}    & \frac{2}{3} \\
            0.11 & \frac{1}{3} + \frac{1}{9} = \frac{4}{9}     & \frac{4}{27}   & \frac{4}{9} \\
            0.21 & \frac{2}{3} + \frac{1}{9} = \frac{7}{9}     & \frac{7}{27}   & \frac{7}{9} \\
            0.12 & \frac{1}{3} + \frac{2}{9} = \frac{5}{9}     & \frac{5}{27}   & \frac{5}{9} \\
            0.22 & \frac{2}{3} + \frac{2}{9} = \frac{8}{9}     & \frac{8}{27}   & \frac{8}{9} \\\hline
        \end{array}
    \end{equation*}

    Zusätzlich gehört noch die 0 zu $\mathbb{M}_3(2,[-1,0])$.
    \item % (ii)
   
    $\mathring{\mathbb{M}}_3(2,[-1,0])$
    \begin{equation*}    
        \begin{array}[t]{| c | l | c | c |}\hline
            m    & x = \frac{\alpha_1}{3} + \frac{\alpha_2}{9} & x \cdot 3^{-1} & x \cdot 3^{0} \\\hline
            0.00 & 0                                           & 0              & 0 \\
            0.10 & \frac{1}{3}                                 & \frac{1}{9}    & \frac{1}{3} \\
            0.20 & \frac{2}{3}                                 & \frac{2}{9}    & \frac{2}{3} \\
            0.01 & \frac{1}{9}                                 & \frac{1}{27}   & \frac{1}{9} \\
            0.11 & \frac{1}{3} + \frac{1}{9} = \frac{4}{9}     & \frac{4}{27}   & \frac{4}{9} \\
            0.21 & \frac{2}{3} + \frac{1}{9} = \frac{7}{9}     & \frac{7}{27}   & \frac{7}{9} \\
            0.02 & \frac{2}{9}                                 & \frac{2}{27}   & \frac{2}{9} \\
            0.12 & \frac{1}{3} + \frac{2}{9} = \frac{5}{9}     & \frac{5}{27}   & \frac{5}{9} \\
            0.22 & \frac{2}{3} + \frac{2}{9} = \frac{8}{9}     & \frac{8}{27}   & \frac{8}{9} \\\hline
        \end{array}
    \end{equation*}

\end{enumerate}
\end{enumerate}

\end{document}
