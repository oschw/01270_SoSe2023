\documentclass[a4paper,12pt]{article}
\usepackage{fancyhdr}
\usepackage[ngerman,german]{babel}
\usepackage{german}
\usepackage[utf8]{inputenc}
\usepackage[active]{srcltx}
\usepackage{algorithm}
\usepackage[noend]{algorithmic}
\usepackage{amsmath}
\usepackage{amssymb}
\usepackage{amsthm}
\usepackage{bbm}
\usepackage{enumerate}
\usepackage{graphicx}
\usepackage{ifthen}
\usepackage{listings}
\usepackage{struktex}
\usepackage{hyperref}
\usepackage[onehalfspacing]{setspace}
\usepackage{geometry}
\usepackage{calc}

%%%%%%%%%%%%%%%%%%%%%%%%%%%%%%%%%%%%%%%%%%%%%%%%%%%%%%
%%%%%%%%%%%%%% EDIT THIS PART %%%%%%%%%%%%%%%%%%%%%%%%
%%%%%%%%%%%%%%%%%%%%%%%%%%%%%%%%%%%%%%%%%%%%%%%%%%%%%%

\newcommand{\Fach}{01270 Numerische Mathematik I}
\newcommand{\FachKurz}{NMI}
\newcommand{\Name}{Oliver Schwarz}
\newcommand{\Matrikelnummer}{6883389}
\newcommand{\Semester}{SoSe 23}
\newcommand{\Kurseinheit}{5}
\newcommand{\AufgabeNummer}{6}

%%%%%%%%%%%%%%%%%%%%%%%%%%%%%%%%%%%%%%%%%%%%%%%%%%%%%%
%%%%%%%%%%%%%% DO NOT EDIT THIS PART %%%%%%%%%%%%%%%%%
%%%%%%%%%%%%%%%%%%%%%%%%%%%%%%%%%%%%%%%%%%%%%%%%%%%%%%

\newcommand{\Aufgabe}[1]{
  {
  \vspace*{0.5cm}
  \textbf{Aufgabe #1}
  \vspace*{0.2cm}
  }
}

\newcommand{\cond}[0]{
  {
  \textrm{cond}
  }
}

% Beschreibung
% Unterscheidet im Mathe-Mode zwischen
% "Dezimal-komma" und Satzzeichen.
% Muss bei Eingabe durch kein/ein Leerzeichen
% dahinter angegeben werden.
% Literatur Richard Hirsch in: DTK, 1/1994, S. 42 ff.
\mathchardef\CommaOrdinary="013B \mathchardef\CommaPunct="613B
\mathcode`,="8000 % , im Math-Mode aktiv ("8000) machen
{\catcode`\,=\active \gdef
,{\obeyspaces\futurelet\next\CommaCheck}}
\def\CommaCheck{\if\space\next\CommaPunct\else\CommaOrdinary\fi}

\newcommand*\diff{\mathop{}\!\mathrm{\,d}}
%%%%%%%%%%%%%%%%%%%%%%%%%%%%%%%%%%%%%%%%%%%%%%%%%%%%%%
%%%%%%%%%%%%%% PAGE SETTINGS %%%%%%%%%%%%%%%%%%%%%%%%%
%%%%%%%%%%%%%%%%%%%%%%%%%%%%%%%%%%%%%%%%%%%%%%%%%%%%%%

\setlength{\parindent}{0em}
\topmargin 0cm
\oddsidemargin 0cm
\evensidemargin 0cm

\geometry{%
  left=45.0mm,
  right=15.0mm,
  top=25mm,
  bottom=25mm,
  bindingoffset=0mm,
  headheight=30pt,
  includehead
}

\fancyheadoffset[L]{20mm}
\renewcommand{\headrulewidth}{1pt}

\pagestyle{fancy}

%%%%%%%%%%%%%%%%%%%%%%%%%%%%%%%%%%%%%%%%%%%%%%%%%%%%%%
%%%%%%%%%%%%%% PDF SETTINGS %%%%%%%%%%%%%%%%%%%%%%%%%%
%%%%%%%%%%%%%%%%%%%%%%%%%%%%%%%%%%%%%%%%%%%%%%%%%%%%%%

\hypersetup{
    %pdftitle={\Fach{}: Übungsblatt \Uebungsblatt{}},
    pdftitle={\Fach{}: Kurseinheit \Kurseinheit{} Aufgabe \AufgabeNummer{}},
    pdfauthor={\Name},
    pdfborder={0 0 0}
}

%%%%%%%%%%%%%%%%%%%%%%%%%%%%%%%%%%%%%%%%%%%%%%%%%%%%%%
%%%%%%%%%%%%%% CODE SETTINGS %%%%%%%%%%%%%%%%%%%%%%%%%
%%%%%%%%%%%%%%%%%%%%%%%%%%%%%%%%%%%%%%%%%%%%%%%%%%%%%%

\lstset{ %
language=java,
basicstyle=\footnotesize\tt,
showtabs=false,
tabsize=2,
captionpos=b,
breaklines=true,
extendedchars=true,
showstringspaces=false,
flexiblecolumns=true,
}

%%%%%%%%%%%%%%%%%%%%%%%%%%%%%%%%%%%%%%%%%%%%%%%%%%%%%%
%%%%%%%%%%%%%% DOCUMENT %%%%%%%%%%%%%%%%%%%%%%%%%%%%%%
%%%%%%%%%%%%%%%%%%%%%%%%%%%%%%%%%%%%%%%%%%%%%%%%%%%%%%

\title{Kurseinheit \Kurseinheit{} Aufgabe \AufgabeNummer{}}
\author{\Name{}}

\begin{document}

%\renewcommand{\theequation}{L\Kurseinheit{}.\AufgabeNummer{}.\arabic{equation}}
\renewcommand{\theequation}{L \AufgabeNummer{}.\arabic{equation}}

%%%%%%%%%%%%%%%%%%%%%%%%%%%%%%%%%%%%%%%%%%%%%%%%%%%%%%
%%%%%%%%%%%%%% HEADER %%%%%%%%%%%%%%%%%%%%%%%%%%%%%%%%
%%%%%%%%%%%%%%%%%%%%%%%%%%%%%%%%%%%%%%%%%%%%%%%%%%%%%%

\lhead{\sf \large \Fach{} \\ \small \Name{} - \Matrikelnummer{}}
%\rhead{\sf \Semester{}}
\rhead{\sf \FachKurz{} \quad E \Kurseinheit{}/\AufgabeNummer{}}

%%%%%%%%%%%%%%%%%%%%%%%%%%%%%%%%%%%%%%%%%%%%%%%%%%%%%%
%%%%%%%%%%%%%% START HERE %%%%%%%%%%%%%%%%%%%%%%%%%%%%
%%%%%%%%%%%%%%%%%%%%%%%%%%%%%%%%%%%%%%%%%%%%%%%%%%%%%%

\Aufgabe{\AufgabeNummer{}}
\begin{enumerate}[a)]
    \item %a
Sei $p := \phi \psi$, $p \in \Pi_{n,m}$, $\phi \in \Pi_n$ und $\psi \in \Pi_m$. Die \textsc{Lagrange}schen Grundpolynome zu den Stützstellen $(x_k, y_\ell)$ sind
\begin{equation*}
    \phi_k(x) := \prod\limits_{\substack{i=0 \\ i \neq k}}^{n} \frac{x - x_i}{x_k - x_i} \quad \textrm{und} \quad 
    \psi_\ell(x) := \prod\limits_{\substack{j=0 \\ j \neq \ell}}^{m} \frac{y - y_j}{y_\ell - y_j}.
\end{equation*}

Nach 6.4 sind $\phi_k(x_i) = \delta_{ki}$ und $\psi_\ell(y_j) = \delta_{\ell j}$. Also ist
\begin{equation}
    p_{k\ell}(x_k, y_\ell) = z_{k\ell} = \left\{
 \begin{aligned}
	 &1 \quad \textrm{für } (k, \ell) = (i, j), \\
  &0 \quad \textrm{für } (k, \ell) \neq (i, j).
	\end{aligned}
	\right. \label{eqn_1}
\end{equation}

Hiermit können wir analog zu 6.6 und mit \ref{eqn_1} setzen:
\begin{equation}
    p(x_k,y_\ell) = \sum\limits_{k=0}^{n}\sum\limits_{\ell=0}^{m} f_{k,\ell} p_{k\ell}(x_k,y_\ell) = f(x_k, y_\ell). \label{eqn_2}
\end{equation}

Folglich erfüllt $p(x_k,y_\ell)$ die Interpolationsbedingungen und die Interpolationsaufgabe ist lösbar.

\item %b)

Es ist für $k=0,\dots,n$ und $\ell=0,\dots,m$
\begin{equation}
    f(x_k,y_{\ell}) = \sum\limits_{i=0}^{n}\sum\limits_{j=0}^{m} a_{ij} x_k^i y_{\ell}^j. \label{eqn_3}
\end{equation}

Wir wenden Satz 5.1.8.(i) an für $n+1$ bzw. $m+1$ paarweise verschiedene Stützstellen.

Für den Fall $(m+1)$ erhalten wir aus \ref{eqn_3}:
\begin{align*}
    f(x_k,y_{\ell}) =    
    \sum\limits_{i=0}^{n}\underbrace{\left(\sum\limits_{j=0}^{m} a_{ij} y_{\ell}^j\right)}_{= \upsilon_{i\ell}} x_k^i = \sum\limits_{i=0}^{n}\upsilon_{i\ell} x_k^i.
\end{align*}
Gesucht ist für jedes feste $\ell$ mit $0 \leq \ell \leq m$ (also für jedes feste $y_{\ell}$) ein Polynom $\varphi_{\ell} \in \Pi_n$, das die Interpolationsbedingungen $$\varphi_{\ell}(x) = \sum\limits_{i=0}^{n}\upsilon_{i\ell} x^i \quad \textrm{und} \quad \varphi_{\ell}(x_k) = f(x_k,y_{\ell}), \quad k=0,1,\dots,n$$ erfüllt.

Nach der vorangegangenen Teilaufgabe ist diese Interpolationsaufgabe lösbar und nach Satz 5.1.8.(i) sind die $\varphi_{\ell}$ eindeutig bestimmt.

Für den Fall $(n+1)$ suchen wir nun für jedes $i$ mit $0 \leq i \leq n$ ein Polynom $\chi_i \in \Pi_m$, das die Interpolationsbedingungen
$$\chi_i(y) = \sum\limits_{j=0}^{m} a_{ij} y^j \quad \textrm{und} \quad \chi_i(y_{\ell}) = \upsilon_{i\ell}, \quad \ell=0,1,\dots,m$$ erfüllt.

Diese Interpolationsaufgabe ist lösbar und die $a_{ij}$ sind somit eindeutig bestimmt.

Hieraus folgt, dass höchstens eine Lösung existiert.
\end{enumerate}


\end{document}
