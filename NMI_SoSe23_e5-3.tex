\documentclass[a4paper,12pt]{article}
\usepackage{fancyhdr}
\usepackage[ngerman,german]{babel}
\usepackage{german}
\usepackage[utf8]{inputenc}
\usepackage[active]{srcltx}
\usepackage{algorithm}
\usepackage[noend]{algorithmic}
\usepackage{amsmath}
\usepackage{amssymb}
\usepackage{amsthm}
\usepackage{bbm}
\usepackage{enumerate}
\usepackage{graphicx}
\usepackage{ifthen}
\usepackage{listings}
\usepackage{struktex}
\usepackage{hyperref}
\usepackage[onehalfspacing]{setspace}
\usepackage{geometry}
\usepackage{calc}

%%%%%%%%%%%%%%%%%%%%%%%%%%%%%%%%%%%%%%%%%%%%%%%%%%%%%%
%%%%%%%%%%%%%% EDIT THIS PART %%%%%%%%%%%%%%%%%%%%%%%%
%%%%%%%%%%%%%%%%%%%%%%%%%%%%%%%%%%%%%%%%%%%%%%%%%%%%%%

\newcommand{\Fach}{01270 Numerische Mathematik I}
\newcommand{\FachKurz}{NMI}
\newcommand{\Name}{Oliver Schwarz}
\newcommand{\Matrikelnummer}{6883389}
\newcommand{\Semester}{SoSe 23}
\newcommand{\Kurseinheit}{5}
\newcommand{\AufgabeNummer}{3}

%%%%%%%%%%%%%%%%%%%%%%%%%%%%%%%%%%%%%%%%%%%%%%%%%%%%%%
%%%%%%%%%%%%%% DO NOT EDIT THIS PART %%%%%%%%%%%%%%%%%
%%%%%%%%%%%%%%%%%%%%%%%%%%%%%%%%%%%%%%%%%%%%%%%%%%%%%%

\newcommand{\Aufgabe}[1]{
  {
  \vspace*{0.5cm}
  \textbf{Aufgabe #1}
  \vspace*{0.2cm}
  }
}

\newcommand{\cond}[0]{
  {
  \textrm{cond}
  }
}

% Beschreibung
% Unterscheidet im Mathe-Mode zwischen
% "Dezimal-komma" und Satzzeichen.
% Muss bei Eingabe durch kein/ein Leerzeichen
% dahinter angegeben werden.
% Literatur Richard Hirsch in: DTK, 1/1994, S. 42 ff.
\mathchardef\CommaOrdinary="013B \mathchardef\CommaPunct="613B
\mathcode`,="8000 % , im Math-Mode aktiv ("8000) machen
{\catcode`\,=\active \gdef
,{\obeyspaces\futurelet\next\CommaCheck}}
\def\CommaCheck{\if\space\next\CommaPunct\else\CommaOrdinary\fi}

\newcommand*\diff{\mathop{}\!\mathrm{\,d}}

\newcommand*\euler{\mathrm{e}}

\newcommand\redarrow{%
        \mathrel{\vcenter{\mathsurround0pt
                \ialign{##\crcr
                        \noalign{\nointerlineskip}$\searrow\vspace{2.5pt}$\crcr 
                        \noalign{\nointerlineskip}$\rightarrow$\crcr
                        \noalign{\nointerlineskip}$\vspace{12.5pt}$\crcr
                }%
        }}%
}
%%%%%%%%%%%%%%%%%%%%%%%%%%%%%%%%%%%%%%%%%%%%%%%%%%%%%%
%%%%%%%%%%%%%% PAGE SETTINGS %%%%%%%%%%%%%%%%%%%%%%%%%
%%%%%%%%%%%%%%%%%%%%%%%%%%%%%%%%%%%%%%%%%%%%%%%%%%%%%%

\setlength{\parindent}{0em}
\topmargin 0cm
\oddsidemargin 0cm
\evensidemargin 0cm

\geometry{%
  left=45.0mm,
  right=15.0mm,
  top=25mm,
  bottom=25mm,
  bindingoffset=0mm,
  headheight=30pt,
  includehead
}

\fancyheadoffset[L]{20mm}
\renewcommand{\headrulewidth}{1pt}

\pagestyle{fancy}

%%%%%%%%%%%%%%%%%%%%%%%%%%%%%%%%%%%%%%%%%%%%%%%%%%%%%%
%%%%%%%%%%%%%% PDF SETTINGS %%%%%%%%%%%%%%%%%%%%%%%%%%
%%%%%%%%%%%%%%%%%%%%%%%%%%%%%%%%%%%%%%%%%%%%%%%%%%%%%%

\hypersetup{
    %pdftitle={\Fach{}: Übungsblatt \Uebungsblatt{}},
    pdftitle={\Fach{}: Kurseinheit \Kurseinheit{} Aufgabe \AufgabeNummer{}},
    pdfauthor={\Name},
    pdfborder={0 0 0}
}

%%%%%%%%%%%%%%%%%%%%%%%%%%%%%%%%%%%%%%%%%%%%%%%%%%%%%%
%%%%%%%%%%%%%% CODE SETTINGS %%%%%%%%%%%%%%%%%%%%%%%%%
%%%%%%%%%%%%%%%%%%%%%%%%%%%%%%%%%%%%%%%%%%%%%%%%%%%%%%

\lstset{ %
language=java,
basicstyle=\footnotesize\tt,
showtabs=false,
tabsize=2,
captionpos=b,
breaklines=true,
extendedchars=true,
showstringspaces=false,
flexiblecolumns=true,
}

%%%%%%%%%%%%%%%%%%%%%%%%%%%%%%%%%%%%%%%%%%%%%%%%%%%%%%
%%%%%%%%%%%%%% DOCUMENT %%%%%%%%%%%%%%%%%%%%%%%%%%%%%%
%%%%%%%%%%%%%%%%%%%%%%%%%%%%%%%%%%%%%%%%%%%%%%%%%%%%%%

\title{Kurseinheit \Kurseinheit{} Aufgabe \AufgabeNummer{}}
\author{\Name{}}

\begin{document}

%\renewcommand{\theequation}{L\Kurseinheit{}.\AufgabeNummer{}.\arabic{equation}}
\renewcommand{\theequation}{L \AufgabeNummer{}.\arabic{equation}}

%%%%%%%%%%%%%%%%%%%%%%%%%%%%%%%%%%%%%%%%%%%%%%%%%%%%%%
%%%%%%%%%%%%%% HEADER %%%%%%%%%%%%%%%%%%%%%%%%%%%%%%%%
%%%%%%%%%%%%%%%%%%%%%%%%%%%%%%%%%%%%%%%%%%%%%%%%%%%%%%

\lhead{\sf \large \Fach{} \\ \small \Name{} - \Matrikelnummer{}}
%\rhead{\sf \Semester{}}
\rhead{\sf \FachKurz{} \quad E \Kurseinheit{}/\AufgabeNummer{}}

%%%%%%%%%%%%%%%%%%%%%%%%%%%%%%%%%%%%%%%%%%%%%%%%%%%%%%
%%%%%%%%%%%%%% START HERE %%%%%%%%%%%%%%%%%%%%%%%%%%%%
%%%%%%%%%%%%%%%%%%%%%%%%%%%%%%%%%%%%%%%%%%%%%%%%%%%%%%

\Aufgabe{\AufgabeNummer{}}

Gegeben sind die beiden Funktionen $f, g \in C[a,b]$ und $n+1$ paarweise verschiedene Stützstellen $\{x_i\}^n_{i=0} \subset [a, b]$, $n \in \mathbb{N}_0$. Für die Interpolation von $f, g$ nutzen wir die \textsc{Newton}schen Basispolynome $N_i(x)$ bzw. $\hat{N}_i(x)$:
\begin{align*}
    N_i(x) &= \prod\limits_{k=0}^{i-1} (x - x_k), \\
    \hat{N}_i(x) &= \prod\limits_{k=i+1}^{n} (x - x_k).
\end{align*}

Mit den obigen Stützstellen können wir zu $f$ und $g$ die Interpolationspolynome $p \in \Pi_n$ und $q \in \Pi_n$ bilden:
\begin{align*}
    p(x) &:= f[x_0]N_0 + f[x_0, x_1]N_1 + \dots + f[x_0, x_1, \dots, x_n]N_n = \sum\limits_{i=0}^{n} f[x_0, \dots, x_i] N_i(x), \\% \textrm{ und} \\
    q(x) &:= g[x_0, \dots x_n]\hat{N}_0 + g[x_1, \dots, x_n]\hat{N}_1 + \dots + g[x_n]\hat{N}_n =  \sum\limits_{i=0}^{n} g[x_i, \dots, x_n] \hat{N}_i(x).
\end{align*}

Wir bilden das Produkt $fg$. Dann wird $fg$ durch $pq \in \Pi_{2n}$ interpoliert:
\begin{align*}
    p(x)q(x) &= \sum\limits_{i=0}^{n} f[x_0, \dots, x_i] N_i(x) \cdot \sum\limits_{j=0}^{n} g[x_j, \dots, x_n] \hat{N}_j(x) \\
    &=  \sum\limits_{i=0}^{n}  \sum\limits_{j=0}^{n}  f[x_0, \dots, x_i] g[x_j, \dots, x_n] N_i(x) \hat{N}_j(x).
\end{align*}

Der Höchstkoeffizient $(fg)[x_1, \dots, x_n]$ ist also
 $$(fg)[x_1, \dots, x_n] =  \sum\limits_{i=0}^{n}  f[x_0, \dots, x_i] g[x_j, \dots, x_n].$$


\end{document}
