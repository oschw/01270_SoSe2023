\documentclass[a4paper,12pt]{article}
\usepackage{fancyhdr}
\usepackage[ngerman,german]{babel}
\usepackage{german}
\usepackage[utf8]{inputenc}
\usepackage[active]{srcltx}
\usepackage{algorithm}
\usepackage[noend]{algorithmic}
\usepackage{amsmath}
\usepackage{amssymb}
\usepackage{amsthm}
\usepackage{bbm}
\usepackage{enumerate}
\usepackage{graphicx}
\usepackage{ifthen}
\usepackage{listings}
\usepackage{struktex}
\usepackage{hyperref}
\usepackage[onehalfspacing]{setspace}
\usepackage{geometry}
\usepackage{calc}

%%%%%%%%%%%%%%%%%%%%%%%%%%%%%%%%%%%%%%%%%%%%%%%%%%%%%%
%%%%%%%%%%%%%% EDIT THIS PART %%%%%%%%%%%%%%%%%%%%%%%%
%%%%%%%%%%%%%%%%%%%%%%%%%%%%%%%%%%%%%%%%%%%%%%%%%%%%%%

\newcommand{\Fach}{01270 Numerische Mathematik I}
\newcommand{\FachKurz}{NMI}
\newcommand{\Name}{Oliver Schwarz}
\newcommand{\Matrikelnummer}{6883389}
\newcommand{\Semester}{SoSe 23}
\newcommand{\Kurseinheit}{4}
\newcommand{\AufgabeNummer}{5}

%%%%%%%%%%%%%%%%%%%%%%%%%%%%%%%%%%%%%%%%%%%%%%%%%%%%%%
%%%%%%%%%%%%%% DO NOT EDIT THIS PART %%%%%%%%%%%%%%%%%
%%%%%%%%%%%%%%%%%%%%%%%%%%%%%%%%%%%%%%%%%%%%%%%%%%%%%%

\newcommand{\Aufgabe}[1]{
  {
  \vspace*{0.5cm}
  \textbf{Aufgabe #1}
  \vspace*{0.2cm}
  }
}

\newcommand{\cond}[0]{
  {
  \textrm{cond}
  }
}

% Beschreibung
% Unterscheidet im Mathe-Mode zwischen
% "Dezimal-komma" und Satzzeichen.
% Muss bei Eingabe durch kein/ein Leerzeichen
% dahinter angegeben werden.
% Literatur Richard Hirsch in: DTK, 1/1994, S. 42 ff.
\mathchardef\CommaOrdinary="013B \mathchardef\CommaPunct="613B
\mathcode`,="8000 % , im Math-Mode aktiv ("8000) machen
{\catcode`\,=\active \gdef
,{\obeyspaces\futurelet\next\CommaCheck}}
\def\CommaCheck{\if\space\next\CommaPunct\else\CommaOrdinary\fi}

%%%%%%%%%%%%%%%%%%%%%%%%%%%%%%%%%%%%%%%%%%%%%%%%%%%%%%
%%%%%%%%%%%%%% PAGE SETTINGS %%%%%%%%%%%%%%%%%%%%%%%%%
%%%%%%%%%%%%%%%%%%%%%%%%%%%%%%%%%%%%%%%%%%%%%%%%%%%%%%

\setlength{\parindent}{0em}
\topmargin 0cm
\oddsidemargin 0cm
\evensidemargin 0cm

\geometry{%
  left=45.0mm,
  right=15.0mm,
  top=25mm,
  bottom=25mm,
  bindingoffset=0mm,
  headheight=30pt,
  includehead
}

\fancyheadoffset[L]{20mm}
\renewcommand{\headrulewidth}{1pt}

\pagestyle{fancy}

%%%%%%%%%%%%%%%%%%%%%%%%%%%%%%%%%%%%%%%%%%%%%%%%%%%%%%
%%%%%%%%%%%%%% PDF SETTINGS %%%%%%%%%%%%%%%%%%%%%%%%%%
%%%%%%%%%%%%%%%%%%%%%%%%%%%%%%%%%%%%%%%%%%%%%%%%%%%%%%

\hypersetup{
    %pdftitle={\Fach{}: Übungsblatt \Uebungsblatt{}},
    pdftitle={\Fach{}: Kurseinheit \Kurseinheit{} Aufgabe \AufgabeNummer{}},
    pdfauthor={\Name},
    pdfborder={0 0 0}
}

%%%%%%%%%%%%%%%%%%%%%%%%%%%%%%%%%%%%%%%%%%%%%%%%%%%%%%
%%%%%%%%%%%%%% CODE SETTINGS %%%%%%%%%%%%%%%%%%%%%%%%%
%%%%%%%%%%%%%%%%%%%%%%%%%%%%%%%%%%%%%%%%%%%%%%%%%%%%%%

\lstset{ %
language=java,
basicstyle=\footnotesize\tt,
showtabs=false,
tabsize=2,
captionpos=b,
breaklines=true,
extendedchars=true,
showstringspaces=false,
flexiblecolumns=true,
}

%%%%%%%%%%%%%%%%%%%%%%%%%%%%%%%%%%%%%%%%%%%%%%%%%%%%%%
%%%%%%%%%%%%%% DOCUMENT %%%%%%%%%%%%%%%%%%%%%%%%%%%%%%
%%%%%%%%%%%%%%%%%%%%%%%%%%%%%%%%%%%%%%%%%%%%%%%%%%%%%%

\title{Kurseinheit \Kurseinheit{} Aufgabe \AufgabeNummer{}}
\author{\Name{}}

\begin{document}

%\renewcommand{\theequation}{L\Kurseinheit{}.\AufgabeNummer{}.\arabic{equation}}
\renewcommand{\theequation}{L \AufgabeNummer{}.\arabic{equation}}

%%%%%%%%%%%%%%%%%%%%%%%%%%%%%%%%%%%%%%%%%%%%%%%%%%%%%%
%%%%%%%%%%%%%% HEADER %%%%%%%%%%%%%%%%%%%%%%%%%%%%%%%%
%%%%%%%%%%%%%%%%%%%%%%%%%%%%%%%%%%%%%%%%%%%%%%%%%%%%%%

\lhead{\sf \large \Fach{} \\ \small \Name{} - \Matrikelnummer{}}
%\rhead{\sf \Semester{}}
\rhead{\sf \FachKurz{} \quad E \Kurseinheit{}/\AufgabeNummer{}}

%%%%%%%%%%%%%%%%%%%%%%%%%%%%%%%%%%%%%%%%%%%%%%%%%%%%%%
%%%%%%%%%%%%%% START HERE %%%%%%%%%%%%%%%%%%%%%%%%%%%%
%%%%%%%%%%%%%%%%%%%%%%%%%%%%%%%%%%%%%%%%%%%%%%%%%%%%%%

\Aufgabe{\AufgabeNummer{}}

Wir betrachten die dreigliedrige Rekursion nach Definition 5.2.1:
$$T_{n+1} = 2\mu_1T_n-T_{n-1}, \; n=1, 2, \dots, \; T_0 = \mu_0, \; T_1 = \mu_1.$$
Für $$\tilde{T_n}(x) = \frac{(x+\sqrt{x^2-1})^n + (x-\sqrt{x^2-1})^n}{2}$$ ist
$$\tilde{T_0}(x) = \frac{1 + 1}{2} = 1 = \mu_0, \; \tilde{T_1}(x) = \frac{x+x}{2} = x = \mu_1 \quad \textrm{und}$$
\begin{align*}
    \tilde{T}_{n+1}(x) + \tilde{T}_{n-1}(x) &= \Big((x+\sqrt{x^2-1})^{n+1} + (x-\sqrt{x^2-1})^{n+1} \\& \quad + (x+\sqrt{x^2-1})^{n-1} + (x-\sqrt{x^2-1})^{n-1}\Big) \frac{1}{2} \\
    &= \Big((x+\sqrt{x^2-1})^2(x+\sqrt{x^2-1})^{n-1} \\& \quad + (x-\sqrt{x^2-1})^2(x-\sqrt{x^2-1})^{n-1} \\ & \quad + (x+\sqrt{x^2-1})^{n-1} + (x-\sqrt{x^2-1})^{n-1}\Big)\frac{1}{2}  \\
    &= \Big(2x(x+\sqrt{x^2-1})(x+\sqrt{x^2-1})^{n-1} \\& \quad + 2x(x-\sqrt{x^2-1})(x-\sqrt{x^2-1})^{n-1} \Big)\frac{1}{2} \\
    &= 2x \frac{(x+\sqrt{x^2-1})^{n} + (x-\sqrt{x^2-1})^{n}}{2} \\
    &= 2\mu_1\tilde{T_n}(x)\textrm{, also} \\
    \tilde{T}_{n+1}(x) &= 2\mu_1\tilde{T_n}(x)- \tilde{T}_{n-1}(x).
\end{align*}
Mit den selben Anfangswerten $\mu_0$ und $\mu_1$ genügen $T_n$ und $\tilde{T_n}$ der selben Rekursion: Sie sind identisch.

\vspace{1em}

Analog betrachten wir nun die Rekursion für die \textsc{Tschebyscheff}-Polynome zweiter Art nach Definition 5.2.1:
$$U_{n+1} = 2 \mu_1 U_n - U_{n-1}, \; n=1, 2, \dots, \; U_0 = \mu_0, \; U_1 = 2\mu_1.$$
Für $$\tilde{U_n}(x) = \frac{(x+\sqrt{x^2-1})^{n+1} + (x-\sqrt{x^2-1})^{n+1}}{2\sqrt{x^2-1}}$$ ist
$$\tilde{U_0}(x) = 1 = \mu_0, \; \tilde{U_1}(x) = 2x = 2\mu_1 \quad \textrm{und}$$
\begin{align*}
    \tilde{U}_{n+1}(x) + \tilde{U}_{n-1}(x) &= \Big( (x+\sqrt{x^2-1})^{n+2} + (x-\sqrt{x^2-1})^{n+2} \\& \quad + (x+\sqrt{x^2-1})^{n} + (x-\sqrt{x^2-1})^{n} \Big) (2\sqrt{x^2-1})^{-1} \\
    &= \Big((x+\sqrt{x^2-1})^2(x+\sqrt{x^2-1})^{n} \\& \quad + (x-\sqrt{x^2-1})^2(x-\sqrt{x^2-1})^{n} \\ & \quad + (x+\sqrt{x^2-1})^{n} + (x-\sqrt{x^2-1})^{n}\Big) (2\sqrt{x^2-1})^{-1} \\
    &= \Big(2x(x+\sqrt{x^2-1})(x+\sqrt{x^2-1})^{n} \\& \quad + 2x(x-\sqrt{x^2-1})(x-\sqrt{x^2-1})^{n} \Big) (2\sqrt{x^2-1})^{-1} \\
    &= 2x \frac{(x+\sqrt{x^2-1})^{n+1} + (x-\sqrt{x^2-1})^{n+1}}{2\sqrt{x^2-1}} \\
    &= 2\mu_1\tilde{U_n}(x)\textrm{, also} \\
    \tilde{U}_{n+1}(x) &= 2\mu_1\tilde{U_n}(x)- \tilde{U}_{n-1}(x).
\end{align*}
Mit den selben Anfangswerten $\mu_0$ und $\mu_1$ genügen $U_n$ und $\tilde{U_n}$ der selben Rekursion: Sie sind identisch.
\end{document}
