\documentclass[a4paper,12pt]{article}
\usepackage{fancyhdr}
\usepackage[ngerman,german]{babel}
\usepackage{german}
\usepackage[utf8]{inputenc}
\usepackage[active]{srcltx}
\usepackage{algorithm}
\usepackage[noend]{algorithmic}
\usepackage{amsmath}
\usepackage{amssymb}
\usepackage{amsthm}
\usepackage{bbm}
\usepackage{enumerate}
\usepackage{graphicx}
\usepackage{ifthen}
\usepackage{listings}
\usepackage{struktex}
\usepackage{hyperref}
\usepackage[onehalfspacing]{setspace}
\usepackage{geometry}
\usepackage{calc}

%%%%%%%%%%%%%%%%%%%%%%%%%%%%%%%%%%%%%%%%%%%%%%%%%%%%%%
%%%%%%%%%%%%%% EDIT THIS PART %%%%%%%%%%%%%%%%%%%%%%%%
%%%%%%%%%%%%%%%%%%%%%%%%%%%%%%%%%%%%%%%%%%%%%%%%%%%%%%

\newcommand{\Fach}{01270 Numerische Mathematik I}
\newcommand{\FachKurz}{NMI}
\newcommand{\Name}{Oliver Schwarz}
\newcommand{\Matrikelnummer}{6883389}
\newcommand{\Semester}{SoSe 23}
\newcommand{\Kurseinheit}{2}
\newcommand{\AufgabeNummer}{3}

%%%%%%%%%%%%%%%%%%%%%%%%%%%%%%%%%%%%%%%%%%%%%%%%%%%%%%
%%%%%%%%%%%%%% DO NOT EDIT THIS PART %%%%%%%%%%%%%%%%%
%%%%%%%%%%%%%%%%%%%%%%%%%%%%%%%%%%%%%%%%%%%%%%%%%%%%%%

\newcommand{\Aufgabe}[1]{
  {
  \vspace*{0.5cm}
  \textbf{Aufgabe #1}
  \vspace*{0.2cm}
  }
}

%%%%%%%%%%%%%%%%%%%%%%%%%%%%%%%%%%%%%%%%%%%%%%%%%%%%%%
%%%%%%%%%%%%%% PAGE SETTINGS %%%%%%%%%%%%%%%%%%%%%%%%%
%%%%%%%%%%%%%%%%%%%%%%%%%%%%%%%%%%%%%%%%%%%%%%%%%%%%%%

\setlength{\parindent}{0em}
\topmargin 0cm
\oddsidemargin 0cm
\evensidemargin 0cm

\geometry{%
  left=45.0mm,
  right=15.0mm,
  top=25mm,
  bottom=25mm,
  bindingoffset=0mm,
  headheight=30pt,
  includehead
}

\fancyheadoffset[L]{20mm}
\renewcommand{\headrulewidth}{1pt}

\pagestyle{fancy}

%%%%%%%%%%%%%%%%%%%%%%%%%%%%%%%%%%%%%%%%%%%%%%%%%%%%%%
%%%%%%%%%%%%%% PDF SETTINGS %%%%%%%%%%%%%%%%%%%%%%%%%%
%%%%%%%%%%%%%%%%%%%%%%%%%%%%%%%%%%%%%%%%%%%%%%%%%%%%%%

\hypersetup{
    %pdftitle={\Fach{}: Übungsblatt \Uebungsblatt{}},
    pdftitle={\Fach{}: Kurseinheit \Kurseinheit{} Aufgabe \AufgabeNummer{}},
    pdfauthor={\Name},
    pdfborder={0 0 0}
}

%%%%%%%%%%%%%%%%%%%%%%%%%%%%%%%%%%%%%%%%%%%%%%%%%%%%%%
%%%%%%%%%%%%%% CODE SETTINGS %%%%%%%%%%%%%%%%%%%%%%%%%
%%%%%%%%%%%%%%%%%%%%%%%%%%%%%%%%%%%%%%%%%%%%%%%%%%%%%%

\lstset{ %
language=java,
basicstyle=\footnotesize\tt,
showtabs=false,
tabsize=2,
captionpos=b,
breaklines=true,
extendedchars=true,
showstringspaces=false,
flexiblecolumns=true,
}

%%%%%%%%%%%%%%%%%%%%%%%%%%%%%%%%%%%%%%%%%%%%%%%%%%%%%%
%%%%%%%%%%%%%% DOCUMENT %%%%%%%%%%%%%%%%%%%%%%%%%%%%%%
%%%%%%%%%%%%%%%%%%%%%%%%%%%%%%%%%%%%%%%%%%%%%%%%%%%%%%

\title{Kurseinheit \Kurseinheit{} Aufgabe \AufgabeNummer{}}
\author{\Name{}}

\begin{document}

%\renewcommand{\theequation}{L\Kurseinheit{}.\AufgabeNummer{}.\arabic{equation}}
\renewcommand{\theequation}{L \AufgabeNummer{}.\arabic{equation}}

%%%%%%%%%%%%%%%%%%%%%%%%%%%%%%%%%%%%%%%%%%%%%%%%%%%%%%
%%%%%%%%%%%%%% HEADER %%%%%%%%%%%%%%%%%%%%%%%%%%%%%%%%
%%%%%%%%%%%%%%%%%%%%%%%%%%%%%%%%%%%%%%%%%%%%%%%%%%%%%%

\lhead{\sf \large \Fach{} \\ \small \Name{} - \Matrikelnummer{}}
%\rhead{\sf \Semester{}}
\rhead{\sf \FachKurz{} \quad E \Kurseinheit{}/\AufgabeNummer{}}

%%%%%%%%%%%%%%%%%%%%%%%%%%%%%%%%%%%%%%%%%%%%%%%%%%%%%%
%%%%%%%%%%%%%% START HERE %%%%%%%%%%%%%%%%%%%%%%%%%%%%
%%%%%%%%%%%%%%%%%%%%%%%%%%%%%%%%%%%%%%%%%%%%%%%%%%%%%%

\Aufgabe{\AufgabeNummer{}}

\begin{enumerate}[a)]
\item %a
Wir betrachten für den Vektorraum $\mathbb{R}^n$ die Abschätzung
\begin{equation*}
    {||\pmb{x}||}_2 \leq {||\pmb{x}||}_1 \leq  \sqrt{n} \; {||\pmb{x}||}_2.
\end{equation*}
Für den linken Teil der Ungleichung erhalten wir (mit $\{|a|^2 + |b|^2\} \leq \{|a| + |b|\}^2$)
\begin{align*}
{||\pmb{x}||}_2^2 &= \sum\limits_{i=1}^{n} |x_i|^2 \leq \bigg(\sum\limits_{i=1}^{n} |x_i|\bigg)^2 = {||\pmb{x}||}_1^2 \\
\Longrightarrow {||\pmb{x}||}_2 &\leq {||\pmb{x}||}_1
\end{align*}
Gleichheit erhalten wir für alle Vektoren $\pmb{x} \in \mathbb{R}^n$ mit einem $|x_j| \neq 0$ und allen anderen $|x_i| = 0$, $j \in [1, \dots , n], \, i \in [1, \dots, n] \setminus j$.


%Mit der \textsc{Cauchy-Schwarzschen} Ungleichung $|\langle \pmb{x}, \pmb{y} \rangle| \leq ||\pmb{x}|| \ ||\pmb{y}||$ und dem Skalarprodukt $|\langle \pmb{x} , \pmb{1} \rangle| = \sum_{i=1}^{n} 1 \cdot |x_i|$
Mit der \textsc{Cauchy-Schwarz}schen Ungleichung $|\langle \pmb{x}, \pmb{y} \rangle| \leq {\langle \pmb{x}, \pmb{x} \rangle}^{\frac{1}{2}} \ {\langle \pmb{y}, \pmb{y} \rangle}^{\frac{1}{2}}$ und dem Skalarprodukt $|\langle \pmb{x} , \pmb{1} \rangle| = \sum_{i=1}^{n} 1 \cdot |x_i|$ erhalten wir aus der rechten Seite der Ungleichung
\begin{align*}
    {||\pmb{x}||}_1 &= \sum\limits_{i=1}^{n} |x_i| = \sum\limits_{i=1}^{n} 1 \cdot |x_i| = |\langle \pmb{x} , \pmb{1} \rangle| \\
    &\leq {\langle \pmb{x}, \pmb{x} \rangle}^{\frac{1}{2}} \ {\langle \pmb{1}, \pmb{1} \rangle}^{\frac{1}{2}} = \bigg( \sum\limits_{i=1}^{n} |x_i|^2 \bigg)^{\frac{1}{2}} \, \bigg( \sum\limits_{i=1}^{n} 1 \bigg)^{\frac{1}{2}} = {||\pmb{x}||}_2 \sqrt{n}
\end{align*}
Gleichheit besteht für alle Vektoren aus $\mathbb{R}^n$ mit $|x_1| = |x_2| = \cdots = |x_n|$.

\item %b
Wir betrachten für den Vektorraum $\mathbb{R}^n$ die Abschätzung
\begin{equation*}
    {||\pmb{x}||}_{\infty} \leq {||\pmb{x}||}_2 \leq  \sqrt{n} \; {||\pmb{x}||}_{\infty }.
\end{equation*}
Für die Maximumnorm gilt ${||\pmb{x}||}_{\infty} := \max\limits_{i = 1, \dots, n} |x_i|$. Also gibt es ein $k \in \{ 1, \dots , n \}$ mit $|x_k| = \max\limits_{i = 1, \dots, n} |x_i| = {||\pmb{x}||}_{\infty}$. Mit diesem $k$ folgt aus dem linken Teil der Abschätzung:
\begin{equation*}
    {||\pmb{x}||}_{\infty}^2 = |x_k|^2 \leq \sum\limits_{i=1}^{n} |x_i|^2 = {||\pmb{x}||}_{2}^2 \\
    \quad \Longrightarrow \quad {||\pmb{x}||}_{\infty} \leq {||\pmb{x}||}_{2}
\end{equation*}
Gleichheit besteht für alle Vektoren aus $\mathbb{R}^n$, wenn $|x_i| = 0$ und $|x_k| > 0$ für alle $i \in \{ 1, \dots , n \}$, ein $k \in \{ 1, \dots , n \}$ und $i \neq k$.

Mit der \textsc{Cauchy-Schwarz}schen Ungleichung und der Maximumnorm erhalten wir für die rechte Seite der Abschätzung:
\begin{align*}
    {||\pmb{x}||}_2^2 &= \sum\limits_{i=1}^{n} |x_i|^2 \leq \sum\limits_{i=1}^{n} ||\pmb{x}||_{\infty} |x_i| \\
    &\leq \bigg( \sum\limits_{i=1}^{n} ||\pmb{x}||_{\infty}^2 \bigg)^{\frac{1}{2}} \, \bigg( \sum\limits_{i=1}^{n} |x_i|^2 \bigg)^{\frac{1}{2}} = \sqrt{n} \, ||\pmb{x}||_{\infty} \, ||\pmb{x}||_2 \\
    &\Longrightarrow ||\pmb{x}||_2 \leq ||\pmb{x}||_{\infty}
\end{align*}
Gleichheit besteht für alle Vektoren aus $\mathbb{R}^n$ mit $|x_1| = |x_2| = \cdots = |x_n|$.

\end{enumerate}



\end{document}
