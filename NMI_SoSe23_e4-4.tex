\documentclass[a4paper,12pt]{article}
\usepackage{fancyhdr}
\usepackage[ngerman,german]{babel}
\usepackage{german}
\usepackage[utf8]{inputenc}
\usepackage[active]{srcltx}
\usepackage{algorithm}
\usepackage[noend]{algorithmic}
\usepackage{amsmath}
\usepackage{amssymb}
\usepackage{amsthm}
\usepackage{bbm}
\usepackage{enumerate}
\usepackage{graphicx}
\usepackage{ifthen}
\usepackage{listings}
\usepackage{struktex}
\usepackage{hyperref}
\usepackage[onehalfspacing]{setspace}
\usepackage{geometry}
\usepackage{calc}

%%%%%%%%%%%%%%%%%%%%%%%%%%%%%%%%%%%%%%%%%%%%%%%%%%%%%%
%%%%%%%%%%%%%% EDIT THIS PART %%%%%%%%%%%%%%%%%%%%%%%%
%%%%%%%%%%%%%%%%%%%%%%%%%%%%%%%%%%%%%%%%%%%%%%%%%%%%%%

\newcommand{\Fach}{01270 Numerische Mathematik I}
\newcommand{\FachKurz}{NMI}
\newcommand{\Name}{Oliver Schwarz}
\newcommand{\Matrikelnummer}{6883389}
\newcommand{\Semester}{SoSe 23}
\newcommand{\Kurseinheit}{4}
\newcommand{\AufgabeNummer}{4}

%%%%%%%%%%%%%%%%%%%%%%%%%%%%%%%%%%%%%%%%%%%%%%%%%%%%%%
%%%%%%%%%%%%%% DO NOT EDIT THIS PART %%%%%%%%%%%%%%%%%
%%%%%%%%%%%%%%%%%%%%%%%%%%%%%%%%%%%%%%%%%%%%%%%%%%%%%%

\newcommand{\Aufgabe}[1]{
  {
  \vspace*{0.5cm}
  \textbf{Aufgabe #1}
  \vspace*{0.2cm}
  }
}

\newcommand{\cond}[0]{
  {
  \textrm{cond}
  }
}

% Beschreibung
% Unterscheidet im Mathe-Mode zwischen
% "Dezimal-komma" und Satzzeichen.
% Muss bei Eingabe durch kein/ein Leerzeichen
% dahinter angegeben werden.
% Literatur Richard Hirsch in: DTK, 1/1994, S. 42 ff.
\mathchardef\CommaOrdinary="013B \mathchardef\CommaPunct="613B
\mathcode`,="8000 % , im Math-Mode aktiv ("8000) machen
{\catcode`\,=\active \gdef
,{\obeyspaces\futurelet\next\CommaCheck}}
\def\CommaCheck{\if\space\next\CommaPunct\else\CommaOrdinary\fi}

%%%%%%%%%%%%%%%%%%%%%%%%%%%%%%%%%%%%%%%%%%%%%%%%%%%%%%
%%%%%%%%%%%%%% PAGE SETTINGS %%%%%%%%%%%%%%%%%%%%%%%%%
%%%%%%%%%%%%%%%%%%%%%%%%%%%%%%%%%%%%%%%%%%%%%%%%%%%%%%

\setlength{\parindent}{0em}
\topmargin 0cm
\oddsidemargin 0cm
\evensidemargin 0cm

\geometry{%
  left=45.0mm,
  right=15.0mm,
  top=25mm,
  bottom=25mm,
  bindingoffset=0mm,
  headheight=30pt,
  includehead
}

\fancyheadoffset[L]{20mm}
\renewcommand{\headrulewidth}{1pt}

\pagestyle{fancy}

%%%%%%%%%%%%%%%%%%%%%%%%%%%%%%%%%%%%%%%%%%%%%%%%%%%%%%
%%%%%%%%%%%%%% PDF SETTINGS %%%%%%%%%%%%%%%%%%%%%%%%%%
%%%%%%%%%%%%%%%%%%%%%%%%%%%%%%%%%%%%%%%%%%%%%%%%%%%%%%

\hypersetup{
    %pdftitle={\Fach{}: Übungsblatt \Uebungsblatt{}},
    pdftitle={\Fach{}: Kurseinheit \Kurseinheit{} Aufgabe \AufgabeNummer{}},
    pdfauthor={\Name},
    pdfborder={0 0 0}
}

%%%%%%%%%%%%%%%%%%%%%%%%%%%%%%%%%%%%%%%%%%%%%%%%%%%%%%
%%%%%%%%%%%%%% CODE SETTINGS %%%%%%%%%%%%%%%%%%%%%%%%%
%%%%%%%%%%%%%%%%%%%%%%%%%%%%%%%%%%%%%%%%%%%%%%%%%%%%%%

\lstset{ %
language=java,
basicstyle=\footnotesize\tt,
showtabs=false,
tabsize=2,
captionpos=b,
breaklines=true,
extendedchars=true,
showstringspaces=false,
flexiblecolumns=true,
}

%%%%%%%%%%%%%%%%%%%%%%%%%%%%%%%%%%%%%%%%%%%%%%%%%%%%%%
%%%%%%%%%%%%%% DOCUMENT %%%%%%%%%%%%%%%%%%%%%%%%%%%%%%
%%%%%%%%%%%%%%%%%%%%%%%%%%%%%%%%%%%%%%%%%%%%%%%%%%%%%%

\title{Kurseinheit \Kurseinheit{} Aufgabe \AufgabeNummer{}}
\author{\Name{}}

\begin{document}

%\renewcommand{\theequation}{L\Kurseinheit{}.\AufgabeNummer{}.\arabic{equation}}
\renewcommand{\theequation}{L \AufgabeNummer{}.\arabic{equation}}

%%%%%%%%%%%%%%%%%%%%%%%%%%%%%%%%%%%%%%%%%%%%%%%%%%%%%%
%%%%%%%%%%%%%% HEADER %%%%%%%%%%%%%%%%%%%%%%%%%%%%%%%%
%%%%%%%%%%%%%%%%%%%%%%%%%%%%%%%%%%%%%%%%%%%%%%%%%%%%%%

\lhead{\sf \large \Fach{} \\ \small \Name{} - \Matrikelnummer{}}
%\rhead{\sf \Semester{}}
\rhead{\sf \FachKurz{} \quad E \Kurseinheit{}/\AufgabeNummer{}}

%%%%%%%%%%%%%%%%%%%%%%%%%%%%%%%%%%%%%%%%%%%%%%%%%%%%%%
%%%%%%%%%%%%%% START HERE %%%%%%%%%%%%%%%%%%%%%%%%%%%%
%%%%%%%%%%%%%%%%%%%%%%%%%%%%%%%%%%%%%%%%%%%%%%%%%%%%%%

\Aufgabe{\AufgabeNummer{}}
\begin{enumerate}[a)]
    \item %a
Wir stellen das \textsc{Horner}-Schema für $p = 2x^4-7x^4+12x^3-17x^2+3x+15$ auf:
\begin{align*}
\begin{array}[t]{ r | r r r r r r l}
   &   2 &  -7 &  12 & -17 &   3 &  15 \\
 2 &   - &   4 &  -6 &  12 & -10 &  14 \\\cline{1-7}
   &   2 &  -3 &   6 &  -5 &  -7 &   \boxed{1} & \quad = \displaystyle\frac{p(2)}{0!}\\
 2 &   - &   4 &   2 &  16 &  22 &     \\\cline{1-6}
   &   2 &   1 &   8 &  11 &  \boxed{15} &   \multicolumn{2}{l}{\quad =\displaystyle\frac{p'(2)}{1!}}  \\
 2 &   - &   4 &  10 &  36 &     &     \\\cline{1-5}
   &   2 &   5 &  18 &  \boxed{47} & \multicolumn{3}{l}{\quad =\displaystyle\frac{p''(2)}{2!}}\\
 2 &   - &   4 &  18 &     &     &     \\\cline{1-4}
   &   2 &   9 &  \boxed{36} & \multicolumn{4}{l}{\quad =\displaystyle\frac{p^{(3)}(2)}{3!}}     \\
 2 &   - &   4 &     &     &     &     \\\cline{1-3}
   &   2 &  \boxed{13} & \multicolumn{5}{l}{\quad =\displaystyle\frac{p^{(4)}(2)}{4!}}     \\
 2 &   - &     &     &     &     &     \\\cline{1-2}
   &   \boxed{2} &  \multicolumn{6}{l}{\quad =\displaystyle\frac{p^{(5)}(2)}{5!}}    
 \end{array}
\end{align*}
Wir erhalten $p(2)=1$, $p'(2) = 15$, $p''(2) = 94$, $p^{(3)}(2) = 216$, $p^{(4)}(2) = 312$, $p^{(5)}(2) = 240$ und $p^{(6)}(2) = 0$.

Die \textsc{Taylor}-Entwicklung von $p$ an der Stelle $x = 2$ ist:
$$p(x) = 1 + 15(x-2) + 47(x-2)^2 + 36(x-2)^3 + 13(x-2)^4 + 2(x-2)^5.$$

\item %b)
Wir bestimmen die Monom-Darstellungen der \textsc{Tschebyscheff}-Polynome erster Art durch die dreigliedrige Rekursion (Definition 5.2.1):
\begin{align*}
    T_0(x) &= 1,\\
    T_1(x) &= x,\\
    T_2(x) &= 2x^2-1,\\
    T_3(x) &= 4x^3-3x,\\
    T_4(x) &= 8x^4-8x^2+1.
\end{align*}
Wir nehmen für die \textsc{Tschebyscheff}-Darstellung die Form: $$q(x) = \sum\limits_{i=0}^{4} \alpha_iT_i(x).$$
Nach Einsetzen der obigen \textsc{Tschebyscheff}-Polynome erster Art erhalten wir:
\begin{align*}
    q(x) &= 48x^4-16x^3-36x^2+10x \\
    &= (\alpha_0-\alpha_2+\alpha_4)+(\alpha_1-3\alpha_3)x+(2\alpha_2-8\alpha_4)x^2+(4\alpha_3)x^3+(8\alpha_4)x^4.
\end{align*}
Wir müssen also folgendes Gleichungssystem lösen:
\begin{align*}
    \begin{pmatrix}
1 & 0 & -1 & 0 & 1 \\
0 & 1 & 0 & -3 & 0 \\
0 & 0 & 2 & 0 & -8 \\
0 & 0 & 0 & 4 & 0 \\
0 & 0 & 0 & 0 & 8 \\
    \end{pmatrix}
    \begin{pmatrix}
\alpha_0 \\
\alpha_1 \\
\alpha_2 \\
\alpha_3 \\
\alpha_4 
    \end{pmatrix} =
    \begin{pmatrix}
0 \\
10 \\
-36 \\
-16 \\
48
    \end{pmatrix}
\end{align*}
Wir erhalten $\alpha_4 = 6$, $\alpha_3 = -4$, $\alpha_2 = 6$, $\alpha_1 = -2$ und $\alpha_0 = 0$. Also ist
$$q(x) = 6T_4(x) - 4T_3(x) +6T_2(x)-2T_1(x).$$
Wir werten $q(x)$ nun mit dem \textsc{Clenshaw}-Algorithmus an der Stelle $x=1$ aus:
$$
\def\arraystretch{1.3}
\begin{array}[t]{ r | r r r r r r}
  & 6 & -4 &  6 & -2 &   0 \\
  & - & -  & -6 & -8 & -16 \\
1 & - & 12 & 16 & 32 &  22 \\\cline{1-6}
  & 6 &  8 & 16 & 22 &  \multicolumn{2}{r}{\quad \boxed{6 =q(1)}}
\end{array}
$$

Nun betrachten wir die \textsc{Tschebyscheff}-Polynome zweiter Art über die Rekursion  (Definition 5.2.1):
\begin{align*}
    U_0(x) &= 1,\\
    U_1(x) &= 2x,\\
    U_2(x) &= 4x^2-1,\\
    U_3(x) &= 8^3-4x,\\
    U_4(x) &= 16x^4-12x^2+1.
\end{align*}
Wir nehmen für die \textsc{Tschebyscheff}-Darstellung die Form: $$q(x) = \sum\limits_{i=0}^{4} \nu_iU_i(x).$$
Nach Einsetzen der \textsc{Tschebyscheff}-Polynome zweiter Art erhalten wir:
\begin{align*}
    q(x) &= 48x^4-16x^3-36x^2+10x \\
    &= (\nu_0-\nu_2+\nu_4)+(2\nu_1-4\nu_3)x+(4\nu_2-12\nu_4)x^2+(8\nu_3)x^3+(16\nu_4)x^4.
\end{align*}
Wir müssen also folgendes Gleichungssystem lösen:
\begin{align*}
    \begin{pmatrix}
1 & 0 & -1 & 0 & 0 \\
0 & 2 & 0 & -4 & 0 \\
0 & 0 & 4 & 0 & -12 \\
0 & 0 & 0 & 8 & 0 \\
0 & 0 & 0 & 0 & 16 \\
    \end{pmatrix}
    \begin{pmatrix}
\nu_0 \\
\nu_1 \\
\nu_2 \\
\nu_3 \\
\nu_4 
    \end{pmatrix} =
    \begin{pmatrix}
0 \\
10 \\
-36 \\
-16 \\
48
    \end{pmatrix}
\end{align*}
Wir erhalten $\nu_4 = 3$, $\nu_3 = -2$, $\nu_2 = 0$, $\nu_1 = 1$ und $\nu_0 = 3$. Also ist
$$q(x) = 4U_4(x) - 2U_3(x) +U_1(x)+3U_0(x).$$
Wir werten $q(x)$ nun mit dem \textsc{Clenshaw}-Algorithmus an der Stelle $x=1$ aus:
$$
\def\arraystretch{1.3}
\begin{array}[t]{ r | r r r r r r}
  & 3 & -2 &  0 &  1 &   3 \\
  & - & -  & -3 & -4 &  -5 \\
1 & - &  6 &  8 & 10 &  14 \\\cline{1-6}
  & 3 &  4 &  5 &  7 &  \multicolumn{2}{r}{\quad \boxed{6 =q(1)}}
\end{array}
$$

\end{enumerate}


\end{document}

