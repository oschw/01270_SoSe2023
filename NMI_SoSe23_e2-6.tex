\documentclass[a4paper,12pt]{article}
\usepackage{fancyhdr}
\usepackage[ngerman,german]{babel}
\usepackage{german}
\usepackage[utf8]{inputenc}
\usepackage[active]{srcltx}
\usepackage{algorithm}
\usepackage[noend]{algorithmic}
\usepackage{amsmath}
\usepackage{amssymb}
\usepackage{amsthm}
\usepackage{bbm}
\usepackage{enumerate}
\usepackage{graphicx}
\usepackage{ifthen}
\usepackage{listings}
\usepackage{struktex}
\usepackage{hyperref}
\usepackage[onehalfspacing]{setspace}
\usepackage{geometry}
\usepackage{calc}

%%%%%%%%%%%%%%%%%%%%%%%%%%%%%%%%%%%%%%%%%%%%%%%%%%%%%%
%%%%%%%%%%%%%% EDIT THIS PART %%%%%%%%%%%%%%%%%%%%%%%%
%%%%%%%%%%%%%%%%%%%%%%%%%%%%%%%%%%%%%%%%%%%%%%%%%%%%%%

\newcommand{\Fach}{01270 Numerische Mathematik I}
\newcommand{\FachKurz}{NMI}
\newcommand{\Name}{Oliver Schwarz}
\newcommand{\Matrikelnummer}{6883389}
\newcommand{\Semester}{SoSe 23}
\newcommand{\Kurseinheit}{2}
\newcommand{\AufgabeNummer}{6}

%%%%%%%%%%%%%%%%%%%%%%%%%%%%%%%%%%%%%%%%%%%%%%%%%%%%%%
%%%%%%%%%%%%%% DO NOT EDIT THIS PART %%%%%%%%%%%%%%%%%
%%%%%%%%%%%%%%%%%%%%%%%%%%%%%%%%%%%%%%%%%%%%%%%%%%%%%%

\newcommand{\Aufgabe}[1]{
  {
  \vspace*{0.5cm}
  \textbf{Aufgabe #1}
  \vspace*{0.2cm}
  }
}

%%%%%%%%%%%%%%%%%%%%%%%%%%%%%%%%%%%%%%%%%%%%%%%%%%%%%%
%%%%%%%%%%%%%% PAGE SETTINGS %%%%%%%%%%%%%%%%%%%%%%%%%
%%%%%%%%%%%%%%%%%%%%%%%%%%%%%%%%%%%%%%%%%%%%%%%%%%%%%%

\setlength{\parindent}{0em}
\topmargin 0cm
\oddsidemargin 0cm
\evensidemargin 0cm

\geometry{%
  left=45.0mm,
  right=15.0mm,
  top=25mm,
  bottom=25mm,
  bindingoffset=0mm,
  headheight=30pt,
  includehead
}

\fancyheadoffset[L]{20mm}
\renewcommand{\headrulewidth}{1pt}

\pagestyle{fancy}

%%%%%%%%%%%%%%%%%%%%%%%%%%%%%%%%%%%%%%%%%%%%%%%%%%%%%%
%%%%%%%%%%%%%% PDF SETTINGS %%%%%%%%%%%%%%%%%%%%%%%%%%
%%%%%%%%%%%%%%%%%%%%%%%%%%%%%%%%%%%%%%%%%%%%%%%%%%%%%%

\hypersetup{
    %pdftitle={\Fach{}: Übungsblatt \Uebungsblatt{}},
    pdftitle={\Fach{}: Kurseinheit \Kurseinheit{} Aufgabe \AufgabeNummer{}},
    pdfauthor={\Name},
    pdfborder={0 0 0}
}

%%%%%%%%%%%%%%%%%%%%%%%%%%%%%%%%%%%%%%%%%%%%%%%%%%%%%%
%%%%%%%%%%%%%% CODE SETTINGS %%%%%%%%%%%%%%%%%%%%%%%%%
%%%%%%%%%%%%%%%%%%%%%%%%%%%%%%%%%%%%%%%%%%%%%%%%%%%%%%

\lstset{ %
language=java,
basicstyle=\footnotesize\tt,
showtabs=false,
tabsize=2,
captionpos=b,
breaklines=true,
extendedchars=true,
showstringspaces=false,
flexiblecolumns=true,
}

%%%%%%%%%%%%%%%%%%%%%%%%%%%%%%%%%%%%%%%%%%%%%%%%%%%%%%
%%%%%%%%%%%%%% DOCUMENT %%%%%%%%%%%%%%%%%%%%%%%%%%%%%%
%%%%%%%%%%%%%%%%%%%%%%%%%%%%%%%%%%%%%%%%%%%%%%%%%%%%%%

\title{Kurseinheit \Kurseinheit{} Aufgabe \AufgabeNummer{}}
\author{\Name{}}

\begin{document}

%\renewcommand{\theequation}{L\Kurseinheit{}.\AufgabeNummer{}.\arabic{equation}}
\renewcommand{\theequation}{L \AufgabeNummer{}.\arabic{equation}}

%%%%%%%%%%%%%%%%%%%%%%%%%%%%%%%%%%%%%%%%%%%%%%%%%%%%%%
%%%%%%%%%%%%%% HEADER %%%%%%%%%%%%%%%%%%%%%%%%%%%%%%%%
%%%%%%%%%%%%%%%%%%%%%%%%%%%%%%%%%%%%%%%%%%%%%%%%%%%%%%

\lhead{\sf \large \Fach{} \\ \small \Name{} - \Matrikelnummer{}}
%\rhead{\sf \Semester{}}
\rhead{\sf \FachKurz{} \quad E \Kurseinheit{}/\AufgabeNummer{}}

%%%%%%%%%%%%%%%%%%%%%%%%%%%%%%%%%%%%%%%%%%%%%%%%%%%%%%
%%%%%%%%%%%%%% START HERE %%%%%%%%%%%%%%%%%%%%%%%%%%%%
%%%%%%%%%%%%%%%%%%%%%%%%%%%%%%%%%%%%%%%%%%%%%%%%%%%%%%

\Aufgabe{\AufgabeNummer{}}

Zu zeigen ist, dass für jede Matrix $\pmb{A} \in \mathbb{C}^{n,n}$
\begin{equation*}
||\pmb{A}||_2 := \varrho (\pmb{A}^H \pmb{A})^{\frac{1}{2}} = \max\limits_{\pmb{x} \in \mathbb{C}^{n}: ||\pmb{x}|| = 1} ||\pmb{Ax}||_2.
\end{equation*}

Wobei 
\begin{equation*}
\varrho (\pmb{A}^H \pmb{A})^{\frac{1}{2}} = \max\{\sqrt{\lambda} \textrm{: } \lambda \textrm{ ist Eigenwert von } \pmb{A}^H\pmb{A} \}.
\end{equation*}

Dies ist genau dann der Fall, wenn
\begin{align*}
   &\, &\textrm{(i)} &\quad ||\pmb{Ax}||_2 \leq ||\pmb{A}||_2 \textrm{ für alle } \pmb{x} \in \mathbb{C}^{n} \textrm{ mit } ||\pmb{x}||_2 = 1 \textrm{, und}\\
   &\, &\textrm{(ii)} &\quad \textrm{es existiert ein } \pmb{x}^* \in \mathbb{C}^{n} \textrm{ mit } ||\pmb{x}^*||_2 = 1 \textrm{ und } ||\pmb{Ax}^*||_2 = ||\pmb{A}||_{\infty}.
\end{align*}

\begin{enumerate}[(i)]
    \item 
Die Eigenwerte von $\pmb{A}^H\pmb{A}$ sind reell und nicht negativ ($\pmb{A}^H\pmb{A}$ ist \textsc{hermite}sch und positiv semidefinit). Es gibt es eine ONB $\{\pmb{x}_1, \dots, \pmb{x}_n\}$ von $\mathbb{C}^n$ aus Eigenvektoren von $\pmb{A}^H\pmb{A}$ mit den zugehörigen Eigenwerten (der Größe nach geordnet) $$\lambda_1 \geq \lambda_2 \geq \dots \geq \lambda_n \geq 0.$$
Sei $\pmb{x} \in \mathbb{C}^{n}$ mit $||\pmb{x}||_2 = 1$ beliebig, dann gibt es $\xi_i \in \mathbb{C}$, $i=1, \dots, n$ und $\pmb{x} = \sum_{i=1}^n \xi_i \pmb{x}_i$:

\begin{align*}
    1 &= ||\pmb{x}||_2^2 = \langle \pmb{x},\pmb{x} \rangle = \pmb{x}^H \pmb{x} = \sum\limits_{i=1}^n \overline{\xi}_i \pmb{x}^H_i  \sum\limits_{j=1}^n \xi_j \pmb{x}_j = \sum\limits_{i,j=1}^n \overline{\xi}_i \xi_j \pmb{x}_i \pmb{x}_j \\
    &= \sum\limits_{i=1}^n |\xi_i|^2
\end{align*}
und
\begin{align}
    ||\pmb{Ax}||_2^2 &= \langle \pmb{Ax},\pmb{Ax} \rangle = \pmb{x}^H \pmb{A}^H \pmb{A} \pmb{x} = \sum\limits_{i=1}^n \overline{\xi}_i \pmb{x}^H_i  \, \pmb{A}^H \pmb{A} \, \sum\limits_{j=1}^n \xi_j \pmb{x}_j \nonumber \\
    &= \sum\limits_{i,j=1}^n \overline{\xi}_i \xi_j \underbrace{\pmb{x}_i \pmb{A}^H \pmb{A} \pmb{x}_j}_{\textrm{Lösung 2.5.16}} = 
    \sum\limits_{i,j=1}^n \overline{\xi}_i \xi_j \pmb{x}_i  \lambda _j \pmb{x}_j
    = \sum\limits_{i=1}^n |\xi_i|^2 \lambda_i \leq \lambda_1 \textrm{, also} \nonumber\\
    %&\leq \lambda_1 = \max\{\sqrt{\lambda} \textrm{: } \lambda \textrm{ ist Eigenwert von } \pmb{A}^H\pmb{A} \} = ||\pmb{A}||_2 \\
    ||\pmb{Ax}||_2 &\leq \sqrt{\lambda_1} = \max\{\sqrt{\lambda} \textrm{: } \lambda \textrm{ ist Eigenwert von } \pmb{A}^H\pmb{A} \} = ||\pmb{A}||_2 \label{eq:formel1}
\end{align}
    
    \item 
Gleichheit in \ref{eq:formel1} herrscht genau dann, wenn $\pmb{x}^*$ der Eigenvektor von $\pmb{A}^H\pmb{A}$ und dem zugehörigen Eigenwert $\lambda_1 = \max\{\lambda \textrm{: } \lambda \textrm{ ist Eigenwert von } \pmb{A}^H\pmb{A} \}$ ist.

\end{enumerate}

\end{document}
