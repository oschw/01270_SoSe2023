\documentclass[a4paper,12pt]{article}
\usepackage{fancyhdr}
\usepackage[ngerman,german]{babel}
\usepackage{german}
\usepackage[utf8]{inputenc}
\usepackage[active]{srcltx}
\usepackage{algorithm}
\usepackage[noend]{algorithmic}
\usepackage{amsmath}
\usepackage{amssymb}
\usepackage{amsthm}
\usepackage{bbm}
\usepackage{enumerate}
\usepackage{graphicx}
\usepackage{ifthen}
\usepackage{listings}
\usepackage{struktex}
\usepackage{hyperref}
\usepackage[onehalfspacing]{setspace}
\usepackage{geometry}
\usepackage{calc}

%%%%%%%%%%%%%%%%%%%%%%%%%%%%%%%%%%%%%%%%%%%%%%%%%%%%%%
%%%%%%%%%%%%%% EDIT THIS PART %%%%%%%%%%%%%%%%%%%%%%%%
%%%%%%%%%%%%%%%%%%%%%%%%%%%%%%%%%%%%%%%%%%%%%%%%%%%%%%

\newcommand{\Fach}{01270 Numerische Mathematik I}
\newcommand{\FachKurz}{NMI}
\newcommand{\Name}{Oliver Schwarz}
\newcommand{\Matrikelnummer}{6883389}
\newcommand{\Semester}{SoSe 23}
\newcommand{\Kurseinheit}{4}
\newcommand{\AufgabeNummer}{1}

%%%%%%%%%%%%%%%%%%%%%%%%%%%%%%%%%%%%%%%%%%%%%%%%%%%%%%
%%%%%%%%%%%%%% DO NOT EDIT THIS PART %%%%%%%%%%%%%%%%%
%%%%%%%%%%%%%%%%%%%%%%%%%%%%%%%%%%%%%%%%%%%%%%%%%%%%%%

\newcommand{\Aufgabe}[1]{
  {
  \vspace*{0.5cm}
  \textbf{Aufgabe #1}
  \vspace*{0.2cm}
  }
}

\newcommand{\cond}[0]{
  {
  \textrm{cond}
  }
}

% Beschreibung
% Unterscheidet im Mathe-Mode zwischen
% "Dezimal-komma" und Satzzeichen.
% Muss bei Eingabe durch kein/ein Leerzeichen
% dahinter angegeben werden.
% Literatur Richard Hirsch in: DTK, 1/1994, S. 42 ff.
\mathchardef\CommaOrdinary="013B \mathchardef\CommaPunct="613B
\mathcode`,="8000 % , im Math-Mode aktiv ("8000) machen
{\catcode`\,=\active \gdef
,{\obeyspaces\futurelet\next\CommaCheck}}
\def\CommaCheck{\if\space\next\CommaPunct\else\CommaOrdinary\fi}

%%%%%%%%%%%%%%%%%%%%%%%%%%%%%%%%%%%%%%%%%%%%%%%%%%%%%%
%%%%%%%%%%%%%% PAGE SETTINGS %%%%%%%%%%%%%%%%%%%%%%%%%
%%%%%%%%%%%%%%%%%%%%%%%%%%%%%%%%%%%%%%%%%%%%%%%%%%%%%%

\setlength{\parindent}{0em}
\topmargin 0cm
\oddsidemargin 0cm
\evensidemargin 0cm

\geometry{%
  left=45.0mm,
  right=15.0mm,
  top=25mm,
  bottom=25mm,
  bindingoffset=0mm,
  headheight=30pt,
  includehead
}

\fancyheadoffset[L]{20mm}
\renewcommand{\headrulewidth}{1pt}

\pagestyle{fancy}

%%%%%%%%%%%%%%%%%%%%%%%%%%%%%%%%%%%%%%%%%%%%%%%%%%%%%%
%%%%%%%%%%%%%% PDF SETTINGS %%%%%%%%%%%%%%%%%%%%%%%%%%
%%%%%%%%%%%%%%%%%%%%%%%%%%%%%%%%%%%%%%%%%%%%%%%%%%%%%%

\hypersetup{
    %pdftitle={\Fach{}: Übungsblatt \Uebungsblatt{}},
    pdftitle={\Fach{}: Kurseinheit \Kurseinheit{} Aufgabe \AufgabeNummer{}},
    pdfauthor={\Name},
    pdfborder={0 0 0}
}

%%%%%%%%%%%%%%%%%%%%%%%%%%%%%%%%%%%%%%%%%%%%%%%%%%%%%%
%%%%%%%%%%%%%% CODE SETTINGS %%%%%%%%%%%%%%%%%%%%%%%%%
%%%%%%%%%%%%%%%%%%%%%%%%%%%%%%%%%%%%%%%%%%%%%%%%%%%%%%

\lstset{ %
language=java,
basicstyle=\footnotesize\tt,
showtabs=false,
tabsize=2,
captionpos=b,
breaklines=true,
extendedchars=true,
showstringspaces=false,
flexiblecolumns=true,
}

%%%%%%%%%%%%%%%%%%%%%%%%%%%%%%%%%%%%%%%%%%%%%%%%%%%%%%
%%%%%%%%%%%%%% DOCUMENT %%%%%%%%%%%%%%%%%%%%%%%%%%%%%%
%%%%%%%%%%%%%%%%%%%%%%%%%%%%%%%%%%%%%%%%%%%%%%%%%%%%%%

\title{Kurseinheit \Kurseinheit{} Aufgabe \AufgabeNummer{}}
\author{\Name{}}

\begin{document}

%\renewcommand{\theequation}{L\Kurseinheit{}.\AufgabeNummer{}.\arabic{equation}}
\renewcommand{\theequation}{L \AufgabeNummer{}.\arabic{equation}}

%%%%%%%%%%%%%%%%%%%%%%%%%%%%%%%%%%%%%%%%%%%%%%%%%%%%%%
%%%%%%%%%%%%%% HEADER %%%%%%%%%%%%%%%%%%%%%%%%%%%%%%%%
%%%%%%%%%%%%%%%%%%%%%%%%%%%%%%%%%%%%%%%%%%%%%%%%%%%%%%

\lhead{\sf \large \Fach{} \\ \small \Name{} - \Matrikelnummer{}}
%\rhead{\sf \Semester{}}
\rhead{\sf \FachKurz{} \quad E \Kurseinheit{}/\AufgabeNummer{}}

%%%%%%%%%%%%%%%%%%%%%%%%%%%%%%%%%%%%%%%%%%%%%%%%%%%%%%
%%%%%%%%%%%%%% START HERE %%%%%%%%%%%%%%%%%%%%%%%%%%%%
%%%%%%%%%%%%%%%%%%%%%%%%%%%%%%%%%%%%%%%%%%%%%%%%%%%%%%

\Aufgabe{\AufgabeNummer{}}

Es sind $\pmb{Q}_1 \pmb{R}_1 = \pmb{Q}_2 \pmb{R}_2 = \pmb{A}$ zwei $QR$-Zerlegungen der Matrix $\pmb{A} \in \mathbb{C}^{n,n}$. $\pmb{Q}_1$, $\pmb{Q}_2$ und $\pmb{R}_1$, $\pmb{R}_2$ sind unitär bzw. regulär, d. h., sie sind invertierbar. Daraus folgt:
\begin{equation*}
    \pmb{Q}_2^{-1}\pmb{Q}_1 = \pmb{R}_2 \pmb{R}_1^{-1} =: \pmb{S}.
\end{equation*}
Für $\pmb{S}$ lassen sich folgene Aussagen tätigen:
\begin{align*}
&\pmb{S} \textrm{ ist unitär,} \\
&\pmb{S} \textrm{ ist eine obere Dreiecksmatrix.}
\end{align*}
Es ist aber auch
\begin{equation*}
    \pmb{Q}_1^{-1}\pmb{Q}_2 = \pmb{R}_1 \pmb{R}_2^{-1} = \underbrace{\ \pmb{S}^{-1} = \pmb{S}^H}_{S \textrm{ ist unitär!}}.
\end{equation*}
Somit ist $\pmb{S}$ sogar eine unitäre Diagonalmatrix. Es ist also $|s_i| = 1$ für alle $i = 1, \dots, n$.

\end{document}
