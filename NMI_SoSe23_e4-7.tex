\documentclass[a4paper,12pt]{article}
\usepackage{fancyhdr}
\usepackage[ngerman,german]{babel}
\usepackage{german}
\usepackage[utf8]{inputenc}
\usepackage[active]{srcltx}
\usepackage{algorithm}
\usepackage[noend]{algorithmic}
\usepackage{amsmath}
\usepackage{amssymb}
\usepackage{amsthm}
\usepackage{bbm}
\usepackage{enumerate}
\usepackage{graphicx}
\usepackage{ifthen}
\usepackage{listings}
\usepackage{struktex}
\usepackage{hyperref}
\usepackage[onehalfspacing]{setspace}
\usepackage{geometry}
\usepackage{calc}

%%%%%%%%%%%%%%%%%%%%%%%%%%%%%%%%%%%%%%%%%%%%%%%%%%%%%%
%%%%%%%%%%%%%% EDIT THIS PART %%%%%%%%%%%%%%%%%%%%%%%%
%%%%%%%%%%%%%%%%%%%%%%%%%%%%%%%%%%%%%%%%%%%%%%%%%%%%%%

\newcommand{\Fach}{01270 Numerische Mathematik I}
\newcommand{\FachKurz}{NMI}
\newcommand{\Name}{Oliver Schwarz}
\newcommand{\Matrikelnummer}{6883389}
\newcommand{\Semester}{SoSe 23}
\newcommand{\Kurseinheit}{4}
\newcommand{\AufgabeNummer}{7}

%%%%%%%%%%%%%%%%%%%%%%%%%%%%%%%%%%%%%%%%%%%%%%%%%%%%%%
%%%%%%%%%%%%%% DO NOT EDIT THIS PART %%%%%%%%%%%%%%%%%
%%%%%%%%%%%%%%%%%%%%%%%%%%%%%%%%%%%%%%%%%%%%%%%%%%%%%%

\newcommand{\Aufgabe}[1]{
  {
  \vspace*{0.5cm}
  \textbf{Aufgabe #1}
  \vspace*{0.2cm}
  }
}

\newcommand{\cond}[0]{
  {
  \textrm{cond}
  }
}

% Beschreibung
% Unterscheidet im Mathe-Mode zwischen
% "Dezimal-komma" und Satzzeichen.
% Muss bei Eingabe durch kein/ein Leerzeichen
% dahinter angegeben werden.
% Literatur Richard Hirsch in: DTK, 1/1994, S. 42 ff.
\mathchardef\CommaOrdinary="013B \mathchardef\CommaPunct="613B
\mathcode`,="8000 % , im Math-Mode aktiv ("8000) machen
{\catcode`\,=\active \gdef
,{\obeyspaces\futurelet\next\CommaCheck}}
\def\CommaCheck{\if\space\next\CommaPunct\else\CommaOrdinary\fi}

\newcommand*\diff{\mathop{}\!\mathrm{\,d}}
%%%%%%%%%%%%%%%%%%%%%%%%%%%%%%%%%%%%%%%%%%%%%%%%%%%%%%
%%%%%%%%%%%%%% PAGE SETTINGS %%%%%%%%%%%%%%%%%%%%%%%%%
%%%%%%%%%%%%%%%%%%%%%%%%%%%%%%%%%%%%%%%%%%%%%%%%%%%%%%

\setlength{\parindent}{0em}
\topmargin 0cm
\oddsidemargin 0cm
\evensidemargin 0cm

\geometry{%
  left=45.0mm,
  right=15.0mm,
  top=25mm,
  bottom=25mm,
  bindingoffset=0mm,
  headheight=30pt,
  includehead
}

\fancyheadoffset[L]{20mm}
\renewcommand{\headrulewidth}{1pt}

\pagestyle{fancy}

%%%%%%%%%%%%%%%%%%%%%%%%%%%%%%%%%%%%%%%%%%%%%%%%%%%%%%
%%%%%%%%%%%%%% PDF SETTINGS %%%%%%%%%%%%%%%%%%%%%%%%%%
%%%%%%%%%%%%%%%%%%%%%%%%%%%%%%%%%%%%%%%%%%%%%%%%%%%%%%

\hypersetup{
    %pdftitle={\Fach{}: Übungsblatt \Uebungsblatt{}},
    pdftitle={\Fach{}: Kurseinheit \Kurseinheit{} Aufgabe \AufgabeNummer{}},
    pdfauthor={\Name},
    pdfborder={0 0 0}
}

%%%%%%%%%%%%%%%%%%%%%%%%%%%%%%%%%%%%%%%%%%%%%%%%%%%%%%
%%%%%%%%%%%%%% CODE SETTINGS %%%%%%%%%%%%%%%%%%%%%%%%%
%%%%%%%%%%%%%%%%%%%%%%%%%%%%%%%%%%%%%%%%%%%%%%%%%%%%%%

\lstset{ %
language=java,
basicstyle=\footnotesize\tt,
showtabs=false,
tabsize=2,
captionpos=b,
breaklines=true,
extendedchars=true,
showstringspaces=false,
flexiblecolumns=true,
}

%%%%%%%%%%%%%%%%%%%%%%%%%%%%%%%%%%%%%%%%%%%%%%%%%%%%%%
%%%%%%%%%%%%%% DOCUMENT %%%%%%%%%%%%%%%%%%%%%%%%%%%%%%
%%%%%%%%%%%%%%%%%%%%%%%%%%%%%%%%%%%%%%%%%%%%%%%%%%%%%%

\title{Kurseinheit \Kurseinheit{} Aufgabe \AufgabeNummer{}}
\author{\Name{}}

\begin{document}

%\renewcommand{\theequation}{L\Kurseinheit{}.\AufgabeNummer{}.\arabic{equation}}
\renewcommand{\theequation}{L \AufgabeNummer{}.\arabic{equation}}

%%%%%%%%%%%%%%%%%%%%%%%%%%%%%%%%%%%%%%%%%%%%%%%%%%%%%%
%%%%%%%%%%%%%% HEADER %%%%%%%%%%%%%%%%%%%%%%%%%%%%%%%%
%%%%%%%%%%%%%%%%%%%%%%%%%%%%%%%%%%%%%%%%%%%%%%%%%%%%%%

\lhead{\sf \large \Fach{} \\ \small \Name{} - \Matrikelnummer{}}
%\rhead{\sf \Semester{}}
\rhead{\sf \FachKurz{} \quad E \Kurseinheit{}/\AufgabeNummer{}}

%%%%%%%%%%%%%%%%%%%%%%%%%%%%%%%%%%%%%%%%%%%%%%%%%%%%%%
%%%%%%%%%%%%%% START HERE %%%%%%%%%%%%%%%%%%%%%%%%%%%%
%%%%%%%%%%%%%%%%%%%%%%%%%%%%%%%%%%%%%%%%%%%%%%%%%%%%%%

\Aufgabe{\AufgabeNummer{}}
\begin{enumerate}[a)]
    \item %a
Bilden die Ableitungen $k$-ter Ordnung der \textsc{Legendre}-Polynome $P_{i+k}$ ein System orthogonaler Polynome bezüglich der Gewichtsfunktion $\varrho_{k,k}(x)$, so ist (mit $i \neq j$)
$$\big\langle P_{i+k}^{(k)}, P_{j+k}^{(k)} \big\rangle_{\varrho_{k,k}} = 0 = \int\limits_{-1}^{1} P_{i+k}^{(k)}(x) P_{j+k}^{(k)}(x) \varrho_{k,k}(x) \diff x.$$
Die Gewichtsfunktion $\varrho_{k,k}(x) := (x^2 - 1)^k$ hat die Nullstellen $x = 1$ und $x = -1$ mit einer Vielfachheit von $k$ und ist selbst ein Polynom vom Grad $k \geq 0$. Das Polynom $P_{j+k}^{(k)}$ hat wiederum den Grad $j \geq 0$. Also ist das Hilfspolynom $$P_{j+k}^{(k)} \varrho_{k,k}(x) =: \psi(x)$$ ein Polynom vom Grad $(j+2k) \geq 0$ mit den Nullstellen $x = 1$ und $x = -1$ mit (mindestens) einer Vielfachheit von $k$.

Wir integrieren $k-$-mal partiell und mit $\psi^{(k)}(1) = \psi^{(k)}(-1) = 0$
\begin{align*}
    \int\limits_{-1}^{1} P_{i+k}^{(k)}(x) P_{j+k}^{(k)}(x) \varrho_{k,k}(x) \diff x = (-1)^k \int\limits_{-1}^{1} P_{i+k}(x) \psi^{(k)}(x) \diff x = (-1)^k \big\langle P_{i+k}^{(k)}, \psi^{(k)} \big\rangle_{\varrho_*}
\end{align*}

Nun ist die Gewichtsfunktion $\varrho_* : (-1, 1) \rightarrow \mathbb{R}_+ : x \mapsto \varrho_*(x) = 1$. Laut Definition sind die \textsc{Legendre}-Polynome bezüglich dieser Gewichtsfunktion orthogonal:
\begin{align*}
    (-1)^k \big\langle P_{i+k}^{(k)}, \psi^{(k)} \big\rangle_{\varrho_*} = 0 = \big\langle P_{i+k}^{(k)}, P_{j+k}^{(k)} \big\rangle_{\varrho_{k,k}}.
\end{align*}

\item %b)
Wir betrachten die (leicht veränderte) \textsc{Rodrigues}-Formel nach 5.23 $$P_{i+1}(x) = \frac{1}{2^{i+1} (i+1)!} \, \frac{\diff^{i+1}}{\diff x^{i+1}} \left( (x^2 - 1)^{i+1} \right)$$ und das Hilfspolynom $\psi_i(x) = 2^i i! P_i(x)$ $$\psi_{i+1}(x) = \frac{\diff^{i+1}}{\diff x^{i+1}} \left( (x^2 - 1)^{i+1} \right).$$

\begin{align*}
    \psi_{i+1}'(x) &= \frac{\diff^{i+2}}{\diff x^{i+2}} \left( (x^2 - 1)^{i+1} \right) \\ %\textrm{ und wegen } \frac{\diff}{\diff x}\left( (x^2 - 1)^n\right) = 2nx(x^2 - 1)^{n-1}\\
    &=  2(i+1) \frac{\diff^{i}}{\diff x^{i}} \left( \frac{\diff}{\diff x}\left( x (x^2 - 1)^{i} \right)\right) \\
    &=  2(i+1) \frac{\diff^{i}}{\diff x^{i}} \left( (x^2 - 1)^{i} + 2ix^2 (x^2 - 1)^{i-1} \right) \\
    &=  2(i+1) \frac{\diff^{i}}{\diff x^{i}} \left( (x^2 - 1)^{i-1} ((x^2-1)+ 2ix^2) \right) \\
    &=  2(i+1) \frac{\diff^{i}}{\diff x^{i}} \left( (x^2 - 1)^{i-1} (2ix^2 - 2i + x^2 - 1 + 2i) \right) \\
    &=  2(i+1) \frac{\diff^{i}}{\diff x^{i}} \left( (x^2 - 1)^{i-1} ((2i+1)(x^2-1)+2i) \right) \\
    &=  2(i+1) \frac{\diff^{i}}{\diff x^{i}} \left( (2i+1)(x^2 - 1)^{i} + 2i(x^2 - 1)^{i-1}\right) \\
    &=  2(i+1) \left((2i+1) \frac{\diff^{i}}{\diff x^{i}} \left( (x^2 - 1)^{i}\right) + 2i \frac{\diff^{i}}{\diff x^{i}} \left((x^2 - 1)^{i-1}\right)\right) \\
\psi_{i+1}'(x)  &=  2(i+1) \left((2i+1) \psi_{i}(x) + 2i \psi_{i-1}'(x)\right).
\end{align*}

Mit $\psi_i(x) = 2^i i! P_i(x)$ erhalten wir
\begin{align*}
2^{i+1} (i+1)! P'_{i+1}(x) &= 2(i+1) \left((2i+1) 2^i i! P_i(x) + 2i 2^{i-1} (i-1)! P'_{i-1}(x)\right) \\
P'_{i+1}(x) &= \frac{1}{2^i i!} \left((2i+1) 2^i i! P_i(x) + 2^i i! P'_{i-1}(x)\right) \\
P'_{i+1}(x) &= (2i+1) P_i(x) + P'_{i-1}(x).
\end{align*}
Durch Umstellung folgt 
\begin{align}
 (2i+1) P_i(x) &= P'_{i+1}(x) - P'_{i-1}(x).  \label{eqn7_1}
 \end{align}

\item %c)
Mit der \textsc{Rodrigues}-Formel nach 5.23 ist
$$P_i^{(k)}(x) = \displaystyle\frac{1}{2^i i!} \, \frac{\diff^{i+k}}{\diff^{i+k} x} \left( (x^2 - 1)^i\right),$$
wobei $P_i^{(k)}(x) = 0$ für $k>i$ ist, oder
$$\boxed{P_{k-1}^{(k)}(x) = 0.}$$

Für $P_k^{(k)}$ gilt also $$P_k^{(k)}(x) = \displaystyle\frac{1}{2^k k!} \, \frac{\diff^{2k}}{\diff^{2k} x} \left( (x^2 - 1)^k\right) = \boxed{\frac{(2k)!}{2^k k!}.}$$

Die dreigliedrige Rekursion nach 5.24 ist
$$(i+1)P_{i+1} = (2i+1) \mu_1 P_i - i P_{i-1},\; i \in \mathbb{N}, \; P_1 = \mu_1, \; P_0 = \mu_0.$$

Mit der Leibniz-Regel $(fg)^{(n)} = \sum_{k=0}^n \binom{n}{k}f^{(n)}g^{(n-k)}$ differenzieren wir $k$-mal:
\begin{align}
(i+1)P^{(k)}_{i+1} &= (2i+1) \sum\limits_{\nu = 0}^{k} \binom{k}{\nu}\mu^{(\nu)}_1 P^{(k-\nu)}_i - i P^{(k)}_{i-1} \quad &[\mu_1^{(\nu)} = 0 \textrm{ für } \nu > 1] \nonumber \\
&= (2i+1) \left(\binom{k}{0}\mu_1 P^{(k)}_{i} + \binom{k}{1}P^{(k-1)}_{i} \right) - i P^{(k)}_{i-1} \nonumber \\
(i+1)P^{(k)}_{i+1} &= (2i+1) \left(\mu_1 P^{(k)}_{i} + kP^{(k-1)}_{i} \right) - i P^{(k)}_{i-1}. \label{eqn7_2}
\end{align}

Aus \ref{eqn7_1} erhalten wir 
\begin{align}
(2i+1) P_i^{(k-1)}(x) &= P^{(k)}_{i+1}(x) - P^{(k)}_{i-1}(x) \label{eqn7_3}.   
\end{align}

Wir setzen \ref{eqn7_3} in \ref{eqn7_2} ein: $$(i+1)P^{(k)}_{i+1} = (2i+1) \mu_1 P^{(k)}_{i} + kP^{(k)}_{i+1}(x) - kP^{(k)}_{i-1}(x)  - i P^{(k)}_{i-1}.$$

Daraus erhalten wir die Behauptung $$\boxed{(i+1 -1)P^{(k)}_{i+1} = (2i+1) \mu_1 P^{(k)}_{i} - (i+k)P^{(k)}_{i-1}(x).}$$
\end{enumerate}


\end{document}
