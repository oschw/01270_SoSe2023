\documentclass[a4paper,12pt]{article}
\usepackage{fancyhdr}
\usepackage[ngerman,german]{babel}
\usepackage{german}
\usepackage[utf8]{inputenc}
\usepackage[active]{srcltx}
\usepackage{algorithm}
\usepackage[noend]{algorithmic}
\usepackage{amsmath}
\usepackage{amssymb}
\usepackage{amsthm}
\usepackage{bbm}
\usepackage{enumerate}
\usepackage{graphicx}
\usepackage{ifthen}
\usepackage{listings}
\usepackage{struktex}
\usepackage{hyperref}
\usepackage[onehalfspacing]{setspace}
\usepackage{geometry}
\usepackage{calc}
\usepackage{arydshln}

%%%%%%%%%%%%%%%%%%%%%%%%%%%%%%%%%%%%%%%%%%%%%%%%%%%%%%
%%%%%%%%%%%%%% EDIT THIS PART %%%%%%%%%%%%%%%%%%%%%%%%
%%%%%%%%%%%%%%%%%%%%%%%%%%%%%%%%%%%%%%%%%%%%%%%%%%%%%%

\newcommand{\Fach}{01270 Numerische Mathematik I}
\newcommand{\FachKurz}{NMI}
\newcommand{\Name}{Oliver Schwarz}
\newcommand{\Matrikelnummer}{6883389}
\newcommand{\Semester}{SoSe 23}
\newcommand{\Kurseinheit}{3}
\newcommand{\AufgabeNummer}{6}

%%%%%%%%%%%%%%%%%%%%%%%%%%%%%%%%%%%%%%%%%%%%%%%%%%%%%%
%%%%%%%%%%%%%% DO NOT EDIT THIS PART %%%%%%%%%%%%%%%%%
%%%%%%%%%%%%%%%%%%%%%%%%%%%%%%%%%%%%%%%%%%%%%%%%%%%%%%

\newcommand{\Aufgabe}[1]{
  {
  \vspace*{0.5cm}
  \textbf{Aufgabe #1}
  \vspace*{0.2cm}
  }
}

\newcommand{\cond}[0]{
  {
  \textrm{cond}
  }
}

% Beschreibung
% Unterscheidet im Mathe-Mode zwischen
% "Dezimal-komma" und Satzzeichen.
% Muss bei Eingabe durch kein/ein Leerzeichen
% dahinter angegeben werden.
% Literatur Richard Hirsch in: DTK, 1/1994, S. 42 ff.
\mathchardef\CommaOrdinary="013B \mathchardef\CommaPunct="613B
\mathcode`,="8000 % , im Math-Mode aktiv ("8000) machen
{\catcode`\,=\active \gdef
,{\obeyspaces\futurelet\next\CommaCheck}}
\def\CommaCheck{\if\space\next\CommaPunct\else\CommaOrdinary\fi}

%%%%%%%%%%%%%%%%%%%%%%%%%%%%%%%%%%%%%%%%%%%%%%%%%%%%%%
%%%%%%%%%%%%%% PAGE SETTINGS %%%%%%%%%%%%%%%%%%%%%%%%%
%%%%%%%%%%%%%%%%%%%%%%%%%%%%%%%%%%%%%%%%%%%%%%%%%%%%%%

\setlength{\parindent}{0em}
\topmargin 0cm
\oddsidemargin 0cm
\evensidemargin 0cm

\geometry{%
  left=45.0mm,
  right=15.0mm,
  top=25mm,
  bottom=25mm,
  bindingoffset=0mm,
  headheight=30pt,
  includehead
}

\fancyheadoffset[L]{20mm}
\renewcommand{\headrulewidth}{1pt}

\pagestyle{fancy}

%%%%%%%%%%%%%%%%%%%%%%%%%%%%%%%%%%%%%%%%%%%%%%%%%%%%%%
%%%%%%%%%%%%%% PDF SETTINGS %%%%%%%%%%%%%%%%%%%%%%%%%%
%%%%%%%%%%%%%%%%%%%%%%%%%%%%%%%%%%%%%%%%%%%%%%%%%%%%%%

\hypersetup{
    %pdftitle={\Fach{}: Übungsblatt \Uebungsblatt{}},
    pdftitle={\Fach{}: Kurseinheit \Kurseinheit{} Aufgabe \AufgabeNummer{}},
    pdfauthor={\Name},
    pdfborder={0 0 0}
}

%%%%%%%%%%%%%%%%%%%%%%%%%%%%%%%%%%%%%%%%%%%%%%%%%%%%%%
%%%%%%%%%%%%%% CODE SETTINGS %%%%%%%%%%%%%%%%%%%%%%%%%
%%%%%%%%%%%%%%%%%%%%%%%%%%%%%%%%%%%%%%%%%%%%%%%%%%%%%%

\lstset{ %
language=java,
basicstyle=\footnotesize\tt,
showtabs=false,
tabsize=2,
captionpos=b,
breaklines=true,
extendedchars=true,
showstringspaces=false,
flexiblecolumns=true,
}

%%%%%%%%%%%%%%%%%%%%%%%%%%%%%%%%%%%%%%%%%%%%%%%%%%%%%%
%%%%%%%%%%%%%% DOCUMENT %%%%%%%%%%%%%%%%%%%%%%%%%%%%%%
%%%%%%%%%%%%%%%%%%%%%%%%%%%%%%%%%%%%%%%%%%%%%%%%%%%%%%

\title{Kurseinheit \Kurseinheit{} Aufgabe \AufgabeNummer{}}
\author{\Name{}}

\begin{document}

%\renewcommand{\theequation}{L\Kurseinheit{}.\AufgabeNummer{}.\arabic{equation}}
\renewcommand{\theequation}{L \AufgabeNummer{}.\arabic{equation}}

%%%%%%%%%%%%%%%%%%%%%%%%%%%%%%%%%%%%%%%%%%%%%%%%%%%%%%
%%%%%%%%%%%%%% HEADER %%%%%%%%%%%%%%%%%%%%%%%%%%%%%%%%
%%%%%%%%%%%%%%%%%%%%%%%%%%%%%%%%%%%%%%%%%%%%%%%%%%%%%%

\lhead{\sf \large \Fach{} \\ \small \Name{} - \Matrikelnummer{}}
%\rhead{\sf \Semester{}}
\rhead{\sf \FachKurz{} \quad E \Kurseinheit{}/\AufgabeNummer{}}

%%%%%%%%%%%%%%%%%%%%%%%%%%%%%%%%%%%%%%%%%%%%%%%%%%%%%%
%%%%%%%%%%%%%% START HERE %%%%%%%%%%%%%%%%%%%%%%%%%%%%
%%%%%%%%%%%%%%%%%%%%%%%%%%%%%%%%%%%%%%%%%%%%%%%%%%%%%%

\Aufgabe{\AufgabeNummer{}}

Es ist die Matrix $\pmb{A} = \textrm{tridiag}(\pmb{e}, \pmb{d}, \pmb{f}) \in \mathbb{C}^{7,7}$ und $e_i = -2, i = {2, \dots, 7}$, $f_i = -1, i = 1, \dots, 6$, $d_1 = 2$ und $d_i = 3, i = 2, \dots, 7$.

\begin{enumerate}[a)]
    \item %a)

\begin{align*}
%\intertext{If}
&\text{S1.} \quad &&\\
&&&\begin{array}[t]{ c | l | l | l}
i & l := e_i/d_{i-1} & d_i := d_i - l \, f_{i-1} & b_i := b_i - l \, b_{i-1} \\\cline{1-4}
2 & l = -1 & d_2 =  2 &  b_2 = -3\\
3 & l = -1 & d_3 =  2 &  b_3 = -2\\
4 & l = -1 & d_4 =  2 &  b_4 = -1\\
5 & l = -1 & d_5 =  2 &  b_5 =  0\\
6 & l = -1 & d_6 =  2 &  b_6 =  1\\
7 & l = -1 & d_7 =  2 &  b_7 =  6\\
\end{array} \\
&\text{S2.} \quad &&x_7 := b_7/d_7 = 3 \\
&&&\begin{array}[t]{ c | l}
i & x_i := (b_i - f_i \, x_{i+1}) / d_i\\\cline{1-2}
6 & x_6 =  2\\
5 & x_5 =  1\\
4 & x_4 =  0\\
3 & x_3 = -1\\
2 & x_2 = -2\\
1 & x_1 = -3\\
\end{array}
\end{align*}
\item  %b

\begin{equation*}
m := \left \lfloor \frac{n+1}{2} \right \rfloor = 4
\end{equation*}
\begin{align*}
&i = 1 \quad &\\
&&\text{S1.} \quad &\\
&&(i &= 1) \\
%l   := e_i/d_{i-1}
%d_i := d_i - l f_{i-1}
%b_i := b_i - l b_{i-1}
%e_i := -l e_{i-1}
&&\text{S2.} \quad &\\
&&l &:= \frac{f_1}{d_2} = - \frac{1}{3} &\quad d_1 &:= d_1 - l \, e_2 = \frac{4}{3} \\
&&b_1 &:= b_1 - l \, b_2 = - \frac{11}{3} &\quad f_1 &:= -l \, f_2 = - \frac{1}{3}
\end{align*}

\begin{align*}
&i = 3 \quad &\\
&&\text{S1.} \quad &\\
&&l &:= \frac{e_3}{d_2} = - \frac{2}{3} &\quad d_3 &:= d_3 - l \, f_2 = \frac{7}{3} \\
&&b_3 &:= b_3 - l \, b_2 = \frac{5}{3} &\quad e_3 &:= -l \, e_2 = - \frac{4}{3} \\
&&\text{S2.} \quad &\\
&&l &:= \frac{f_3}{d_4} = - \frac{1}{3} &\quad d_3 &:= d_3 - l \, e_4 = \frac{5}{3} \\
&&b_3 &:= b_3 - l \, b_4 = 2 &\quad f_3 &:= -l \, f_4 = - \frac{1}{3}
\end{align*}

\begin{align*}
&i = 5 \quad &\\
&&\text{S1.} \quad &\\
&&l &:= \frac{e_5}{d_4} = - \frac{2}{3} &\quad d_5 &:= d_5 - l \, f_4 = \frac{7}{3} \\
&&b_5 &:= b_5 - l \, b_4 = \frac{5}{3} &\quad e_5 &:= -l \, e_4 = - \frac{4}{3} \\
&&\text{S2.} \quad &\\
&&l &:= \frac{f_3}{d_4} = - \frac{1}{3} &\quad d_5 &:= d_5 - l \, e_6 = \frac{5}{3} \\
&&b_5 &:= b_5 - l \, b_6 = 2 &\quad f_5 &:= -l \, f_6 = - \frac{1}{3}
\end{align*}

\begin{align*}
&i = 7 \quad &\\
&&\text{S1.} \quad &\\
&&l &:= \frac{e_7}{d_6} = - \frac{2}{3} &\quad d_7 &:= d_7 - l \, f_6 = \frac{7}{3} \\
&&b_7 &:= b_7 - l \, b_6 = \frac{17}{3} &\quad e_7 &:= -l \, e_6 = - \frac{4}{3} \\
&&\text{S2.} \quad &\\
&&(i &= 7)
\end{align*}

\textit{Ergebnis:}
\begin{equation*}
\begin{pmatrix}
 \frac{4}{3} & -\frac{1}{3} &  0           & 0 \\
-\frac{4}{3} &  \frac{5}{3} & -\frac{1}{3} & 0 \\
 0           & -\frac{4}{3} &  \frac{5}{3} & -\frac{1}{3} \\
 0           & 0            & -\frac{4}{3} &  \frac{7}{3}
\end{pmatrix}
\begin{pmatrix}
x_1 \\
x_3 \\
x_5 \\
x_7
\end{pmatrix} = 
\begin{pmatrix}
-\frac{11}{3} \\
2 \\
2 \\
\frac{17}{3}
\end{pmatrix}.
\end{equation*}
Der Vektor $\hat{\pmb{x}} = (-3, -1, 1, 3)^T$ löst das Gleichungssystem.

\item %c
\begin{equation*}
m := \left \lfloor \frac{n+1}{2} \right \rfloor = 2
\end{equation*}
\begin{align*}
&i = 1 \quad &\\
&&\text{S1.} \quad &\\
&&(i &= 1) \\
%l   := e_i/d_{i-1}
%d_i := d_i - l f_{i-1}
%b_i := b_i - l b_{i-1}
%e_i := -l e_{i-1}
&&\text{S2.} \quad &\\
&&l &:= \frac{f_1}{d_2} = - \frac{1}{5} &\quad d_1 &:= d_1 - l \, e_2 = \frac{16}{15} \\
&&b_1 &:= b_1 - l \, b_2 = - \frac{49}{15} &\quad f_1 &:= -l \, f_2 = - \frac{1}{15}
\end{align*}

\begin{align*}
&i = 3 \quad &\\
&&\text{S1.} \quad &\\
&&l &:= \frac{e_3}{d_2} = - \frac{4}{5} &\quad d_3 &:= d_3 - l \, f_2 = \frac{7}{5} \\
&&b_3 &:= b_3 - l \, b_2 = \frac{18}{5} &\quad e_3 &:= -l \, e_2 = - \frac{16}{5} \\
&&\text{S2.} \quad &\\
&&l &:= \frac{f_3}{d_4} = - \frac{1}{7} &\quad d_3 &:= d_3 - l \, e_4 = \frac{127}{105} \\
&&b_3 &:= b_3 - l \, b_4 = \frac{463}{105} &\quad f_3 &:= -l \, f_4 = - \frac{1}{21}
\end{align*}

\textit{Ergebnis:}
\begin{equation*}
\begin{pmatrix}
 \frac{16}{15} & -\frac{1}{15} \\
-\frac{16}{15} &  \frac{127}{105}
\end{pmatrix}
\begin{pmatrix}
x_1 \\
x_5
\end{pmatrix} = 
\begin{pmatrix}
-\frac{49}{15} \\
\frac{463}{105}
\end{pmatrix}.
\end{equation*}
Der Vektor $\hat{\pmb{x}} = (-3, 1)^T$ löst das Gleichungssystem.

\end{enumerate}
\end{document}
