\documentclass[a4paper,12pt]{article}
\usepackage{fancyhdr}
\usepackage[ngerman,german]{babel}
\usepackage{german}
\usepackage[utf8]{inputenc}
\usepackage[active]{srcltx}
\usepackage{algorithm}
\usepackage[noend]{algorithmic}
\usepackage{amsmath}
\usepackage{amssymb}
\usepackage{amsthm}
\usepackage{bbm}
\usepackage{enumerate}
\usepackage{graphicx}
\usepackage{ifthen}
\usepackage{listings}
\usepackage{struktex}
\usepackage{hyperref}
\usepackage[onehalfspacing]{setspace}
\usepackage{geometry}
\usepackage{calc}

%%%%%%%%%%%%%%%%%%%%%%%%%%%%%%%%%%%%%%%%%%%%%%%%%%%%%%
%%%%%%%%%%%%%% EDIT THIS PART %%%%%%%%%%%%%%%%%%%%%%%%
%%%%%%%%%%%%%%%%%%%%%%%%%%%%%%%%%%%%%%%%%%%%%%%%%%%%%%

\newcommand{\Fach}{01270 Numerische Mathematik I}
\newcommand{\FachKurz}{NMI}
\newcommand{\Name}{Oliver Schwarz}
\newcommand{\Matrikelnummer}{6883389}
\newcommand{\Semester}{SoSe 23}
\newcommand{\Kurseinheit}{1}
\newcommand{\AufgabeNummer}{3}

%%%%%%%%%%%%%%%%%%%%%%%%%%%%%%%%%%%%%%%%%%%%%%%%%%%%%%
%%%%%%%%%%%%%% DO NOT EDIT THIS PART %%%%%%%%%%%%%%%%%
%%%%%%%%%%%%%%%%%%%%%%%%%%%%%%%%%%%%%%%%%%%%%%%%%%%%%%

\newcommand{\Aufgabe}[1]{
  {
  \vspace*{0.5cm}
  \textbf{Aufgabe #1}
  \vspace*{0.2cm}
  }
}

%Maschinengenauigkeit eps
\newcommand{\eps}[0]{
  {
  \mathtt{eps}
  }
}

%epsilon
\newcommand{\e}[0]{
  {
  \varepsilon
  }
}

%%%%%%%%%%%%%%%%%%%%%%%%%%%%%%%%%%%%%%%%%%%%%%%%%%%%%%
%%%%%%%%%%%%%% PAGE SETTINGS %%%%%%%%%%%%%%%%%%%%%%%%%
%%%%%%%%%%%%%%%%%%%%%%%%%%%%%%%%%%%%%%%%%%%%%%%%%%%%%%

\setlength{\parindent}{0em}
\topmargin 0cm
\oddsidemargin 0cm
\evensidemargin 0cm

\geometry{%
  left=45.0mm,
  right=15.0mm,
  top=25mm,
  bottom=25mm,
  bindingoffset=0mm,
  headheight=30pt,
  includehead
}

\fancyheadoffset[L]{20mm}
\renewcommand{\headrulewidth}{1pt}

\pagestyle{fancy}

%%%%%%%%%%%%%%%%%%%%%%%%%%%%%%%%%%%%%%%%%%%%%%%%%%%%%%
%%%%%%%%%%%%%% PDF SETTINGS %%%%%%%%%%%%%%%%%%%%%%%%%%
%%%%%%%%%%%%%%%%%%%%%%%%%%%%%%%%%%%%%%%%%%%%%%%%%%%%%%

\hypersetup{
    %pdftitle={\Fach{}: Übungsblatt \Uebungsblatt{}},
    pdftitle={\Fach{}: Kurseinheit \Kurseinheit{} Aufgabe \AufgabeNummer{}},
    pdfauthor={\Name},
    pdfborder={0 0 0}
}

%%%%%%%%%%%%%%%%%%%%%%%%%%%%%%%%%%%%%%%%%%%%%%%%%%%%%%
%%%%%%%%%%%%%% CODE SETTINGS %%%%%%%%%%%%%%%%%%%%%%%%%
%%%%%%%%%%%%%%%%%%%%%%%%%%%%%%%%%%%%%%%%%%%%%%%%%%%%%%

\lstset{ %
language=java,
basicstyle=\footnotesize\tt,
showtabs=false,
tabsize=2,
captionpos=b,
breaklines=true,
extendedchars=true,
showstringspaces=false,
flexiblecolumns=true,
}

%%%%%%%%%%%%%%%%%%%%%%%%%%%%%%%%%%%%%%%%%%%%%%%%%%%%%%
%%%%%%%%%%%%%% DOCUMENT %%%%%%%%%%%%%%%%%%%%%%%%%%%%%%
%%%%%%%%%%%%%%%%%%%%%%%%%%%%%%%%%%%%%%%%%%%%%%%%%%%%%%

\title{Kurseinheit \Kurseinheit{} Aufgabe \AufgabeNummer{}}
\author{\Name{}}

\begin{document}

%\renewcommand{\theequation}{L\Kurseinheit{}.\AufgabeNummer{}.\arabic{equation}}
\renewcommand{\theequation}{L \AufgabeNummer{}.\arabic{equation}}

%%%%%%%%%%%%%%%%%%%%%%%%%%%%%%%%%%%%%%%%%%%%%%%%%%%%%%
%%%%%%%%%%%%%% HEADER %%%%%%%%%%%%%%%%%%%%%%%%%%%%%%%%
%%%%%%%%%%%%%%%%%%%%%%%%%%%%%%%%%%%%%%%%%%%%%%%%%%%%%%

\lhead{\sf \large \Fach{} \\ \small \Name{} - \Matrikelnummer{}}
%\rhead{\sf \Semester{}}
\rhead{\sf \FachKurz{} \quad E \Kurseinheit{}/\AufgabeNummer{}}

%%%%%%%%%%%%%%%%%%%%%%%%%%%%%%%%%%%%%%%%%%%%%%%%%%%%%%
%%%%%%%%%%%%%% START HERE %%%%%%%%%%%%%%%%%%%%%%%%%%%%
%%%%%%%%%%%%%%%%%%%%%%%%%%%%%%%%%%%%%%%%%%%%%%%%%%%%%%

\Aufgabe{\AufgabeNummer{} \quad (Rundungsfehler)}

Wir betrachten die Gleitkommazahlen $x, y \in \mathbb{M}_b (m, [e_*, e^*])$ und berechnen
\begin{equation*}
z := x^2 - y^2 = (x-y)(x+y).    
\end{equation*}

Weiterhin sei $\eps$ die Maschinengenauigkeit.


\begin{enumerate}[a)]
\item % a)
$z_1 := (x \otimes x) \ominus (y \otimes y)$

Es gilt
\begin{align*}
    z_1 &= (x \otimes x) \ominus (y \otimes y) = ((1 + \e_1)x^2 - (1 + \e_2)y^2)(1 + \e_3) \\
    &= (x^2 - y^2) + (\e_1 + \e_3 + \e_1\e_3) x^2 - (\e_2 + \e_3 + \e_2\e_3) y^2
\end{align*}
mit $|\e_i| \leq \eps$ und $i = 1, 2, 3$. 

Wir berechnen den relativen Fehler:
\begin{align*}
\frac{z_1 - z}{z} &= \frac{(x^2 - y^2) + (\e_1 + \e_3 + \e_1\e_3) x^2 - (\e_2 + \e_3 + \e_2\e_3) y^2 - (x^2 - y^2)}{(x^2 - y^2)} \\
                  &= \frac{(\e_1 + \e_3 + \e_1\e_3) x^2 - (\e_2 + \e_3 + \e_2\e_3) y^2}{(x^2 - y^2)} \\
                  &\doteqdot (\e_1 + \e_3) \frac{ x^2}{(x^2 - y^2)} - (\e_2 + \e_3) \frac{y^2}{(x^2 - y^2)}.
\end{align*}
Demnach wird der Fehler für $x \approx y$ groß.

\item % b)
$z_2 := (x \ominus y) \otimes (x \oplus y)$

Es ist
\begin{align*}
    z_2 &= (x \ominus y) \otimes (x \oplus y) = (1 + \e_1)(x - y) (1 + \e_2)(x + y) (1 + \e_3) \\
    &= (x^2 - y^2)(1 + \e_1)(1 + \e_2)(1 + \e_3).
\end{align*}
mit $|\e_i| \leq \eps$ und $i = 1, 2, 3$. 

Wir berechnen den relativen Fehler:
\begin{align*}
\frac{z_2 - z}{z} &= \frac{(x^2 - y^2)(1 + \e_1)(1 + \e_2)(1 + \e_3) - (x^2 - y^2)}{(x^2 - y^2)} \\
                  &= \frac{((1 + \e_1)(1 + \e_2)(1 + \e_3) - 1)(x^2 - y^2)}{(x^2 - y^2)} \\
                  &= (1 + \e_1)(1 + \e_2)(1 + \e_3) - 1 \\
                  &= 1 + \e_1 + \e_2 + \e_3 + \e_1\e_2 + \e_1\e_3 + \e_2\e_3 + \e_1\e_2\e_3 - 1 \\
                  &\doteqdot \e_1 + \e_2 + \e_3.
\end{align*}
Demnach ist der relative Fehler von $z_2$ immer maximal $3 \eps$ und unabhängig von $x$ und $y$. Somit ist $z_2$ gegenüber $z_1$ vorzuziehen.
\end{enumerate}
\end{document}
