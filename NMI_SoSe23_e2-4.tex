\documentclass[a4paper,12pt]{article}
\usepackage{fancyhdr}
\usepackage[ngerman,german]{babel}
\usepackage{german}
\usepackage[utf8]{inputenc}
\usepackage[active]{srcltx}
\usepackage{algorithm}
\usepackage[noend]{algorithmic}
\usepackage{amsmath}
\usepackage{amssymb}
\usepackage{amsthm}
\usepackage{bbm}
\usepackage{enumerate}
\usepackage{graphicx}
\usepackage{ifthen}
\usepackage{listings}
\usepackage{struktex}
\usepackage{hyperref}
\usepackage[onehalfspacing]{setspace}
\usepackage{geometry}
\usepackage{calc}

%%%%%%%%%%%%%%%%%%%%%%%%%%%%%%%%%%%%%%%%%%%%%%%%%%%%%%
%%%%%%%%%%%%%% EDIT THIS PART %%%%%%%%%%%%%%%%%%%%%%%%
%%%%%%%%%%%%%%%%%%%%%%%%%%%%%%%%%%%%%%%%%%%%%%%%%%%%%%

\newcommand{\Fach}{01270 Numerische Mathematik I}
\newcommand{\FachKurz}{NMI}
\newcommand{\Name}{Oliver Schwarz}
\newcommand{\Matrikelnummer}{6883389}
\newcommand{\Semester}{SoSe 23}
\newcommand{\Kurseinheit}{2}
\newcommand{\AufgabeNummer}{4}

%%%%%%%%%%%%%%%%%%%%%%%%%%%%%%%%%%%%%%%%%%%%%%%%%%%%%%
%%%%%%%%%%%%%% DO NOT EDIT THIS PART %%%%%%%%%%%%%%%%%
%%%%%%%%%%%%%%%%%%%%%%%%%%%%%%%%%%%%%%%%%%%%%%%%%%%%%%

\newcommand{\Aufgabe}[1]{
  {
  \vspace*{0.5cm}
  \textbf{Aufgabe #1}
  \vspace*{0.2cm}
  }
}

%%%%%%%%%%%%%%%%%%%%%%%%%%%%%%%%%%%%%%%%%%%%%%%%%%%%%%
%%%%%%%%%%%%%% PAGE SETTINGS %%%%%%%%%%%%%%%%%%%%%%%%%
%%%%%%%%%%%%%%%%%%%%%%%%%%%%%%%%%%%%%%%%%%%%%%%%%%%%%%

\setlength{\parindent}{0em}
\topmargin 0cm
\oddsidemargin 0cm
\evensidemargin 0cm

\geometry{%
  left=45.0mm,
  right=15.0mm,
  top=25mm,
  bottom=25mm,
  bindingoffset=0mm,
  headheight=30pt,
  includehead
}

\fancyheadoffset[L]{20mm}
\renewcommand{\headrulewidth}{1pt}

\pagestyle{fancy}

%%%%%%%%%%%%%%%%%%%%%%%%%%%%%%%%%%%%%%%%%%%%%%%%%%%%%%
%%%%%%%%%%%%%% PDF SETTINGS %%%%%%%%%%%%%%%%%%%%%%%%%%
%%%%%%%%%%%%%%%%%%%%%%%%%%%%%%%%%%%%%%%%%%%%%%%%%%%%%%

\hypersetup{
    %pdftitle={\Fach{}: Übungsblatt \Uebungsblatt{}},
    pdftitle={\Fach{}: Kurseinheit \Kurseinheit{} Aufgabe \AufgabeNummer{}},
    pdfauthor={\Name},
    pdfborder={0 0 0}
}

%%%%%%%%%%%%%%%%%%%%%%%%%%%%%%%%%%%%%%%%%%%%%%%%%%%%%%
%%%%%%%%%%%%%% CODE SETTINGS %%%%%%%%%%%%%%%%%%%%%%%%%
%%%%%%%%%%%%%%%%%%%%%%%%%%%%%%%%%%%%%%%%%%%%%%%%%%%%%%

\lstset{ %
language=java,
basicstyle=\footnotesize\tt,
showtabs=false,
tabsize=2,
captionpos=b,
breaklines=true,
extendedchars=true,
showstringspaces=false,
flexiblecolumns=true,
}

%%%%%%%%%%%%%%%%%%%%%%%%%%%%%%%%%%%%%%%%%%%%%%%%%%%%%%
%%%%%%%%%%%%%% DOCUMENT %%%%%%%%%%%%%%%%%%%%%%%%%%%%%%
%%%%%%%%%%%%%%%%%%%%%%%%%%%%%%%%%%%%%%%%%%%%%%%%%%%%%%

\title{Kurseinheit \Kurseinheit{} Aufgabe \AufgabeNummer{}}
\author{\Name{}}

\begin{document}

%\renewcommand{\theequation}{L\Kurseinheit{}.\AufgabeNummer{}.\arabic{equation}}
\renewcommand{\theequation}{L \AufgabeNummer{}.\arabic{equation}}

%%%%%%%%%%%%%%%%%%%%%%%%%%%%%%%%%%%%%%%%%%%%%%%%%%%%%%
%%%%%%%%%%%%%% HEADER %%%%%%%%%%%%%%%%%%%%%%%%%%%%%%%%
%%%%%%%%%%%%%%%%%%%%%%%%%%%%%%%%%%%%%%%%%%%%%%%%%%%%%%

\lhead{\sf \large \Fach{} \\ \small \Name{} - \Matrikelnummer{}}
%\rhead{\sf \Semester{}}
\rhead{\sf \FachKurz{} \quad E \Kurseinheit{}/\AufgabeNummer{}}

%%%%%%%%%%%%%%%%%%%%%%%%%%%%%%%%%%%%%%%%%%%%%%%%%%%%%%
%%%%%%%%%%%%%% START HERE %%%%%%%%%%%%%%%%%%%%%%%%%%%%
%%%%%%%%%%%%%%%%%%%%%%%%%%%%%%%%%%%%%%%%%%%%%%%%%%%%%%

\Aufgabe{\AufgabeNummer{}}
Wir überprüfen die Normeneigenschaften. $||\pmb{A}||_G := n \cdot \max\limits_{i, j=1,\dots,n} |a_{ij}|$ ist offensichtlich positiv definit und absolut homogen. Für die Dreiecksungleichung gilt
\begin{align*}
    ||\pmb{A} + \pmb{B}||_G &= \max\limits_{i, j=1,\dots,n} |a_{ij} + b_{ij}| \leq \max\limits_{i, j=1,\dots,n} |a_{ij}| + |b_{ij}| \\
    &\leq \max\limits_{i, j=1,\dots,n} |a_{ij}| +  \max\limits_{i, j=1,\dots,n} |b_{ij}| = ||\pmb{A}||_G + ||\pmb{B}||_G.
\end{align*}
Zu verifizieren bleibt die Submultiplikativität von $||\pmb{A}||_G$: Für beliebige Matrizen $\pmb{A}, \pmb{B} \in \mathbb{K}^{n, n}$ ist zu zeigen, dass
\begin{equation*}
    ||\pmb{A} \pmb{B}||_G \leq ||\pmb{A}||_G ||\pmb{B}||_G,
\end{equation*}
was äquivalent ist zu
\begin{align*}
    ||\pmb{A} \pmb{B}||_G &= n \cdot \max\limits_{i, j = 1, \dots, n} |\sum\limits_{k=1}^{n} a_{ik}b_{kj}|  \\
    &\leq
    n \cdot \max\limits_{i, j = 1, \dots, n} \sum\limits_{k=1}^{n} |a_{ik}| \, |b_{kj}| \\
    &\leq
    n \cdot \max\limits_{i, j = 1, \dots, n} |a_{ij}| \; n \cdot \max\limits_{i, j = 1, \dots, n} |b_{ij}| = ||\pmb{A}||_G \ ||\pmb{B}||_G \\
\Longrightarrow  \quad ||\pmb{A} \pmb{B}||_G &\leq ||\pmb{A}||_G \ ||\pmb{B}||_G.
\end{align*}
Die \textbf{Gesamtnorm} $||\cdot||_G$ ist somit eine Matrixnorm.

Nun prüfen wir die Verträglichkeit mit der Vektornorm $||\cdot||_p$, $p \in [1, \infty]$.

Für $p = \infty$ 
\begin{align*}
    ||\pmb{A} \pmb{x}||_{\infty} &= \max\limits_{i=1,\dots,n} \Bigg|\sum\limits_{k=1}^n a_{ik} x_k \Bigg| \\
    &\leq n \cdot \max\limits_{i, j = 1, \dots, n} |a_{ij}| \cdot \max\limits_{i = 1, \dots, n} |x_{i}| = ||\pmb{A}||_G ||\pmb{x}||_{\infty} \\
\Longrightarrow \quad ||\pmb{A} \pmb{x}||_{\infty} &\leq ||\pmb{A}||_G ||\pmb{x}||_{\infty}.
\end{align*} 

Für $p \in [1, \infty)$
\begin{align*}
    ||\pmb{A} \pmb{x}||_p^p &= \sum\limits_{i=1}^n \Bigg| \sum\limits_{j=1}^n a_{ij} x_j \Bigg|^p \\
    &\leq \sum\limits_{i=1}^n \max\limits_{j = 1, \dots, n} |a_{ij}|^p \cdot \Bigg| \sum\limits_{j=1}^n x_j \Bigg|^p \\
    &\leq n \cdot \max\limits_{j = 1, \dots, n} |a_{ij}|^p \cdot \underbrace{||\pmb{x}||_1^p \leq n^{p-1} \cdot ||\pmb{x}||_p^p}_{\textrm{Beispiel 2.1.6}} \cdot n \cdot \max\limits_{j = 1, \dots, n} |a_{ij}|^p \\
    &= n^p \cdot \max\limits_{j = 1, \dots, n} |a_{ij}|^p \cdot ||\pmb{x}||_p^p = ||\pmb{A}||_G^p \, ||\pmb{x}||_p^p \\
\Longrightarrow \quad ||\pmb{A} \pmb{x}||_p &\leq ||\pmb{A}||_G \, ||\pmb{x}||_p
\end{align*}
\end{document}
