\documentclass[a4paper,12pt]{article}
\usepackage{fancyhdr}
\usepackage[ngerman,german]{babel}
\usepackage{german}
\usepackage[utf8]{inputenc}
\usepackage[active]{srcltx}
\usepackage{algorithm}
\usepackage[noend]{algorithmic}
\usepackage{amsmath}
\usepackage{amssymb}
\usepackage{amsthm}
\usepackage{bbm}
\usepackage{enumerate}
\usepackage{graphicx}
\usepackage{ifthen}
\usepackage{listings}
\usepackage{struktex}
\usepackage{hyperref}
\usepackage[onehalfspacing]{setspace}
\usepackage{geometry}
\usepackage{calc}

%%%%%%%%%%%%%%%%%%%%%%%%%%%%%%%%%%%%%%%%%%%%%%%%%%%%%%
%%%%%%%%%%%%%% EDIT THIS PART %%%%%%%%%%%%%%%%%%%%%%%%
%%%%%%%%%%%%%%%%%%%%%%%%%%%%%%%%%%%%%%%%%%%%%%%%%%%%%%

\newcommand{\Fach}{01270 Numerische Mathematik I}
\newcommand{\FachKurz}{NMI}
\newcommand{\Name}{Oliver Schwarz}
\newcommand{\Matrikelnummer}{6883389}
\newcommand{\Semester}{SoSe 23}
\newcommand{\Kurseinheit}{5}
\newcommand{\AufgabeNummer}{2}

%%%%%%%%%%%%%%%%%%%%%%%%%%%%%%%%%%%%%%%%%%%%%%%%%%%%%%
%%%%%%%%%%%%%% DO NOT EDIT THIS PART %%%%%%%%%%%%%%%%%
%%%%%%%%%%%%%%%%%%%%%%%%%%%%%%%%%%%%%%%%%%%%%%%%%%%%%%

\newcommand{\Aufgabe}[1]{
  {
  \vspace*{0.5cm}
  \textbf{Aufgabe #1}
  \vspace*{0.2cm}
  }
}

\newcommand{\cond}[0]{
  {
  \textrm{cond}
  }
}

% Beschreibung
% Unterscheidet im Mathe-Mode zwischen
% "Dezimal-komma" und Satzzeichen.
% Muss bei Eingabe durch kein/ein Leerzeichen
% dahinter angegeben werden.
% Literatur Richard Hirsch in: DTK, 1/1994, S. 42 ff.
\mathchardef\CommaOrdinary="013B \mathchardef\CommaPunct="613B
\mathcode`,="8000 % , im Math-Mode aktiv ("8000) machen
{\catcode`\,=\active \gdef
,{\obeyspaces\futurelet\next\CommaCheck}}
\def\CommaCheck{\if\space\next\CommaPunct\else\CommaOrdinary\fi}

\newcommand*\diff{\mathop{}\!\mathrm{\,d}}

\newcommand*\euler{\mathrm{e}}

\newcommand\redarrow{%
        \mathrel{\vcenter{\mathsurround0pt
                \ialign{##\crcr
                        \noalign{\nointerlineskip}$\searrow\vspace{2.5pt}$\crcr 
                        \noalign{\nointerlineskip}$\rightarrow$\crcr
                        \noalign{\nointerlineskip}$\vspace{12.5pt}$\crcr
                }%
        }}%
}
%%%%%%%%%%%%%%%%%%%%%%%%%%%%%%%%%%%%%%%%%%%%%%%%%%%%%%
%%%%%%%%%%%%%% PAGE SETTINGS %%%%%%%%%%%%%%%%%%%%%%%%%
%%%%%%%%%%%%%%%%%%%%%%%%%%%%%%%%%%%%%%%%%%%%%%%%%%%%%%

\setlength{\parindent}{0em}
\topmargin 0cm
\oddsidemargin 0cm
\evensidemargin 0cm

\geometry{%
  left=45.0mm,
  right=15.0mm,
  top=25mm,
  bottom=25mm,
  bindingoffset=0mm,
  headheight=30pt,
  includehead
}

\fancyheadoffset[L]{20mm}
\renewcommand{\headrulewidth}{1pt}

\pagestyle{fancy}

%%%%%%%%%%%%%%%%%%%%%%%%%%%%%%%%%%%%%%%%%%%%%%%%%%%%%%
%%%%%%%%%%%%%% PDF SETTINGS %%%%%%%%%%%%%%%%%%%%%%%%%%
%%%%%%%%%%%%%%%%%%%%%%%%%%%%%%%%%%%%%%%%%%%%%%%%%%%%%%

\hypersetup{
    %pdftitle={\Fach{}: Übungsblatt \Uebungsblatt{}},
    pdftitle={\Fach{}: Kurseinheit \Kurseinheit{} Aufgabe \AufgabeNummer{}},
    pdfauthor={\Name},
    pdfborder={0 0 0}
}

%%%%%%%%%%%%%%%%%%%%%%%%%%%%%%%%%%%%%%%%%%%%%%%%%%%%%%
%%%%%%%%%%%%%% CODE SETTINGS %%%%%%%%%%%%%%%%%%%%%%%%%
%%%%%%%%%%%%%%%%%%%%%%%%%%%%%%%%%%%%%%%%%%%%%%%%%%%%%%

\lstset{ %
language=java,
basicstyle=\footnotesize\tt,
showtabs=false,
tabsize=2,
captionpos=b,
breaklines=true,
extendedchars=true,
showstringspaces=false,
flexiblecolumns=true,
}

%%%%%%%%%%%%%%%%%%%%%%%%%%%%%%%%%%%%%%%%%%%%%%%%%%%%%%
%%%%%%%%%%%%%% DOCUMENT %%%%%%%%%%%%%%%%%%%%%%%%%%%%%%
%%%%%%%%%%%%%%%%%%%%%%%%%%%%%%%%%%%%%%%%%%%%%%%%%%%%%%

\title{Kurseinheit \Kurseinheit{} Aufgabe \AufgabeNummer{}}
\author{\Name{}}

\begin{document}

%\renewcommand{\theequation}{L\Kurseinheit{}.\AufgabeNummer{}.\arabic{equation}}
\renewcommand{\theequation}{L \AufgabeNummer{}.\arabic{equation}}

%%%%%%%%%%%%%%%%%%%%%%%%%%%%%%%%%%%%%%%%%%%%%%%%%%%%%%
%%%%%%%%%%%%%% HEADER %%%%%%%%%%%%%%%%%%%%%%%%%%%%%%%%
%%%%%%%%%%%%%%%%%%%%%%%%%%%%%%%%%%%%%%%%%%%%%%%%%%%%%%

\lhead{\sf \large \Fach{} \\ \small \Name{} - \Matrikelnummer{}}
%\rhead{\sf \Semester{}}
\rhead{\sf \FachKurz{} \quad E \Kurseinheit{}/\AufgabeNummer{}}

%%%%%%%%%%%%%%%%%%%%%%%%%%%%%%%%%%%%%%%%%%%%%%%%%%%%%%
%%%%%%%%%%%%%% START HERE %%%%%%%%%%%%%%%%%%%%%%%%%%%%
%%%%%%%%%%%%%%%%%%%%%%%%%%%%%%%%%%%%%%%%%%%%%%%%%%%%%%

\Aufgabe{\AufgabeNummer{}}

Wir wählen $x \in [a,b]\setminus\{x_0, x_1, x_2\}$ beliebig aber fest und setzen
\begin{equation*}
    K := \frac{f(x) - p_3(x)}{\omega_4(x)}.
\end{equation*}
$K$ ist wohldefiniert, da $\omega_4(x) \neq 0$. Wir führen nun eine Hilfsfunktion $\varphi$ ein:
\begin{equation}
    \varphi:[a,b] \rightarrow \mathbb{R} : t \mapsto \varphi(t) = f(t) - p_3(t) - K\omega_4(t). \label{eq:1}
\end{equation}
Sie besitzt vier verschiedene Nullstellen $a \leq \eta_{0, 0} < \eta_{0, 1} < \cdots < \eta_{0, 3} \leq b$, nämlich die Interpolationsstellen $x_0, x_1, x_2$ und $x$.

Mit dem Satz von \textsc{Rolle} folgt, dass die Ableitung $\varphi'$ in jedem Interval $(\eta_{0, i}, \eta_{0, i+1})$ eine Nullstelle besitzt. Darüber hinaus ist nach Voraussetzung auch $x_1$ eine Nullstelle von $\varphi'$. Das heißt, $\varphi'$ besitzt vier Nullstellen $a \leq \eta_{1, 0} < \eta_{1, 1} < \cdots < \eta_{1, 3} \leq b$.

Wir wiederholen die Argumentation induktiv. Es folgt: Die $k$-te Ableitung $\varphi^{(k)}$ von $\varphi$ besitzt $(5-k)$ Nullstellen $a \leq \eta_{k, 0} < \eta_{k, 1} < \cdots < \eta_{k, 5-k} \leq b$. Insbesondere besitzt $\varphi^{(4)}$ eine Nullstelle $\xi \in [a,b]$. Durch viermalige Differentiation von \ref{eq:1} erhalten wir
\begin{align*}
     &0 = \varphi^{(4)}(\xi) = f^{(4)}(\xi) - \underbrace{p_3^{(4)}(\xi)}_{=0 \textrm{, da } p_3 \in \Pi_3} - K \underbrace{\omega_4^{(4)}(\xi)}_{= 4!} \\
    \Longrightarrow \quad &K = \frac{ f^{(4)}(\xi)}{4!}.
\end{align*}

Also existiert zu jedem $x \in [a,b]\setminus\{x_0, x_1, x_2\}$ ein $\xi \in [a,b]$ mit
\begin{equation*}
  f(x) - p_3(x) =  \frac{ f^{(4)}(\xi)}{4!} (x - x_0)(x - x_1)^2(x - x_2).
\end{equation*}
\end{document}
