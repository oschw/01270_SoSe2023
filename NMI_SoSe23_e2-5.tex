\documentclass[a4paper,12pt]{article}
\usepackage{fancyhdr}
\usepackage[ngerman,german]{babel}
\usepackage{german}
\usepackage[utf8]{inputenc}
\usepackage[active]{srcltx}
\usepackage{algorithm}
\usepackage[noend]{algorithmic}
\usepackage{amsmath}
\usepackage{amssymb}
\usepackage{amsthm}
\usepackage{bbm}
\usepackage{enumerate}
\usepackage{graphicx}
\usepackage{ifthen}
\usepackage{listings}
\usepackage{struktex}
\usepackage{hyperref}
\usepackage[onehalfspacing]{setspace}
\usepackage{geometry}
\usepackage{calc}

%%%%%%%%%%%%%%%%%%%%%%%%%%%%%%%%%%%%%%%%%%%%%%%%%%%%%%
%%%%%%%%%%%%%% EDIT THIS PART %%%%%%%%%%%%%%%%%%%%%%%%
%%%%%%%%%%%%%%%%%%%%%%%%%%%%%%%%%%%%%%%%%%%%%%%%%%%%%%

\newcommand{\Fach}{01270 Numerische Mathematik I}
\newcommand{\FachKurz}{NMI}
\newcommand{\Name}{Oliver Schwarz}
\newcommand{\Matrikelnummer}{6883389}
\newcommand{\Semester}{SoSe 23}
\newcommand{\Kurseinheit}{2}
\newcommand{\AufgabeNummer}{5}

%%%%%%%%%%%%%%%%%%%%%%%%%%%%%%%%%%%%%%%%%%%%%%%%%%%%%%
%%%%%%%%%%%%%% DO NOT EDIT THIS PART %%%%%%%%%%%%%%%%%
%%%%%%%%%%%%%%%%%%%%%%%%%%%%%%%%%%%%%%%%%%%%%%%%%%%%%%

\newcommand{\Aufgabe}[1]{
  {
  \vspace*{0.5cm}
  \textbf{Aufgabe #1}
  \vspace*{0.2cm}
  }
}

%%%%%%%%%%%%%%%%%%%%%%%%%%%%%%%%%%%%%%%%%%%%%%%%%%%%%%
%%%%%%%%%%%%%% PAGE SETTINGS %%%%%%%%%%%%%%%%%%%%%%%%%
%%%%%%%%%%%%%%%%%%%%%%%%%%%%%%%%%%%%%%%%%%%%%%%%%%%%%%

\setlength{\parindent}{0em}
\topmargin 0cm
\oddsidemargin 0cm
\evensidemargin 0cm

\geometry{%
  left=45.0mm,
  right=15.0mm,
  top=25mm,
  bottom=25mm,
  bindingoffset=0mm,
  headheight=30pt,
  includehead
}

\fancyheadoffset[L]{20mm}
\renewcommand{\headrulewidth}{1pt}

\pagestyle{fancy}

%%%%%%%%%%%%%%%%%%%%%%%%%%%%%%%%%%%%%%%%%%%%%%%%%%%%%%
%%%%%%%%%%%%%% PDF SETTINGS %%%%%%%%%%%%%%%%%%%%%%%%%%
%%%%%%%%%%%%%%%%%%%%%%%%%%%%%%%%%%%%%%%%%%%%%%%%%%%%%%

\hypersetup{
    %pdftitle={\Fach{}: Übungsblatt \Uebungsblatt{}},
    pdftitle={\Fach{}: Kurseinheit \Kurseinheit{} Aufgabe \AufgabeNummer{}},
    pdfauthor={\Name},
    pdfborder={0 0 0}
}

%%%%%%%%%%%%%%%%%%%%%%%%%%%%%%%%%%%%%%%%%%%%%%%%%%%%%%
%%%%%%%%%%%%%% CODE SETTINGS %%%%%%%%%%%%%%%%%%%%%%%%%
%%%%%%%%%%%%%%%%%%%%%%%%%%%%%%%%%%%%%%%%%%%%%%%%%%%%%%

\lstset{ %
language=java,
basicstyle=\footnotesize\tt,
showtabs=false,
tabsize=2,
captionpos=b,
breaklines=true,
extendedchars=true,
showstringspaces=false,
flexiblecolumns=true,
}

%%%%%%%%%%%%%%%%%%%%%%%%%%%%%%%%%%%%%%%%%%%%%%%%%%%%%%
%%%%%%%%%%%%%% DOCUMENT %%%%%%%%%%%%%%%%%%%%%%%%%%%%%%
%%%%%%%%%%%%%%%%%%%%%%%%%%%%%%%%%%%%%%%%%%%%%%%%%%%%%%

\title{Kurseinheit \Kurseinheit{} Aufgabe \AufgabeNummer{}}
\author{\Name{}}

\begin{document}

%\renewcommand{\theequation}{L\Kurseinheit{}.\AufgabeNummer{}.\arabic{equation}}
\renewcommand{\theequation}{L \AufgabeNummer{}.\arabic{equation}}

%%%%%%%%%%%%%%%%%%%%%%%%%%%%%%%%%%%%%%%%%%%%%%%%%%%%%%
%%%%%%%%%%%%%% HEADER %%%%%%%%%%%%%%%%%%%%%%%%%%%%%%%%
%%%%%%%%%%%%%%%%%%%%%%%%%%%%%%%%%%%%%%%%%%%%%%%%%%%%%%

\lhead{\sf \large \Fach{} \\ \small \Name{} - \Matrikelnummer{}}
%\rhead{\sf \Semester{}}
\rhead{\sf \FachKurz{} \quad E \Kurseinheit{}/\AufgabeNummer{}}

%%%%%%%%%%%%%%%%%%%%%%%%%%%%%%%%%%%%%%%%%%%%%%%%%%%%%%
%%%%%%%%%%%%%% START HERE %%%%%%%%%%%%%%%%%%%%%%%%%%%%
%%%%%%%%%%%%%%%%%%%%%%%%%%%%%%%%%%%%%%%%%%%%%%%%%%%%%%

\Aufgabe{\AufgabeNummer{}}

Sei $C[0,1]$ der Raum der reellwertigen, stetigen Funktionen über $[0, 1]$.

\begin{enumerate}[a)]
    \item %a
Nach Definition 2.1.1 muss eine Norm drei Voraussetzungen erfüllen: Sie ist positiv definit, absolut homogen und es gilt die Dreiecksungleichung $||\pmb{x} + \pmb{y}|| \leq ||\pmb{x}|| + ||\pmb{y}|| \; \forall \, \pmb{x}, \pmb{y} \in \mathbb{V}$.
        \begin{enumerate}[1)]
            \item 
$||f||_{\infty} := \max\limits_{x \in [0,1]} |f(x)| \quad \forall \, f \in C[0,1]$
\begin{align*}
N_1: \qquad  &\; |f(x)| \geq 0 \, \Rightarrow \, \max\limits_{x \in [0,1]} |f(x)| \geq 0\\
 \Longrightarrow &\; ||f||_{\infty} > 0 \quad \forall \, f \in C[0,1] \setminus 0 \; \mathrm{und} \\
 &\; ||f||_{\infty} = 0 \; \Leftrightarrow \; f = 0 \\
N_2: \qquad  &\; ||\alpha f||_{\infty} = \max\limits_{x \in [0,1]} |\alpha f(x)| = |\alpha| \max\limits_{x \in [0,1]} |f(x)| \\
 \Longrightarrow &\; ||\alpha f||_{\infty} = |\alpha| \, ||f||_{\infty} \quad \forall \, f \in C[0,1], \, \alpha \in \mathbb{K} \\
N_3: \qquad  &\; ||f + g||_{\infty} = \max\limits_{x \in [0,1]} |f(x) + g(x)| \\
&\; \leq \max\limits_{x \in [0,1]} (|f(x)| + |g(x)|) = \max\limits_{x \in [0,1]} (|f(x)|) + \max\limits_{x \in [0,1]} (|g(x)|) \\
 \Longrightarrow &\; ||f + g||_{\infty} \leq ||f||_{\infty} + ||g||_{\infty}  \quad \forall \, f, g \in C[0,1].
\end{align*}

            \item
$||f||_{1} := \int_{0}^{1} |f(x)| \, dx \quad \forall \, f \in C[0,1]$

Äquivalent zu Teilaufgabe 1 sind $N_1$ ((positive) Definitheit) und $N_2$ (absolute Homogenität) erfüllt. Weiterhin gilt:
\begin{align*}
N_3: \qquad  \; ||f + g||_{1} &= \int\limits_{0}^{1} |f(x) + g(x)| \, dx \leq \int\limits_{0}^{1} |f(x)| + |g(x)| \, dx \\
&= \int\limits_{0}^{1} |f(x)| \, dx + \int\limits_{0}^{1} |g(x)| \, dx \\
 \Longrightarrow \; ||f + g||_{1} &\leq ||f||_{1} + ||g||_{1}  \quad \forall \, f, g \in C[0,1].
\end{align*}
        \end{enumerate}

    \item %b
Laut Bemerkung 2.1.8 gilt der Normenäquivalenzsatz für $f \in C[0,1]$ i.Allg. nicht.

Wir betrachten die Folge von Funktionen $(f_k)_{k \in \mathbb{N}}$ mit

\begin{equation*}
    f_k(x) :=\begin{cases}
			\cos{k \pi x} & \textrm{für } x \in [0, \frac{1}{2k}],\\
            0 & \textrm{für } x \in (\frac{1}{2k}, 1].
		 \end{cases}
\end{equation*}

Alle Funktionen $(f_k)_{k \in \mathbb{N}}$ sind stetig im Intervall $[0,1]$ mit $f_k(x) = 1$ für $x = 0$ und $f_k(x) = 0$ für $x = (2k)^{-1}$. Also ist $(f_k)_{k \in \mathbb{N}} \in C[0,1]$ für alle $k \in \mathbb{N}$.

Es ist
\begin{equation*}
    ||f_k||_1 = \int\limits_{0}^{1} |f_k(x)| \, dx = \int\limits_{0}^{\frac{1}{2k}} |\cos{k \pi x}| \, dx = \frac{\sin{k \pi x}}{k \pi} 
    \Bigg|_{0}^{\frac{1}{2k}} = \frac{1}{k \pi}.
\end{equation*}
Also geht $||f_k||_1$ gegen $0$ für $k \rightarrow \infty$.

Hingegen ist $||f_k||_{\infty} := \max\limits_{x \in [0,1]} |f_k(x)| = \cos{0} = 1$ für alle $k \in \mathbb{N}$.

Also gibt es für die Konstanten $A, B > 0$, $A, B \in \mathbb{R}$ immer ein $\hat{k} \in \mathbb{N}$, ab dem die Ungleichung
\begin{equation*}
    A ||f_k||_{\infty} \leq ||f_k||_1 \leq B ||f_k||_{\infty}
\end{equation*}
nicht mehr erfüllt ist. Die Normen sind demnach nicht äquivalent.

\end{enumerate}

\end{document}
