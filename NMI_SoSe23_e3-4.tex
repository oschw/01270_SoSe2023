\documentclass[a4paper,12pt]{article}
\usepackage{fancyhdr}
\usepackage[ngerman,german]{babel}
\usepackage{german}
\usepackage[utf8]{inputenc}
\usepackage[active]{srcltx}
\usepackage{algorithm}
\usepackage[noend]{algorithmic}
\usepackage{amsmath}
\usepackage{amssymb}
\usepackage{amsthm}
\usepackage{bbm}
\usepackage{enumerate}
\usepackage{graphicx}
\usepackage{ifthen}
\usepackage{listings}
\usepackage{struktex}
\usepackage{hyperref}
\usepackage[onehalfspacing]{setspace}
\usepackage{geometry}
\usepackage{calc}
\usepackage{arydshln}

%%%%%%%%%%%%%%%%%%%%%%%%%%%%%%%%%%%%%%%%%%%%%%%%%%%%%%
%%%%%%%%%%%%%% EDIT THIS PART %%%%%%%%%%%%%%%%%%%%%%%%
%%%%%%%%%%%%%%%%%%%%%%%%%%%%%%%%%%%%%%%%%%%%%%%%%%%%%%

\newcommand{\Fach}{01270 Numerische Mathematik I}
\newcommand{\FachKurz}{NMI}
\newcommand{\Name}{Oliver Schwarz}
\newcommand{\Matrikelnummer}{6883389}
\newcommand{\Semester}{SoSe 23}
\newcommand{\Kurseinheit}{3}
\newcommand{\AufgabeNummer}{4}

%%%%%%%%%%%%%%%%%%%%%%%%%%%%%%%%%%%%%%%%%%%%%%%%%%%%%%
%%%%%%%%%%%%%% DO NOT EDIT THIS PART %%%%%%%%%%%%%%%%%
%%%%%%%%%%%%%%%%%%%%%%%%%%%%%%%%%%%%%%%%%%%%%%%%%%%%%%

\newcommand{\Aufgabe}[1]{
  {
  \vspace*{0.5cm}
  \textbf{Aufgabe #1}
  \vspace*{0.2cm}
  }
}

\newcommand{\cond}[0]{
  {
  \textrm{cond}
  }
}

% Beschreibung
% Unterscheidet im Mathe-Mode zwischen
% "Dezimal-komma" und Satzzeichen.
% Muss bei Eingabe durch kein/ein Leerzeichen
% dahinter angegeben werden.
% Literatur Richard Hirsch in: DTK, 1/1994, S. 42 ff.
\mathchardef\CommaOrdinary="013B \mathchardef\CommaPunct="613B
\mathcode`,="8000 % , im Math-Mode aktiv ("8000) machen
{\catcode`\,=\active \gdef
,{\obeyspaces\futurelet\next\CommaCheck}}
\def\CommaCheck{\if\space\next\CommaPunct\else\CommaOrdinary\fi}

%%%%%%%%%%%%%%%%%%%%%%%%%%%%%%%%%%%%%%%%%%%%%%%%%%%%%%
%%%%%%%%%%%%%% PAGE SETTINGS %%%%%%%%%%%%%%%%%%%%%%%%%
%%%%%%%%%%%%%%%%%%%%%%%%%%%%%%%%%%%%%%%%%%%%%%%%%%%%%%

\setlength{\parindent}{0em}
\topmargin 0cm
\oddsidemargin 0cm
\evensidemargin 0cm

\geometry{%
  left=45.0mm,
  right=15.0mm,
  top=25mm,
  bottom=25mm,
  bindingoffset=0mm,
  headheight=30pt,
  includehead
}

\fancyheadoffset[L]{20mm}
\renewcommand{\headrulewidth}{1pt}

\pagestyle{fancy}

%%%%%%%%%%%%%%%%%%%%%%%%%%%%%%%%%%%%%%%%%%%%%%%%%%%%%%
%%%%%%%%%%%%%% PDF SETTINGS %%%%%%%%%%%%%%%%%%%%%%%%%%
%%%%%%%%%%%%%%%%%%%%%%%%%%%%%%%%%%%%%%%%%%%%%%%%%%%%%%

\hypersetup{
    %pdftitle={\Fach{}: Übungsblatt \Uebungsblatt{}},
    pdftitle={\Fach{}: Kurseinheit \Kurseinheit{} Aufgabe \AufgabeNummer{}},
    pdfauthor={\Name},
    pdfborder={0 0 0}
}

%%%%%%%%%%%%%%%%%%%%%%%%%%%%%%%%%%%%%%%%%%%%%%%%%%%%%%
%%%%%%%%%%%%%% CODE SETTINGS %%%%%%%%%%%%%%%%%%%%%%%%%
%%%%%%%%%%%%%%%%%%%%%%%%%%%%%%%%%%%%%%%%%%%%%%%%%%%%%%

\lstset{ %
language=java,
basicstyle=\footnotesize\tt,
showtabs=false,
tabsize=2,
captionpos=b,
breaklines=true,
extendedchars=true,
showstringspaces=false,
flexiblecolumns=true,
}

%%%%%%%%%%%%%%%%%%%%%%%%%%%%%%%%%%%%%%%%%%%%%%%%%%%%%%
%%%%%%%%%%%%%% DOCUMENT %%%%%%%%%%%%%%%%%%%%%%%%%%%%%%
%%%%%%%%%%%%%%%%%%%%%%%%%%%%%%%%%%%%%%%%%%%%%%%%%%%%%%

\title{Kurseinheit \Kurseinheit{} Aufgabe \AufgabeNummer{}}
\author{\Name{}}

\begin{document}

%\renewcommand{\theequation}{L\Kurseinheit{}.\AufgabeNummer{}.\arabic{equation}}
\renewcommand{\theequation}{L \AufgabeNummer{}.\arabic{equation}}

%%%%%%%%%%%%%%%%%%%%%%%%%%%%%%%%%%%%%%%%%%%%%%%%%%%%%%
%%%%%%%%%%%%%% HEADER %%%%%%%%%%%%%%%%%%%%%%%%%%%%%%%%
%%%%%%%%%%%%%%%%%%%%%%%%%%%%%%%%%%%%%%%%%%%%%%%%%%%%%%

\lhead{\sf \large \Fach{} \\ \small \Name{} - \Matrikelnummer{}}
%\rhead{\sf \Semester{}}
\rhead{\sf \FachKurz{} \quad E \Kurseinheit{}/\AufgabeNummer{}}

%%%%%%%%%%%%%%%%%%%%%%%%%%%%%%%%%%%%%%%%%%%%%%%%%%%%%%
%%%%%%%%%%%%%% START HERE %%%%%%%%%%%%%%%%%%%%%%%%%%%%
%%%%%%%%%%%%%%%%%%%%%%%%%%%%%%%%%%%%%%%%%%%%%%%%%%%%%%

\Aufgabe{\AufgabeNummer{}}

\begin{enumerate}[a)]
    \item \label{aufgabe_a} %a)
Wir wenden das \textsc{Gauss}sche Eliminationsverfahren an.    
\begin{align*}
    \begin{array}[t]{ r  r  r : r r r r}
x_1 & x_2 & x_3 &  & \quad & &\\\cline{1-4}
4 & 98 & 9998 & 10100 & & \cdot \frac{1}{2} & \cdot \frac{1}{4} \\
2 & 9 & 103 & 114 & & &\\
1 & 1 & 1 & 3 & & &\\\cline{1-4}
4 & 98 & 9998 & 10100 & &  & \\
0 & -40 & -4896 & -4936 & & \cdot \frac{47}{80} &\\
0 & -23,5 & -2498,5 & -2522 & & &\\\cline{1-4}
4 & 98 & 9998 & 10100 & &  & \\
0 & -40 & -4896 & -4936 & &  &\\
0 & 0 & 377,9 & 377,9 & & \cdot \frac{10}{3779} &\\%\cline{1-4}
%1 & 0 & 0 & 1 & &  & \\
%0 & 1 & 0 & 1 & &  &\\
%0 & 0 & 1 & 1 & & \cdot  &\\\cline{1-4}
\end{array}
\end{align*}
Rückwärtsersetzung ergibt die Lösung $(1, 1, 1)^T$.

\item %b)
Wir wandeln die Matrix in 3-stellige Gleitkomma-Arithmetrik um.
\begin{align*}
    \begin{array}[t]{ r  r  r : r r r r}
x_1 & x_2 & x_3 &  & \quad & &\\\cline{1-4}
0.400 \cdot 10^1 & 0.980 \cdot 10^2  & 0.100 \cdot 10^5 & 0.101 \cdot 10^5 & & &  \\
0.200 \cdot 10^1 & 0.900 \cdot 10^1  & 0.103 \cdot 10^3 & 0.114 \cdot 10^3 & & &\\
0.100 \cdot 10^1 & 0.100 \cdot 10^1 & 0.100 \cdot 10^1 & 0.300 \cdot 10^1 & & &\\
\end{array}
\end{align*}

Wir gehen nun wie in Teilaufgabe \ref{aufgabe_a} vor.
\begin{align*}
    \begin{array}[t]{ r  r  r : r r r r}
x_1 & x_2 & x_3 &  & \quad & &\\\cline{1-4}
0.400 \cdot 10^1 & 0.980 \cdot 10^2  & 0.100 \cdot 10^5 & 0.101 \cdot 10^5 & & \cdot (0.500 \cdot 10^0) & \cdot (0.250 \cdot 10^0) \\
0.200 \cdot 10^1 & 0.900 \cdot 10^1  & 0.103 \cdot 10^3 & 0.114 \cdot 10^3 & & &\\
0.100 \cdot 10^1 & 0.100 \cdot 10^1 & 0.100 \cdot 10^1 & 0.300 \cdot 10^1 & & &\\\cline{1-4}
0.400 \cdot 10^1 & 0.980 \cdot 10^2  & 0.100 \cdot 10^5 & 0.101 \cdot 10^5 & &  &  \\
0.000 \cdot 10^0 & -0.400 \cdot 10^2  & -0.490 \cdot 10^4 & - 0.494 \cdot 10^4 & & \cdot (0.588 \cdot 10^0) &\\
0.000 \cdot 10^0 & -0.235 \cdot 10^2 & -0.250 \cdot 10^4 & - 0.253 \cdot 10^4 & & &\\\cline{1-4}
0.400 \cdot 10^1 & 0.980 \cdot 10^2  & 0.100 \cdot 10^5 & 0.101 \cdot 10^5 & &  &  \\
0.000 \cdot 10^0 & -0.400 \cdot 10^2  & -0.490 \cdot 10^4 & - 0.494 \cdot 10^4 & & &\\
0.000 \cdot 10^0 & 0.000 \cdot 10^0 & -0.380 \cdot 10^3 & - 0.370 \cdot 10^3 & & &\\%\cline{1-4}
\end{array}
\end{align*}

Rückwärtsersetzung ergibt die Lösung $(- 0.143 \cdot 10^2, 0.425 \cdot 10^1, 0.974 \cdot 10^0)^T$.

\item %c)
Wir führen die skalierte Spalten-Pivot-Suche durch.
\begin{align*}
    \begin{array}[t]{ r  r  r : r r r r}
x_1 & x_2 & x_3 &  & \quad & &\\\cline{1-4}
0.400 \cdot 10^1 & 0.980 \cdot 10^2  & 0.100 \cdot 10^5 & 0.101 \cdot 10^5 & &  & \\
0.200 \cdot 10^1 & 0.900 \cdot 10^1  & 0.103 \cdot 10^3 & 0.114 \cdot 10^3 & & &\\
0.100 \cdot 10^1 & 0.100 \cdot 10^1 & 0.100 \cdot 10^1 & 0.300 \cdot 10^1 & & \textrm{I} \leftrightarrows \textrm{III} &\\\cline{1-4}
0.100 \cdot 10^1 & 0.100 \cdot 10^1 & 0.100 \cdot 10^1 & 0.300 \cdot 10^1 & & \cdot (0.200 \cdot 10^1) & \cdot (0.400 \cdot 10^1)\\
0.200 \cdot 10^1 & 0.900 \cdot 10^1  & 0.103 \cdot 10^3 & 0.114 \cdot 10^3 & & &\\
0.400 \cdot 10^1 & 0.980 \cdot 10^2  & 0.100 \cdot 10^5 & 0.101 \cdot 10^5 & &  & \\\cline{1-4}
0.100 \cdot 10^1 & 0.100 \cdot 10^1 & 0.100 \cdot 10^1 & 0.300 \cdot 10^1 & & & \\
0.000 \cdot 10^0 & 0.700 \cdot 10^1  & 0.101 \cdot 10^3 & 0.108 \cdot 10^3 & & \cdot \frac{(0.940 \cdot 10^2)}{(0.700 \cdot 10^1)} &\\
0.000 \cdot 10^0 & 0.940 \cdot 10^2  & 0.100 \cdot 10^5 & 0.101 \cdot 10^5 & &  & \\\cline{1-4}
0.100 \cdot 10^1 & 0.100 \cdot 10^1 & 0.100 \cdot 10^1 & 0.300 \cdot 10^1 & & & \\
0.000 \cdot 10^0 & 0.700 \cdot 10^1  & 0.101 \cdot 10^3 & 0.108 \cdot 10^3 & & &\\
0.000 \cdot 10^0 & 0.000 \cdot 10^0  & 0.865 \cdot 10^4 & 0.865 \cdot 10^4 & &  & \\
\end{array}
\end{align*}

Rückwärtsersetzung ergibt die Lösung $(0.100 \cdot 10^1, 0.100 \cdot 10^1, 0.100 \cdot 10^1)^T$.
\end{enumerate}
\end{document}
