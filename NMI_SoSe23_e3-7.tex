\documentclass[a4paper,12pt]{article}
\usepackage{fancyhdr}
\usepackage[ngerman,german]{babel}
\usepackage{german}
\usepackage[utf8]{inputenc}
\usepackage[active]{srcltx}
\usepackage{algorithm}
\usepackage[noend]{algorithmic}
\usepackage{amsmath}
\usepackage{amssymb}
\usepackage{amsthm}
\usepackage{bbm}
\usepackage{enumerate}
\usepackage{graphicx}
\usepackage{ifthen}
\usepackage{listings}
\usepackage{struktex}
\usepackage{hyperref}
\usepackage[onehalfspacing]{setspace}
\usepackage{geometry}
\usepackage{calc}

%%%%%%%%%%%%%%%%%%%%%%%%%%%%%%%%%%%%%%%%%%%%%%%%%%%%%%
%%%%%%%%%%%%%% EDIT THIS PART %%%%%%%%%%%%%%%%%%%%%%%%
%%%%%%%%%%%%%%%%%%%%%%%%%%%%%%%%%%%%%%%%%%%%%%%%%%%%%%

\newcommand{\Fach}{01270 Numerische Mathematik I}
\newcommand{\FachKurz}{NMI}
\newcommand{\Name}{Oliver Schwarz}
\newcommand{\Matrikelnummer}{6883389}
\newcommand{\Semester}{SoSe 23}
\newcommand{\Kurseinheit}{3}
\newcommand{\AufgabeNummer}{7}

%%%%%%%%%%%%%%%%%%%%%%%%%%%%%%%%%%%%%%%%%%%%%%%%%%%%%%
%%%%%%%%%%%%%% DO NOT EDIT THIS PART %%%%%%%%%%%%%%%%%
%%%%%%%%%%%%%%%%%%%%%%%%%%%%%%%%%%%%%%%%%%%%%%%%%%%%%%

\newcommand{\Aufgabe}[1]{
  {
  \vspace*{0.5cm}
  \textbf{Aufgabe #1}
  \vspace*{0.2cm}
  }
}

\newcommand{\cond}[0]{
  {
  \textrm{cond}
  }
}

% Beschreibung
% Unterscheidet im Mathe-Mode zwischen
% "Dezimal-komma" und Satzzeichen.
% Muss bei Eingabe durch kein/ein Leerzeichen
% dahinter angegeben werden.
% Literatur Richard Hirsch in: DTK, 1/1994, S. 42 ff.
\mathchardef\CommaOrdinary="013B \mathchardef\CommaPunct="613B
\mathcode`,="8000 % , im Math-Mode aktiv ("8000) machen
{\catcode`\,=\active \gdef
,{\obeyspaces\futurelet\next\CommaCheck}}
\def\CommaCheck{\if\space\next\CommaPunct\else\CommaOrdinary\fi}

%%%%%%%%%%%%%%%%%%%%%%%%%%%%%%%%%%%%%%%%%%%%%%%%%%%%%%
%%%%%%%%%%%%%% PAGE SETTINGS %%%%%%%%%%%%%%%%%%%%%%%%%
%%%%%%%%%%%%%%%%%%%%%%%%%%%%%%%%%%%%%%%%%%%%%%%%%%%%%%

\setlength{\parindent}{0em}
\topmargin 0cm
\oddsidemargin 0cm
\evensidemargin 0cm

\geometry{%
  left=45.0mm,
  right=15.0mm,
  top=25mm,
  bottom=25mm,
  bindingoffset=0mm,
  headheight=30pt,
  includehead
}

\fancyheadoffset[L]{20mm}
\renewcommand{\headrulewidth}{1pt}

\pagestyle{fancy}

%%%%%%%%%%%%%%%%%%%%%%%%%%%%%%%%%%%%%%%%%%%%%%%%%%%%%%
%%%%%%%%%%%%%% PDF SETTINGS %%%%%%%%%%%%%%%%%%%%%%%%%%
%%%%%%%%%%%%%%%%%%%%%%%%%%%%%%%%%%%%%%%%%%%%%%%%%%%%%%

\hypersetup{
    %pdftitle={\Fach{}: Übungsblatt \Uebungsblatt{}},
    pdftitle={\Fach{}: Kurseinheit \Kurseinheit{} Aufgabe \AufgabeNummer{}},
    pdfauthor={\Name},
    pdfborder={0 0 0}
}

%%%%%%%%%%%%%%%%%%%%%%%%%%%%%%%%%%%%%%%%%%%%%%%%%%%%%%
%%%%%%%%%%%%%% CODE SETTINGS %%%%%%%%%%%%%%%%%%%%%%%%%
%%%%%%%%%%%%%%%%%%%%%%%%%%%%%%%%%%%%%%%%%%%%%%%%%%%%%%

\lstset{ %
language=java,
basicstyle=\footnotesize\tt,
showtabs=false,
tabsize=2,
captionpos=b,
breaklines=true,
extendedchars=true,
showstringspaces=false,
flexiblecolumns=true,
}

%%%%%%%%%%%%%%%%%%%%%%%%%%%%%%%%%%%%%%%%%%%%%%%%%%%%%%
%%%%%%%%%%%%%% DOCUMENT %%%%%%%%%%%%%%%%%%%%%%%%%%%%%%
%%%%%%%%%%%%%%%%%%%%%%%%%%%%%%%%%%%%%%%%%%%%%%%%%%%%%%

\title{Kurseinheit \Kurseinheit{} Aufgabe \AufgabeNummer{}}
\author{\Name{}}

\begin{document}

%\renewcommand{\theequation}{L\Kurseinheit{}.\AufgabeNummer{}.\arabic{equation}}
\renewcommand{\theequation}{L \AufgabeNummer{}.\arabic{equation}}

%%%%%%%%%%%%%%%%%%%%%%%%%%%%%%%%%%%%%%%%%%%%%%%%%%%%%%
%%%%%%%%%%%%%% HEADER %%%%%%%%%%%%%%%%%%%%%%%%%%%%%%%%
%%%%%%%%%%%%%%%%%%%%%%%%%%%%%%%%%%%%%%%%%%%%%%%%%%%%%%

\lhead{\sf \large \Fach{} \\ \small \Name{} - \Matrikelnummer{}}
%\rhead{\sf \Semester{}}
\rhead{\sf \FachKurz{} \quad E \Kurseinheit{}/\AufgabeNummer{}}

%%%%%%%%%%%%%%%%%%%%%%%%%%%%%%%%%%%%%%%%%%%%%%%%%%%%%%
%%%%%%%%%%%%%% START HERE %%%%%%%%%%%%%%%%%%%%%%%%%%%%
%%%%%%%%%%%%%%%%%%%%%%%%%%%%%%%%%%%%%%%%%%%%%%%%%%%%%%

\Aufgabe{\AufgabeNummer{}}
\begin{enumerate}[a)]
    \item %a
Es ist $\pmb{A} = \pmb{LL}^H$: $\pmb{L}  \in \mathbb{C}^{n,n}$ ist eine untere Bidiagonalmatrix:
\begin{align*}
   \pmb{L} :=
\begin{pmatrix}
\lambda_1 & &  &  &\\
l_2 & \lambda_2 &  &  &\\
 & \ddots & \ddots &  &\\
 & & l_{n-1} & \lambda_{n-1} & \\
& & & l_n & \lambda_n
\end{pmatrix}.
\end{align*}
Und damit
\begin{align*}
   \pmb{LL}^H &:=
\begin{pmatrix}
\lambda_1 & &  &  &\\
l_2 & \lambda_2 &  &  &\\
 & \ddots & \ddots &  &\\
 & & l_{n-1} & \lambda_{n-1} & \\
& & & l_n & \lambda_n
\end{pmatrix}
\begin{pmatrix}
\lambda_1 & \bar{l}_2 &  &  &\\
 & \lambda_2 &  \bar{l}_3&  &\\
 &  & \ddots & \ddots &\\
 & &  & \lambda_{n-1} & \bar{l}_n\\
& & &  & \lambda_n
\end{pmatrix} \\
&=
\begin{pmatrix}
\lambda_1^2 & \lambda_1 \bar{l}_2 &  &  &\\
 \lambda_1 l_2& \lambda_2^2+l_2\bar{l}_2 &  \lambda_2\bar{l}_3&  &\\
 &  \ddots & \ddots & \ddots &\\
 & & \lambda_{n-2}l_{n-1} & \lambda_{n-1}^2+l_{n-1}\bar{l}_{n-1} & \lambda_{n-1}\bar{l}_n\\
& & & \lambda_{n-1}l_n & \lambda_n^2 + l_n \bar{l}_n
\end{pmatrix}
= \pmb{A}.
\end{align*}
Wir betrachten $\pmb{A}$: $$a_1 = \lambda_1^2, \quad a_i = \lambda_i^2 + l_i \bar{l}_i, \quad b_i = \lambda_{i-1} l_i, \quad i = 2, \dots, n.$$

\textbf{Algorithmus}
\begin{align*}
\text{S1.} \quad & m_1 := \sqrt{a_1} \\
\text{S2.} \quad & \textsc{for} \quad i = 2, \dots , n \quad \textsc{do} \\
                  & \qquad l_i := b_i / \lambda_{i-1} \\
                  & \qquad \lambda_i := \sqrt{a_i - l_i \bar{l}_i} \\
                  & \textsc{end do}
\end{align*}
\textit{Aufwand:} Es werden $n-1$ Subtraktionen, Multiplikationen und Divisionen durchgeführt und $n$ Quadratwurzeln gezogen. In Aufgabe 3.4.11 wurde der Aufwand für die \textsc{Cholesky}-Faktorisierung bestimmt. Dieser ist im Vergleich um den Faktor $n^2$ höher.

\item %b
Es ist $\pmb{A} = \pmb{CDC}^H$:

\begin{align*}
    \pmb{CDC}^H &:=
\begin{pmatrix}
1 & & & & \\
c_2 & 1 & & \\
& \ddots & \ddots & & \\
& & c_{n-1} & 1 & \\
& & & c_n & 1
\end{pmatrix}
\begin{pmatrix}
d_1 & & & & \\
 & d_2 & & \\
&  & \ddots & & \\
& &  & d_{n-1} & \\
& & &  & d_n
\end{pmatrix}
\begin{pmatrix}
1 & \bar{c}_2 & & & \\
 & 1 & \bar{c}_3 & \\
&  & \ddots & \ddots& \\
& & & 1 & \bar{c}_n\\
& & &  & 1
\end{pmatrix} \\
&=
\begin{pmatrix}
d_1 & d_1\bar{c}_2 & & & \\
 d_1 c_2 & d_1 c_2 \bar{c}_2 + d_2& d_2\bar{c}_3 & \\
& \ddots & \ddots & \ddots& \\
& & d_{n-2} c_{n-1}& d_{n-1} + c_{n-1} \bar{c}_{n-1} & d_{n-1}\bar{c}_n\\
& & & d_{n-1}c_n & d_{n-1} c_n \bar{c}_n+d_n
\end{pmatrix} = \pmb{A}.
\end{align*}
Wir betrachten $\pmb{A}$: $$a_1 = d_1, \quad a_i = d_i + c_i \bar{c}_i d_{i-1}, \quad b_i = c_i d_{i-1}, \quad i = 2, \dots, n.$$

\textbf{Algorithmus}
\begin{align*}
\text{S1.} \quad & d_1 := a_1 \\
\text{S2.} \quad & \textsc{for} \quad i = 2, \dots , n \quad \textsc{do} \\
                  & \qquad c_i := b_i / d_{i-1} \\
                  & \qquad d_i := a_i - c_i \bar{c}_i d_{i-1}\\
                  & \textsc{end do}
\end{align*}
\textit{Aufwand:} Es werden $n-1$ Subtraktionen und Divisionen durchgeführt sowie $2(n-1)$ Multiplikationen. In Aufgabe 3.4.11 wurde der Aufwand für die \textsc{Cholesky}-Faktorisierung bestimmt. Dieser ist im Vergleich um den Faktor $n^2$ höher.

\end{enumerate}
\end{document}
