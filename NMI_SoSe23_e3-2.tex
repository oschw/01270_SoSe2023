\documentclass[a4paper,12pt]{article}
\usepackage{fancyhdr}
\usepackage[ngerman,german]{babel}
\usepackage{german}
\usepackage[utf8]{inputenc}
\usepackage[active]{srcltx}
\usepackage{algorithm}
\usepackage[noend]{algorithmic}
\usepackage{amsmath}
\usepackage{amssymb}
\usepackage{amsthm}
\usepackage{bbm}
\usepackage{enumerate}
\usepackage{graphicx}
\usepackage{ifthen}
\usepackage{listings}
\usepackage{struktex}
\usepackage{hyperref}
\usepackage[onehalfspacing]{setspace}
\usepackage{geometry}
\usepackage{calc}

%%%%%%%%%%%%%%%%%%%%%%%%%%%%%%%%%%%%%%%%%%%%%%%%%%%%%%
%%%%%%%%%%%%%% EDIT THIS PART %%%%%%%%%%%%%%%%%%%%%%%%
%%%%%%%%%%%%%%%%%%%%%%%%%%%%%%%%%%%%%%%%%%%%%%%%%%%%%%

\newcommand{\Fach}{01270 Numerische Mathematik I}
\newcommand{\FachKurz}{NMI}
\newcommand{\Name}{Oliver Schwarz}
\newcommand{\Matrikelnummer}{6883389}
\newcommand{\Semester}{SoSe 23}
\newcommand{\Kurseinheit}{3}
\newcommand{\AufgabeNummer}{2}

%%%%%%%%%%%%%%%%%%%%%%%%%%%%%%%%%%%%%%%%%%%%%%%%%%%%%%
%%%%%%%%%%%%%% DO NOT EDIT THIS PART %%%%%%%%%%%%%%%%%
%%%%%%%%%%%%%%%%%%%%%%%%%%%%%%%%%%%%%%%%%%%%%%%%%%%%%%

\newcommand{\Aufgabe}[1]{
  {
  \vspace*{0.5cm}
  \textbf{Aufgabe #1}
  \vspace*{0.2cm}
  }
}

\newcommand{\cond}[0]{
  {
  \textrm{cond}
  }
}

% Beschreibung
% Unterscheidet im Mathe-Mode zwischen
% "Dezimal-komma" und Satzzeichen.
% Muss bei Eingabe durch kein/ein Leerzeichen
% dahinter angegeben werden.
% Literatur Richard Hirsch in: DTK, 1/1994, S. 42 ff.
\mathchardef\CommaOrdinary="013B \mathchardef\CommaPunct="613B
\mathcode`,="8000 % , im Math-Mode aktiv ("8000) machen
{\catcode`\,=\active \gdef
,{\obeyspaces\futurelet\next\CommaCheck}}
\def\CommaCheck{\if\space\next\CommaPunct\else\CommaOrdinary\fi}

%%%%%%%%%%%%%%%%%%%%%%%%%%%%%%%%%%%%%%%%%%%%%%%%%%%%%%
%%%%%%%%%%%%%% PAGE SETTINGS %%%%%%%%%%%%%%%%%%%%%%%%%
%%%%%%%%%%%%%%%%%%%%%%%%%%%%%%%%%%%%%%%%%%%%%%%%%%%%%%

\setlength{\parindent}{0em}
\topmargin 0cm
\oddsidemargin 0cm
\evensidemargin 0cm

\geometry{%
  left=45.0mm,
  right=15.0mm,
  top=25mm,
  bottom=25mm,
  bindingoffset=0mm,
  headheight=30pt,
  includehead
}

\fancyheadoffset[L]{20mm}
\renewcommand{\headrulewidth}{1pt}

\pagestyle{fancy}

%%%%%%%%%%%%%%%%%%%%%%%%%%%%%%%%%%%%%%%%%%%%%%%%%%%%%%
%%%%%%%%%%%%%% PDF SETTINGS %%%%%%%%%%%%%%%%%%%%%%%%%%
%%%%%%%%%%%%%%%%%%%%%%%%%%%%%%%%%%%%%%%%%%%%%%%%%%%%%%

\hypersetup{
    %pdftitle={\Fach{}: Übungsblatt \Uebungsblatt{}},
    pdftitle={\Fach{}: Kurseinheit \Kurseinheit{} Aufgabe \AufgabeNummer{}},
    pdfauthor={\Name},
    pdfborder={0 0 0}
}

%%%%%%%%%%%%%%%%%%%%%%%%%%%%%%%%%%%%%%%%%%%%%%%%%%%%%%
%%%%%%%%%%%%%% CODE SETTINGS %%%%%%%%%%%%%%%%%%%%%%%%%
%%%%%%%%%%%%%%%%%%%%%%%%%%%%%%%%%%%%%%%%%%%%%%%%%%%%%%

\lstset{ %
language=java,
basicstyle=\footnotesize\tt,
showtabs=false,
tabsize=2,
captionpos=b,
breaklines=true,
extendedchars=true,
showstringspaces=false,
flexiblecolumns=true,
}

%%%%%%%%%%%%%%%%%%%%%%%%%%%%%%%%%%%%%%%%%%%%%%%%%%%%%%
%%%%%%%%%%%%%% DOCUMENT %%%%%%%%%%%%%%%%%%%%%%%%%%%%%%
%%%%%%%%%%%%%%%%%%%%%%%%%%%%%%%%%%%%%%%%%%%%%%%%%%%%%%

\title{Kurseinheit \Kurseinheit{} Aufgabe \AufgabeNummer{}}
\author{\Name{}}

\begin{document}

%\renewcommand{\theequation}{L\Kurseinheit{}.\AufgabeNummer{}.\arabic{equation}}
\renewcommand{\theequation}{L \AufgabeNummer{}.\arabic{equation}}

%%%%%%%%%%%%%%%%%%%%%%%%%%%%%%%%%%%%%%%%%%%%%%%%%%%%%%
%%%%%%%%%%%%%% HEADER %%%%%%%%%%%%%%%%%%%%%%%%%%%%%%%%
%%%%%%%%%%%%%%%%%%%%%%%%%%%%%%%%%%%%%%%%%%%%%%%%%%%%%%

\lhead{\sf \large \Fach{} \\ \small \Name{} - \Matrikelnummer{}}
%\rhead{\sf \Semester{}}
\rhead{\sf \FachKurz{} \quad E \Kurseinheit{}/\AufgabeNummer{}}

%%%%%%%%%%%%%%%%%%%%%%%%%%%%%%%%%%%%%%%%%%%%%%%%%%%%%%
%%%%%%%%%%%%%% START HERE %%%%%%%%%%%%%%%%%%%%%%%%%%%%
%%%%%%%%%%%%%%%%%%%%%%%%%%%%%%%%%%%%%%%%%%%%%%%%%%%%%%

\Aufgabe{\AufgabeNummer{}}

Die Matrix $\pmb{A} \in \mathbb{C}^{n,n}$ ist invertierbar.

\begin{enumerate}[(1)]
    \item \label{aufgabe1}%(1)
Es ist $$\cond\, \pmb{A} = ||\pmb{A}|| \, ||\pmb{A}^{-1}|| = \max\limits_{\pmb{x} \in \mathbb{C}^n: ||\pmb{x}|| = 1} ||\pmb{Ax}|| \, ||\pmb{A}^{-1}||.$$

Es bleibt also zu zeigen, dass
$$||\pmb{A}^{-1}|| = \frac{1}{\min\limits_{\pmb{x} \in \mathbb{C}^n: ||\pmb{x}|| = 1} ||\pmb{Ax}||}.$$


Es ist
\begin{align*}
||\pmb{A}^{-1}|| &= \max\limits_{\pmb{x} \in \mathbb{C}^n \setminus \{\pmb{0}\}} \frac{||\pmb{A}^{-1} \pmb{x}||}{||\pmb{x}||} = \max\limits_{\pmb{Ay} \in \mathbb{C}^n \setminus \{\pmb{0}\}} \frac{||\pmb{y}||}{||\pmb{Ay}||} \\
&= \max\limits_{\pmb{y} \in \mathbb{C}^n \setminus \{\pmb{0}\}} \bigg(\frac{||\pmb{Ay}||}{||\pmb{y}||}\bigg)^{-1} = \bigg(\min\limits_{\pmb{y} \in \mathbb{C}^n \setminus \{\pmb{0}\}} \frac{||\pmb{Ay}||}{||\pmb{y}||}\bigg)^{-1}.
\end{align*}

Mit der Bemerkung 2.4.9 und der Linearität von $\pmb{A}$ folgt
$$||\pmb{A}^{-1}|| = \frac{1}{\min\limits_{\pmb{x} \in \mathbb{C}^n: ||\pmb{x}|| = 1} ||\pmb{Ax}||}.$$

\item %(2)
Die Matrix $\pmb{B} \in \mathbb{C}^{n,n}$ ist nicht invertierbar.

Mit der Definition der Konditionszahl (siehe (\ref{aufgabe1})) erhalten wir aus der Behauptung
$$\frac{||\pmb{A - B}||}{||\pmb{A}||} \geq \frac{1}{||\pmb{A}||\, ||\pmb{A}^{-1}||}.$$

Wir multiplizieren auf beiden Seiten mit $\cond\, \pmb{A}$:
$$||\pmb{A}^{-1}|| \, ||\pmb{A - B}|| \geq 1$$

Ist die Matrixnorm $||\cdot||$ submultiplikativ, so ist dies äquivalent zu
$$1 \leq ||\pmb{A}^{-1}(\pmb{A - B})|| = ||\pmb{I} - \pmb{A}^{-1} \pmb{B}|| \underbrace{= \sup\limits_{\pmb{x} \in \mathbb{C}^n \setminus \{\pmb{0}\}} \frac{||(\pmb{I} - \pmb{A}^{-1} \pmb{B}) \pmb{x}||}{||\pmb{x}||}}_{\textrm{Satz 2.4.6}}.$$

Die Matrix $\pmb{B} \in \mathbb{C}^{n,n}$ ist nicht invertierbar. Es existiert also (mindestens) ein Vektor $\pmb{x} \in \mathbb{C}^n \setminus \{\pmb{0}\}$ mit $\pmb{B x} = \pmb{0}$. Daraus folgt
$$\sup\limits_{\pmb{x} \in \mathbb{C}^n \setminus \{\pmb{0}\}} \frac{||(\pmb{I} - \pmb{A}^{-1} \pmb{B}) \pmb{x}||}{||\pmb{x}||} \geq \sup\limits_{\substack{\pmb{x} \in \mathbb{C}^n \setminus \{\pmb{0}\}\\ \pmb{Bx} = \pmb{0}}} \frac{||(\pmb{I} - \pmb{A}^{-1} \pmb{B}) \pmb{x}||}{||\pmb{x}||} = \sup\limits_{\substack{\pmb{x} \in \mathbb{C}^n \setminus \{\pmb{0}\}\\ \pmb{Bx} = \pmb{0}}} \frac{||\pmb{Ix}||}{||\pmb{x}||} = 1.$$
\end{enumerate}
\end{document}
