\documentclass[a4paper,12pt]{article}
\usepackage{fancyhdr}
\usepackage[ngerman,german]{babel}
\usepackage{german}
\usepackage[utf8]{inputenc}
\usepackage[active]{srcltx}
\usepackage{algorithm}
\usepackage[noend]{algorithmic}
\usepackage{amsmath}
\usepackage{amssymb}
\usepackage{amsthm}
\usepackage{bbm}
\usepackage{enumerate}
\usepackage{graphicx}
\usepackage{ifthen}
\usepackage{listings}
\usepackage{struktex}
\usepackage{hyperref}
\usepackage[onehalfspacing]{setspace}
\usepackage{geometry}
\usepackage{calc}
\usepackage{arydshln}

%%%%%%%%%%%%%%%%%%%%%%%%%%%%%%%%%%%%%%%%%%%%%%%%%%%%%%
%%%%%%%%%%%%%% EDIT THIS PART %%%%%%%%%%%%%%%%%%%%%%%%
%%%%%%%%%%%%%%%%%%%%%%%%%%%%%%%%%%%%%%%%%%%%%%%%%%%%%%

\newcommand{\Fach}{01270 Numerische Mathematik I}
\newcommand{\FachKurz}{NMI}
\newcommand{\Name}{Oliver Schwarz}
\newcommand{\Matrikelnummer}{6883389}
\newcommand{\Semester}{SoSe 23}
\newcommand{\Kurseinheit}{3}
\newcommand{\AufgabeNummer}{5}

%%%%%%%%%%%%%%%%%%%%%%%%%%%%%%%%%%%%%%%%%%%%%%%%%%%%%%
%%%%%%%%%%%%%% DO NOT EDIT THIS PART %%%%%%%%%%%%%%%%%
%%%%%%%%%%%%%%%%%%%%%%%%%%%%%%%%%%%%%%%%%%%%%%%%%%%%%%

\newcommand{\Aufgabe}[1]{
  {
  \vspace*{0.5cm}
  \textbf{Aufgabe #1}
  \vspace*{0.2cm}
  }
}

\newcommand{\cond}[0]{
  {
  \textrm{cond}
  }
}

% Beschreibung
% Unterscheidet im Mathe-Mode zwischen
% "Dezimal-komma" und Satzzeichen.
% Muss bei Eingabe durch kein/ein Leerzeichen
% dahinter angegeben werden.
% Literatur Richard Hirsch in: DTK, 1/1994, S. 42 ff.
\mathchardef\CommaOrdinary="013B \mathchardef\CommaPunct="613B
\mathcode`,="8000 % , im Math-Mode aktiv ("8000) machen
{\catcode`\,=\active \gdef
,{\obeyspaces\futurelet\next\CommaCheck}}
\def\CommaCheck{\if\space\next\CommaPunct\else\CommaOrdinary\fi}


%%%%%%%%%%%%%%%%%%%%%%%%%%%%%%%%%%%%%%%%%%%%%%%%%%%%%%
%%%%%%%%%%%%%% PAGE SETTINGS %%%%%%%%%%%%%%%%%%%%%%%%%
%%%%%%%%%%%%%%%%%%%%%%%%%%%%%%%%%%%%%%%%%%%%%%%%%%%%%%

\setlength{\parindent}{0em}
\topmargin 0cm
\oddsidemargin 0cm
\evensidemargin 0cm

\geometry{%
  left=45.0mm,
  right=15.0mm,
  top=25mm,
  bottom=25mm,
  bindingoffset=0mm,
  headheight=30pt,
  includehead
}

\fancyheadoffset[L]{20mm}
\renewcommand{\headrulewidth}{1pt}

\pagestyle{fancy}

%%%%%%%%%%%%%%%%%%%%%%%%%%%%%%%%%%%%%%%%%%%%%%%%%%%%%%
%%%%%%%%%%%%%% PDF SETTINGS %%%%%%%%%%%%%%%%%%%%%%%%%%
%%%%%%%%%%%%%%%%%%%%%%%%%%%%%%%%%%%%%%%%%%%%%%%%%%%%%%

\hypersetup{
    %pdftitle={\Fach{}: Übungsblatt \Uebungsblatt{}},
    pdftitle={\Fach{}: Kurseinheit \Kurseinheit{} Aufgabe \AufgabeNummer{}},
    pdfauthor={\Name},
    pdfborder={0 0 0}
}

%%%%%%%%%%%%%%%%%%%%%%%%%%%%%%%%%%%%%%%%%%%%%%%%%%%%%%
%%%%%%%%%%%%%% CODE SETTINGS %%%%%%%%%%%%%%%%%%%%%%%%%
%%%%%%%%%%%%%%%%%%%%%%%%%%%%%%%%%%%%%%%%%%%%%%%%%%%%%%

\lstset{ %
language=java,
basicstyle=\footnotesize\tt,
showtabs=false,
tabsize=2,
captionpos=b,
breaklines=true,
extendedchars=true,
showstringspaces=false,
flexiblecolumns=true,
}

%%%%%%%%%%%%%%%%%%%%%%%%%%%%%%%%%%%%%%%%%%%%%%%%%%%%%%
%%%%%%%%%%%%%% DOCUMENT %%%%%%%%%%%%%%%%%%%%%%%%%%%%%%
%%%%%%%%%%%%%%%%%%%%%%%%%%%%%%%%%%%%%%%%%%%%%%%%%%%%%%

\title{Kurseinheit \Kurseinheit{} Aufgabe \AufgabeNummer{}}
\author{\Name{}}

\begin{document}

%\renewcommand{\theequation}{L\Kurseinheit{}.\AufgabeNummer{}.\arabic{equation}}
\renewcommand{\theequation}{L \AufgabeNummer{}.\arabic{equation}}

%%%%%%%%%%%%%%%%%%%%%%%%%%%%%%%%%%%%%%%%%%%%%%%%%%%%%%
%%%%%%%%%%%%%% HEADER %%%%%%%%%%%%%%%%%%%%%%%%%%%%%%%%
%%%%%%%%%%%%%%%%%%%%%%%%%%%%%%%%%%%%%%%%%%%%%%%%%%%%%%

\lhead{\sf \large \Fach{} \\ \small \Name{} - \Matrikelnummer{}}
%\rhead{\sf \Semester{}}
\rhead{\sf \FachKurz{} \quad E \Kurseinheit{}/\AufgabeNummer{}}

%%%%%%%%%%%%%%%%%%%%%%%%%%%%%%%%%%%%%%%%%%%%%%%%%%%%%%
%%%%%%%%%%%%%% START HERE %%%%%%%%%%%%%%%%%%%%%%%%%%%%
%%%%%%%%%%%%%%%%%%%%%%%%%%%%%%%%%%%%%%%%%%%%%%%%%%%%%%

\Aufgabe{\AufgabeNummer{}}
\begin{enumerate}[a)]
    \item %a
Mit dem \textsc{Cholesky}-\textsc{Crout}-Verfahren erhalten wir
\begin{align*}
\pmb{L} =
\begin{pmatrix}
1 & 0 & 0 & 0\\
2+i & 1 & 0 & 0\\
1-i & 2-i & 1 & 0\\
-3+2i & 2i & 1-i & 1
\end{pmatrix}
\end{align*}

\item %b)
Wir wenden das \textsc{Gauss}sche Eliminationsverfahren an und erhalten    
\begin{align*}
    \begin{array}[t]{ r  r  r  r : r}
x_1 & x_2 & x_3 & x_4 & b \\\cline{1-5}
1 & 2-i & 1+i & -3-2i & -1+4i\\
0 & 1 & 2+i & -2i & -4\\
0 & 0 & 1 & 1+i & -i\\
0 & 0 & 0 & 1 & -i\\%\cline{1-5}
\end{array}
\end{align*}
Rückwärtsersetzung ergibt die Lösung $\pmb{x} = (1, i, -1, -i)^T$.

\item %c)
Es ist $\pmb{A} \pmb{A}^{-1} = \pmb{I}^4$. Hierbei entspricht $e^k$, $k=1, 2, 3, 4$ dem $k$-ten Spaltenvektor von $\pmb{I}^4$. Dieser wird gebildet aus den Zeilen von $\pmb{A}$ und dem $k$-ten Spaltenvektor von $\pmb{A}^{-1}$. $\pmb{x}^1$, $\pmb{x}^2$, $\pmb{x}^3$ und $\pmb{x}^4$ sind also gerade die Spaltenvektoren von $\pmb{A}^{-1}$. Diese wird in Teilaufgabe \ref{aufgabe_d} berechnet. Also:

\begin{equation*}
    \pmb{x}^1 =
    \begin{pmatrix}
        64 \\ -38-22i \\ 13 \\ -4+5i
    \end{pmatrix}, \;
    \pmb{x}^2 =
    \begin{pmatrix}
        -38 + 22i \\ 32 \\ -8+5i \\ 1-5i
    \end{pmatrix}, \;
    \pmb{x}^3 =
    \begin{pmatrix}
        13 \\ -8-5i \\ 3 \\ -1+i
    \end{pmatrix}, \;
    \pmb{x}^4 =
    \begin{pmatrix}
        -4-5i \\ 1+5i \\ -1-i \\ 1
    \end{pmatrix}.
\end{equation*}

\item \label{aufgabe_d} %d)
Es ist $\pmb{A}^{-1} = (\pmb{L}\pmb{L}^H)^{-1} = (\pmb{L}^{-1})^H \pmb{L}^{-1}$. Wir bestimmen $\pmb{L}^{-1}$ mit dem \textsc{Gauss}-\textsc{Jordan}-Algorithmus.

\begin{align*}
\left(
\begin{matrix}
1 & 0 & 0 & 0\\
2+i & 1 & 0 & 0\\
1-i & 2-i & 1 & 0\\
-3+2i & 2i & 1-i & 1
\end{matrix}
\quad
\middle|
\quad
\begin{matrix}
1 & 0 & 0 & 0\\
0 & 1 & 0 & 0\\
0 & 0 & 1 & 0\\
0 & 0 & 0 & 1
\end{matrix}
\right)
\quad
\begin{matrix}
\cdot (2+i) & \cdot (1-i) & \cdot (-3+2i) \\
& & \\
& & \\
& &
\end{matrix}
\end{align*}

\begin{align*}
\left(
\begin{matrix}
1 & 0 & 0 & 0\\
0 & 1 & 0 & 0\\
0 & 2-i & 1 & 0\\
0 & 2i & 1-i & 1
\end{matrix}
\quad
\middle|
\quad
\begin{matrix}
1 & 0 & 0 & 0\\
-2-i & 1 & 0 & 0\\
-1+i & 0 & 1 & 0\\
3-2i & 0 & 0 & 1
\end{matrix}
\right)
\quad
\begin{matrix}
& & \\
\cdot (2-i) & \cdot (2i) &  \\
& & \\
& &
\end{matrix}
\end{align*}

\begin{align*}
\left(
\begin{matrix}
1 & 0 & 0 & 0\\
0 & 1 & 0 & 0\\
0 & 0 & 1 & 0\\
0 & 0 & 1-i & 1
\end{matrix}
\quad
\middle|
\quad
\begin{matrix}
1 & 0 & 0 & 0\\
-2-i & 1 & 0 & 0\\
4+i & -2+i & 1 & 0\\
1+2i & -2i & 0 & 1
\end{matrix}
\right)
\quad
\begin{matrix}
& & \\
&  &  \\
\cdot (1-i) & & \\
& &
\end{matrix}
\end{align*}

\begin{align*}
\left(
\begin{matrix}
1 & 0 & 0 & 0\\
0 & 1 & 0 & 0\\
0 & 0 & 1 & 0\\
0 & 0 & 0 & 1
\end{matrix}
\quad
\middle|
\quad
\begin{matrix}
1 & 0 & 0 & 0\\
-2-i & 1 & 0 & 0\\
4+i & -2+i & 1 & 0\\
-4+5i & 1-5i & -1+i & 1
\end{matrix}
\right)
\end{align*}

\begin{align*}
\pmb{L}^{-1} =
\begin{pmatrix}
1 & 0 & 0 & 0\\
-2-i & 1 & 0 & 0\\
4+i & -2+i & 1 & 0\\
-4+5i & 1-5i & -1+i & 1
\end{pmatrix}
\end{align*}

Damit ist
\begin{align*}
\pmb{A}^{-1} &=
\begin{pmatrix}
1 & -2-i & 4+i & -4+5i\\
0 & 1 & i-2 & 1-5i\\
0 & 0 & 1 & i-1\\
0 & 0 & 0 & 1
\end{pmatrix}
\begin{pmatrix}
1 & 0 & 0 & 0\\
-2-i & 1 & 0 & 0\\
4+i & -2+i & 1 & 0\\
-4+5i & 1-5i & -1+i & 1
\end{pmatrix}\\
&=
\begin{pmatrix}
64 & -38+22i & 13 & -4-5i\\
38-22i & 32 & -8-5i & 1+5i\\
13 & -8+5i & 3 & -1-i\\
-4+5i & 1-5i & -1+i & 1
\end{pmatrix}.
\end{align*}

\item %e
$\pmb{L}$ ist eine normierte Dreiecksmatrix und $\det\pmb{A} = \det(\pmb{LL}^H) = \det\pmb{L} \det\pmb{L}^H = 1$.
\end{enumerate}
\end{document}
